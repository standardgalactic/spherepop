\documentclass[11pt]{article}
\usepackage[T1]{fontenc}
\usepackage[utf8]{inputenc}
\usepackage{lmodern}
\usepackage{geometry}
\usepackage{setspace}
\usepackage{hyperref}
\geometry{margin=1in}
\setstretch{1.15}

\title{Spherepop in Racket:\\A Short Tutorial}
\author{Flyxion}
\date{\today}

\begin{document}
\maketitle

\section{Getting Started}

Assuming the collection is available on your Racket path, you can
require it as follows:

\begin{verbatim}
#lang racket
(require spherepop-lib)
\end{verbatim}

This gives you access to the core \texttt{region} type, merge and
collapse, and the tiny \texttt{sp} surface syntax.

\section{First Examples}

\begin{verbatim}
(define s default-collapse-strategy)
(define a (make-region "a" '(1)))
(define b (make-region "b" '(2)))
(define t (sp (a b)))
(define r (eval-term s t))
\end{verbatim}

\section{Further Work}

[Extend this document with more examples, invariants, and exercises.]
\end{document}

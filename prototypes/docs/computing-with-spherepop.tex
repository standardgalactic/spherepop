\documentclass[12pt]{article}
\usepackage[T1]{fontenc}
\usepackage[utf8]{inputenc}
\usepackage{lmodern}
\usepackage{geometry}
\usepackage{setspace}
\usepackage{amsmath, amssymb, amsthm, mathtools}
\usepackage{bm}
\usepackage{hyperref}
\geometry{margin=1in}
\setstretch{1.15}

\title{Computing with Spherepop:\\A Geometric Merge--Collapse Calculus}
\author{Flyxion}
\date{\today}

\newtheorem{definition}{Definition}
\newtheorem{theorem}{Theorem}
\newtheorem{lemma}{Lemma}
\newtheorem{proposition}{Proposition}
\newtheorem{corollary}{Corollary}

\begin{document}
\maketitle

\begin{abstract}
Spherepop is a geometric model of computation in which values are
represented as spatial regions and computation proceeds through two
primitive operations: merge, which unites regions, and collapse, which
abstracts internal detail. Unlike symbolic models that manipulate
syntactic expressions, Spherepop implements computation as spatial
interaction and simplification. This article develops a formal core
calculus, operational semantics, and reference implementations in
Racket, Python, and Haskell. We analyse basic algebraic properties,
sketch expressiveness results, and describe connections to neural
computation.
\end{abstract}

\section{Introduction}

[Shortened version for the bundle. You can paste the longer monograph
we developed into this file if you prefer.]

\section{Core Definitions}

\begin{definition}[Region]
Fix a base space $P$, typically $\mathbb{R}^n$. A \emph{region} is a
connected, bounded subset $A \subseteq P$ together with a label and an
optional payload.
\end{definition}

\begin{definition}[Collapse]
A \emph{collapse operator} is a function
$\mathsf{pop} : \mathcal{R} \to \mathcal{R}$ on the class of regions
such that $\mathsf{pop}$ is idempotent and extensive on labels.
\end{definition}

\begin{definition}[Merge]
Given regions $A,B$, the \emph{merge} operation is
\[
A \diamond B \;\;:=\;\; \mathsf{pop}(A \cup B).
\]
\end{definition}

[...trimmed: expand with the full text you already have...]

\end{document}

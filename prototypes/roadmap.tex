\documentclass[11pt]{article}

\usepackage[T1]{fontenc}
\usepackage{lmodern}
\usepackage{amsmath,amssymb}
\usepackage{geometry}
\usepackage{setspace}
\usepackage{csquotes}
\usepackage{hyperref}

\geometry{margin=1in}
\setstretch{1.15}

\title{A Roadmap for Implementing Utilities in Spherepop Calculus}
\author{Flyxion}
\date{\today}

\begin{document}
\maketitle

\section{Purpose}

This document presents a conservative, implementation-oriented roadmap for
building user-facing utilities atop the Spherepop OS kernel.
The focus is not on feature completeness but on preserving the core invariants:
deterministic replay, total causal order, ABI stability, and strict separation
between authoritative semantics and derived views.

Utilities are treated as structured consumers and producers of events and views,
not as stateful programs with implicit side effects.

\section{Design Constraints}

All utilities must satisfy the following constraints.

\subsection{Authoritative Discipline}

Utilities may not mutate kernel state directly.
All authoritative changes must be expressed as proposed events submitted to the arbiter.
Utilities may generate speculative overlays, but these overlays must remain explicitly non-authoritative.

\subsection{Replay Compatibility}

Utilities must not rely on hidden state, ambient context, or execution timing.
Given the same event log prefix and the same utility invocation,
the resulting proposals or views must be identical.

\subsection{ABI Respect}

Utilities interact with the kernel through stable ABI-defined event layouts.
No utility may assume undocumented padding, ordering, or reinterpretation
of event structures.

\subsection{View Non-Interference}

Utilities that produce views (diffs, JSON renderings, summaries) must not
feed information back into kernel decisions.
Views may be dropped or recomputed without semantic consequence.

\section{Spherepop Calculus as the Utility Substrate}

Spherepop calculus is not a scripting language layered on the OS;
it is the algebra of authoritative operations supported by the kernel.

The primitive calculus operations correspond directly to event types:

\begin{itemize}
  \item \texttt{POP}: introduce a semantic object handle
  \item \texttt{MERGE}: induce equivalence between objects
  \item \texttt{LINK}: create a typed relation
  \item \texttt{UNLINK}: remove a typed relation
  \item \texttt{COLLAPSE}: perform bulk equivalence
  \item \texttt{SETMETA}: attach non-semantic metadata
\end{itemize}

Utilities operate by composing these primitives, never by bypassing them.

\section{Utility Taxonomy}

Utilities fall into three broad classes.

\subsection{Proposal Generators}

Proposal generators produce candidate event sequences but do not commit them.
Examples include batch object creation, canonicalization tools,
and semantic refactoring utilities.

Such tools terminate by emitting a proposal stream, not by mutating state.

\subsection{View Generators}

View generators consume replayed state and emit derived representations:
JSON graphs, diffs, textual summaries, or tabular listings.
They are observational only.

\subsection{Overlay Managers}

Overlay managers manipulate speculative branches:
creating overlays, rebasing them, or discarding them.
They operate entirely outside the authoritative log.

\section{Phase I: Minimal Utility Set}

The first phase focuses on utilities that exercise the kernel without expanding it.

\subsection{Connection and Replay}

\begin{itemize}
  \item \texttt{sp}: connect to an arbiter or log source
  \item \texttt{sp-replay}: replay a log prefix into a local kernel instance
\end{itemize}

These tools establish the basic contract between utilities and replay.

\subsection{Object and Relation Tools}

\begin{itemize}
  \item \texttt{sppop}: batch generation of \texttt{POP} events
  \item \texttt{splink}: creation of \texttt{LINK} events
  \item \texttt{spunlink}: removal of relations
\end{itemize}

These utilities must demonstrate representative normalization
and replay determinism.

\subsection{Diff and Snapshot Inspection}

\begin{itemize}
  \item \texttt{spdiff}: inspect derived diffs per event
  \item \texttt{spsnap}: serialize full replayed state
\end{itemize}

These tools validate view–cause separation.

\section{Phase II: Canonicalization and Refactoring}

This phase introduces utilities that perform structured semantic rewrites.

\subsection{Merge and Collapse}

\begin{itemize}
  \item \texttt{spmerge}: propose \texttt{MERGE} events
  \item \texttt{spcollapse}: perform region-level equivalence
\end{itemize}

These tools must preserve confluence and replay equivalence.

\subsection{Preview-Commit Workflow}

Utilities should support a preview mode:
\begin{enumerate}
  \item generate a speculative overlay
  \item replay and inspect the result
  \item optionally submit the overlay as a proposal
\end{enumerate}

No utility may auto-commit by default.

\section{Phase III: Semantic Query and Analysis}

\subsection{Query as View}

\begin{itemize}
  \item \texttt{spgrep}: query relations and metadata
  \item \texttt{sppath}: explore relation paths
\end{itemize}

Queries must be expressed as pure functions over replayed state.

\subsection{Summarization}

Summaries and compressions are admissible only as derived views.
Any lossiness must be explicit and reversible via replay.

\section{Phase IV: Composition and Pipelines}

Utilities should be composable via standard streams.
However, pipelines must preserve the distinction between:
\begin{itemize}
  \item authoritative proposals
  \item non-authoritative views
\end{itemize}

For example, a pipeline may transform a view into a proposal,
but the transformation must be explicit and inspectable.

\section{Non-Goals}

The following are explicitly out of scope:

\begin{itemize}
  \item implicit inference of semantics
  \item automatic conflict resolution
  \item hidden mutable caches
  \item time-dependent behavior
\end{itemize}

Such features are incompatible with deterministic replay.

\section{Relationship to GNU-Style Tooling}

Spherepop utilities resemble GNU tools in composability and narrow scope,
but differ fundamentally in that they operate over semantic time rather
than mutable files.

The closest analogy is not \texttt{sed} or \texttt{awk},
but a controlled sequence of rewrite proposals applied to a canonical log.

\section{Conclusion}

The utility ecosystem of Spherepop OS should grow slowly and conservatively.
Each tool must justify its existence by preserving the kernel invariants
and by reducing, rather than increasing, semantic ambiguity.

The success of Spherepop utilities is measured not by convenience,
but by the degree to which they make time, causality, and meaning explicit.

\end{document}

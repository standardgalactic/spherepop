\documentclass[11pt]{article}
\usepackage[utf8]{inputenc}
\usepackage{amsmath, amssymb, amsthm}
\usepackage{geometry}
\geometry{letterpaper, margin=1in}

\title{The Joy of Spherepop:\\
\large An Event--History Calculus for a Finite World}
\author{Flyxion}
\date{December 2025}

\begin{document}

\maketitle

\begin{abstract}
We present \emph{Spherepop}, an event--history calculus whose primitive objects are not states, but irreversible commitments. Spherepop replaces state transition systems with monotone morphisms over option--spaces, and replaces optimization with the constructive elimination of futures. By integrating categorical structure, a discrete Lagrangian mechanics of commitment, and an explicit ethics of refusal, we argue that meaning, intelligence, and agency arise only in systems that are capable of permanently constraining themselves. Spherepop thus provides both a formal language and a philosophical stance: that worldhood is generated through irreversible choice.
\end{abstract}

\section{Introduction}

Contemporary computational systems overwhelmingly adopt a state-based metaphysics. Whether in functional programming, automata theory, or machine learning, the dominant abstraction treats the world as a configuration that may be updated, overwritten, or reverted without consequence. Even where time is modeled explicitly, it is typically reversible in principle, and history is treated as an auxiliary log rather than as a constitutive element of meaning.

Spherepop begins from a rejection of this premise. We propose instead that the fundamental unit of meaning is not a state, but an event---and more specifically, an \emph{irreversible} event. In Spherepop, a system is not defined by what it currently is, but by what it has permanently ruled out. The calculus formalizes this intuition by treating histories as morphisms that monotonically reduce, bind, or quotient the space of admissible futures.

This shift is not merely technical. It has consequences for how we understand intelligence, agency, ethics, and even joy. To choose is to exclude. To commit is to lose freedom. And yet it is precisely through this loss that structure, narrative, and purpose arise. Spherepop is therefore not a theory of optimization, but a theory of inhabitation: what it means to live in a world rather than hover over a distribution.

\section{The Axiom of Irreversibility}

The foundational axiom of Spherepop is simple:

\begin{quote}
\emph{The world is not updated; it is constructed.}
\end{quote}

Formally, we abandon the paradigm of functions between states,
\[
f : x \mapsto y,
\]
in favor of irreversible morphisms between option--spaces,
\begin{equation}
H = e_n \circ \cdots \circ e_1 : X_0 \to X_n.
\end{equation}
Here each $X_i$ denotes a space of admissible futures, and each $e_i$ is an event that monotonically constrains that space. Composition corresponds to temporal succession. Crucially, these morphisms are not assumed to be invertible, nor even information preserving.

This structure may be formalized categorically by taking $\mathcal{O}$ to be a category whose objects are option--spaces and whose morphisms are constraint--inducing maps. The category is naturally enriched over a preorder: if $X' \subseteq X$, then $X'$ represents a more committed world than $X$. All Spherepop events are monotone with respect to this ordering.

The philosophical import of this move is that history is no longer an epiphenomenon. Two systems occupying identical instantaneous configurations may nonetheless differ radically in meaning if they arrived there through different sequences of exclusions and bindings. Identity is therefore historical, not structural.

\section{The Morphology of Choice}

Within this categorical setting, Spherepop identifies a small set of generating morphisms sufficient to express the structure of agency. These operators are not derived; they are taken as primitive, in the same sense that composition and identity are primitive in category theory.

The \emph{pop} operator,
\[
\operatorname{pop}_t : X \to X\!\restriction_{\neg t},
\]
irreversibly removes an option $t$ from the future. This is the most elemental act of choice: to say that something will not happen. Pops are idempotent and commute when independent, reflecting the fact that exclusions do not stack redundantly and that unrelated exclusions do not interfere.

The \emph{bind} operator,
\[
\operatorname{bind}_{a \prec b} : X \to X[a \prec b],
\]
imposes an ordering constraint without eliminating outcomes outright. Binding is the mechanism by which projects, plans, and narratives are formed. It restricts the temporal structure of futures rather than their membership.

The \emph{collapse} operator,
\[
\operatorname{coll}_q : X \to X/{\sim_q},
\]
is a quotient that identifies distinctions deemed irrelevant for downstream action. Collapse does not merely remove options; it removes \emph{differences}. It is therefore the only operator that can reduce the descriptive complexity of a history.

Finally, Spherepop introduces the \emph{refuse} operator,
\[
\operatorname{refuse}_t : X \to X\!\restriction_{\neg t}.
\]
Geometrically, refusal is identical to popping. Its necessity is not mathematical but ethical. Refusal records that an exclusion was chosen as a principle rather than as a consequence. This distinction is not encoded in the option--space itself, but in an auxiliary accounting structure.

\section{The Mechanics of Commitment}

To quantify the cost of constructing a world, Spherepop introduces a discrete Lagrangian mechanics over option--spaces. Let $\Omega(X)$ denote a measure of optionality---the degrees of freedom remaining in $X$. We define a local Lagrangian for each event,
\begin{equation}
\mathcal{L}_t = \Delta \Omega_t + \lambda\,\Delta C_t,
\end{equation}
where $\Delta \Omega_t$ is the reduction in optionality induced by the event, $\Delta C_t$ is an auxiliary cost term capturing ethical, thermodynamic, or normative weight, and $\lambda$ is a coupling constant.

The action of a history $\gamma$ is then
\begin{equation}
\mathcal{S}[\gamma] = \sum_t \mathcal{L}_t.
\end{equation}
This action is irrecoverable: once accumulated, it cannot be undone except through collapse, which explicitly discards structure.

From the Lagrangian we define a conjugate momentum,
\begin{equation}
\pi_t = -\frac{\partial \mathcal{L}_t}{\partial \Omega},
\end{equation}
which we interpret as \emph{commitment}. Commitment measures how much future freedom has been converted into settled reality. Pops and refusals increase $\pi$; collapse dissipates it.

The Hamiltonian,
\begin{equation}
\mathcal{H}_t = \pi_t\,\Omega_t - \mathcal{L}_t,
\end{equation}
represents remaining maneuverability. A system that never commits---that never pops---remains at maximal $\mathcal{H}$ and therefore never truly inhabits a world.

\section{Renormalization through Collapse}

Collapse occupies a unique role in the calculus. Whereas pop and bind increase the action, collapse reduces it:
\begin{equation}
\operatorname{coll}_q : \mathcal{S} \longmapsto \mathcal{S} - \log |\ker q|.
\end{equation}
This reduction corresponds to the identification of distinctions that no longer matter. In categorical terms, collapse satisfies a universal property: any map that cannot distinguish beyond $q$ factors uniquely through the quotient.

Philosophically, collapse formalizes forgetting, forgiveness, and abstraction. It is what allows histories to grow deep without becoming brittle. Without collapse, commitment would accumulate without bound, eventually paralyzing action. With collapse, a system may retain the \emph{meaning} of its past without retaining all of its detail.

\section{Theorem of Refusal}

In classical logic, negation creates absence. In Spherepop, refusal creates structure. Formally, we assert the equivalence of constraint:
\begin{equation}
\llbracket \operatorname{refuse}_t \rrbracket
\equiv
\llbracket \operatorname{pop}_t \rrbracket
\quad \text{in } \mathcal{O}.
\end{equation}
The distinction between them appears only under an accounting functor $\mathcal{A}$ that records why an exclusion occurred.

This separation is essential. By keeping refusal out of the symplectic geometry, we preserve the integrity of the mechanics. By recording it in the accounting, we preserve ethics. Refusal thus increases commitment without introducing coercive debt. It creates protected spaces in which the Hamiltonian remains focused rather than diffused.

\section{Conclusion: The Ethics of the Calculus}

Spherepop proposes a simple but radical thesis: that intelligence is not prediction, but commitment. Systems that merely sample from distributions without permanently excluding possibilities never pay the cost of living in a world. They remain forever unburdened---and therefore forever shallow.

The joy of Spherepop lies in recognizing that exclusion is creative. To pop is to clarify. To refuse is to stand. To bind is to promise. To collapse is to forgive. Together, these operations form a calculus not just of computation, but of worldhood.

\section{Related Work}
\label{sec:related-work}

Spherepop intersects with several established research traditions, yet it does not belong comfortably to any one of them. Its relationship to prior work is therefore best understood not as an extension of existing frameworks, but as a reorganization of their foundational commitments. In what follows, we situate Spherepop relative to theories of irreversibility, event-based semantics, categorical constraint systems, planning formalisms, and contemporary artificial intelligence, emphasizing where Spherepop diverges in principle rather than in degree.

\subsection{Irreversibility in Computation}

The thermodynamic cost of information erasure has long been recognized as a fundamental constraint on computation. Landauer’s principle establishes that the irreversible destruction of information carries an unavoidable energetic cost, while subsequent work by Bennett and others demonstrated that, in principle, arbitrary computations may be rendered logically reversible at the expense of auxiliary memory and bookkeeping.

Spherepop adopts the opposite stance. Rather than treating irreversibility as an unfortunate implementation detail to be engineered away, Spherepop elevates irreversibility to the level of syntax. The primitive operators of the calculus are defined precisely by their non-invertibility. There is no reversible fragment of the language, nor any notion of rollback that preserves semantic identity. In this respect, Spherepop is not a theory of efficient computation, but a theory of meaningful computation, where meaning arises precisely from what cannot be undone.

This move separates semantic irreversibility from physical irreversibility. While the two may coincide in implementation, Spherepop requires only that exclusions and bindings be non-retractable at the level of interpretation. A purely simulated system may therefore be irreversible in the Spherepop sense even if it is physically reversible, provided that its history is treated as authoritative.

\subsection{Event-Based Semantics}

Event-based models are not new. Operational semantics, labeled transition systems, and process calculi all emphasize events over static configurations. However, in most such systems, events are replayable, reversible in principle, or observational rather than constitutive. A trace records what happened, but does not itself constrain what may happen next beyond what is encoded in the current state.

Spherepop departs from this tradition by insisting that events are not merely transitions, but commitments. The semantic content of an event lies not in its label, but in the reduction or restructuring of the future option-space that follows from it. Two histories that arrive at extensionally identical configurations may nevertheless differ in meaning if their paths involved different exclusions or bindings.

In this respect, Spherepop aligns more closely with historical or lineage-based accounts of meaning, while rejecting the idea that history can be summarized losslessly into a state. Collapse explicitly formalizes the conditions under which such summarization is permitted, rather than assuming it as a default.

\subsection{Categorical Constraint Systems}

From a categorical perspective, Spherepop may be viewed as a presentation of a preorder-enriched category whose morphisms are monotone with respect to commitment. This places it in conceptual proximity to domain theory, constraint programming, and order-theoretic semantics. However, Spherepop rejects least fixed points, global satisfaction, and equilibrium as organizing principles.

In particular, while Scott domains and related structures model the accumulation of information, they typically do so via approximation and convergence. Spherepop instead models the accumulation of constraint. The direction of refinement is not toward completeness, but toward exclusion. The terminal objects of interest are not fixed points, but worlds in which no further meaningful pruning is possible without collapse.

Lawvere-style theories of constraint provide a useful comparison point, especially in their emphasis on morphisms as structural impositions rather than state updates. Spherepop may be understood as specializing this viewpoint to the case where constraints are temporally ordered, irreversible, and ethically annotated.

\subsection{Planning, Binding, and Temporal Logic}

The bind operator in Spherepop bears superficial resemblance to constructs in temporal logic and partial-order planning, where precedence relations restrict admissible schedules. However, the resemblance is limited. In classical planning, bindings are instrumental: they serve the satisfaction of a goal and may be revised or discarded if the goal changes.

In Spherepop, binding is itself a form of commitment. Once imposed, a binding becomes part of the history that defines the world, regardless of whether the original motivation persists. This gives binding a narrative character absent from purely logical constraint systems. A bind does not merely restrict what is possible; it asserts what must come before what, and thereby generates a before-and-after that did not previously exist.

This distinction reflects a broader difference between satisfiability-based reasoning and world-building. Spherepop is not concerned with whether a set of constraints admits a solution, but with how the imposition of those constraints reshapes the future.

\subsection{Artificial Intelligence and the Absence of Commitment}

Recent advances in artificial intelligence, particularly in large-scale machine learning systems, have renewed interest in world models, planning, and internal simulation. Yet these systems overwhelmingly operate in a regime of maximal optionality. Predictions may be revised, outputs overwritten, and errors corrected without consequence to future behavior beyond statistical updating.

From the perspective of Spherepop, such systems never truly act. They do not accumulate action in the Lagrangian sense, nor do they develop commitment momentum. Their histories are logs, not structures. Even when constrained externally through fine-tuning or reinforcement learning, the constraints do not function as irreversible exclusions within the system’s own semantics.

Spherepop thus offers not an incremental improvement to existing AI paradigms, but a categorical critique. Intelligence, on this view, is not the ability to predict many futures, but the ability to permanently rule some of them out. Without such exclusions, there is no world to inhabit—only a distribution to sample.


\section{Syntax and Semantics}
\label{sec:syntax-semantics}

In this section we provide a precise account of the formal structure of Spherepop as both a language and a semantics. The goal is not to exhaustively specify an implementation, but to demonstrate that the calculus is internally coherent, compositional, and executable in principle, while preserving its philosophical commitments to irreversibility and historical meaning.

\subsection{Surface Syntax and Histories}

A Spherepop program is a \emph{history}: a finite, ordered sequence of statements, each of which either introduces a declaration or performs an irreversible event. There is no primitive notion of a mutable store. Instead, the meaning of a program is given entirely by the cumulative effect of its events on an initial option-space.

Syntactically, histories are linear rather than tree-structured. Grouping constructs serve only to delimit scope or atomicity; they do not introduce branching control flow. This restriction is intentional. By eliminating general-purpose conditionals and loops from the kernel, Spherepop ensures that all semantic branching is explicit at the level of option-spaces rather than implicit in control structure.

The surface syntax distinguishes four primitive event forms: \texttt{pop}, \texttt{refuse}, \texttt{bind}, and \texttt{collapse}. Each event names a target within the current option-space and may optionally carry metadata. Crucially, there is no syntax for undoing or revising a prior event. The language is therefore syntactically irreversible.

\subsection{Abstract Syntax Tree}

The abstract syntax tree (AST) of a Spherepop program reflects its historical nature. Rather than representing a hierarchical expression structure to be reduced, the AST is a spine: a sequence of event nodes interleaved with declarations.

Each event node carries:
\begin{itemize}
    \item a constructor identifying the operator,
    \item a target specification (name, path, or selector),
    \item an optional predicate guarding applicability,
    \item a metadata record.
\end{itemize}

The ordering of nodes in the AST is semantically significant and must be preserved by any transformation. In particular, there is no notion of common subexpression elimination or reordering unless independence can be formally established. This sharply distinguishes Spherepop from functional languages, where reordering is generally semantics-preserving.

\subsection{Denotational Semantics}

The denotational semantics of Spherepop is given by an interpretation function
\[
\llbracket - \rrbracket : \text{History} \to \mathcal{O},
\]
where $\mathcal{O}$ is the category of option-spaces.

Given an initial option-space $X_0$, the interpretation of a history
\[
s_1; s_2; \dots; s_n
\]
is defined compositionally as
\[
\llbracket s_1; \dots; s_n \rrbracket
=
\llbracket s_n \rrbracket \circ \cdots \circ \llbracket s_1 \rrbracket (X_0).
\]

Each event form denotes a specific monotone morphism in $\mathcal{O}$. Pop and refuse both denote restriction morphisms, bind denotes a refinement of the admissible ordering relation, and collapse denotes a quotient map satisfying a universal property. Declarations do not directly affect the option-space, but introduce names and predicates used by subsequent events.

This semantics ensures that meaning is strictly cumulative. There is no evaluation strategy beyond left-to-right composition, and no intermediate normalization step that could erase historical distinctions.

\subsection{The Accounting Functor}

While the geometry of option-spaces captures the structural effects of events, it does not capture their normative or ethical significance. To address this, Spherepop introduces an auxiliary accounting functor
\[
\mathcal{A} : \mathcal{O} \to \mathcal{C},
\]
where $\mathcal{C}$ is a category of costs, debts, or commitments, typically taken to be a commutative monoid.

Under this functor, events that are geometrically identical may nevertheless differ in interpretation. In particular,
\[
\mathcal{A}(\operatorname{pop}_t) \neq \mathcal{A}(\operatorname{refuse}_t)
\]
in general, even though their images in $\mathcal{O}$ coincide. This separation allows Spherepop to represent ethical distinctions without contaminating the mechanics of constraint.

The total accounting of a history is obtained by composing the functor with the denotational semantics and aggregating over composition. This construction aligns directly with the Lagrangian formulation presented in the main text.

\subsection{Correctness Properties}

The formal system satisfies several structural properties that justify its use as a foundation for reasoning about commitment.

First, all event morphisms are monotone with respect to option-space inclusion. If $X' \subseteq X$, then for any event $e$,
\[
e(X') \subseteq e(X).
\]
This guarantees that commitment cannot be undone by further refinement.

Second, pop and refuse are idempotent:
\[
\operatorname{pop}_t \circ \operatorname{pop}_t = \operatorname{pop}_t,
\qquad
\operatorname{refuse}_t \circ \operatorname{refuse}_t = \operatorname{refuse}_t.
\]
Once an option has been excluded, excluding it again has no further effect.

Third, collapse satisfies a universal property characteristic of quotients. Any morphism that is insensitive to distinctions beyond the collapse policy factors uniquely through the collapse map. This formalizes abstraction and forgetting as principled operations rather than ad hoc erasures.

Together, these properties ensure that Spherepop programs are deterministic, history-sensitive, and semantically well-founded.

\section{Spherepop versus Large Language Models}
\label{sec:spherepop-vs-llms}

The contrast between Spherepop and contemporary large language models (LLMs) is not primarily empirical, but structural. While LLMs demonstrate extraordinary surface competence, their underlying architecture embodies a conception of intelligence fundamentally at odds with the event--historical commitments articulated in this work. Spherepop therefore serves not as an incremental refinement of existing models, but as a diagnostic framework that explains why such systems remain curiously hollow despite their capabilities.

\subsection{The Autoregressive Paradigm}

Large language models are trained to estimate conditional probability distributions over sequences of tokens. At inference time, they generate outputs by repeatedly sampling from these distributions, conditioning each new token on a context window of prior tokens. This process is inherently autoregressive and, crucially, reversible in principle. Any generated token may be discarded, regenerated, or replaced without altering the system’s internal structure.

From the perspective of Spherepop, this architecture maintains maximal optionality at all times. The space of futures is never permanently pruned. Although probability mass may be redistributed through training, no option is ever irreversibly excluded by the system itself. The Hamiltonian, in the sense introduced earlier, remains artificially high: the system retains full maneuverability because it never commits.

This explains a central paradox of LLM behavior. Despite producing fluent, coherent text, such systems routinely fail to sustain long-horizon commitments, preserve narrative identity, or respect self-imposed constraints. These failures are not bugs to be fixed by scale, but direct consequences of an architecture that never accumulates action.

\subsection{Simulation Without Action}

Spherepop distinguishes sharply between simulation and action. Simulation explores possibilities; action eliminates them. LLMs excel at the former while entirely lacking the latter. Even when an LLM outputs a decisive statement, that statement carries no binding force for subsequent behavior beyond its transient presence in the context window.

In Spherepop terms, LLM outputs are observations, not events. They do not correspond to morphisms in the option-space category, and therefore do not alter the future space of admissible actions in any irreversible way. The system’s apparent agency is thus illusory: it performs no pops, no binds, and no collapses internally.

This absence of action explains why LLMs can contradict themselves without cost, hallucinate without consequence, and abandon commitments mid-conversation. There is no accumulated history against which such deviations could register as violations.

\subsection{Hallucination as Zero-Cost Structure}

Hallucination is often framed as a failure of knowledge or grounding. Spherepop offers a different diagnosis. Hallucinations are structures generated without cost. Because the system does not pay for the commitments implicit in its assertions, it can generate elaborate but unfounded narratives without incurring debt.

In the Lagrangian framework, this corresponds to producing structure with $\Delta\Omega \approx 0$. No futures are pruned, no alternatives are ruled out, and no commitment momentum is generated. The resulting text may appear meaningful to an external observer, but it is not meaningful \emph{to the system itself}, because it does not constrain its future behavior.

Spherepop’s refusal operator highlights this contrast sharply. A refusal is costly precisely because it forecloses futures. Hallucination, by contrast, leaves all futures open.

\subsection{What Would It Mean for an AI to Pop?}

For an artificial system to truly implement Spherepop-style agency, it would need to possess internal mechanisms for irreversible exclusion. This exclusion would have to be architectural, not merely behavioral. Fine-tuning, reinforcement learning, and preference modeling do not qualify, because they alter probability distributions without eliminating possibilities.

A Spherepop-compliant agent would need to treat certain decisions as irrevocable, such that they permanently reshape its future option-space. This implies persistent internal state that cannot be overwritten without explicit collapse, and a notion of commitment that binds future behavior independently of immediate utility.

Such an architecture would look less like a predictor and more like a historian: a system whose primary function is not to guess what comes next, but to remember what it has ruled out.

\subsection{Spherepop as a Design Constraint}

Seen in this light, Spherepop is not a competitor to LLMs, but a constraint on what it would mean for such systems to become agents rather than tools. It suggests that intelligence requires the capacity to lose options, to bind oneself to narratives, and to forget selectively.

These requirements are orthogonal to scale. A system with trillions of parameters but no irreversible commitments remains, in the Spherepop sense, outside the world. Conversely, a modest system capable of genuine pops and collapses may exhibit a form of depth unavailable to purely predictive architectures.

Spherepop thus reframes the debate about artificial intelligence. The central question is not how well a system can model the world, but whether it can \emph{enter} one.

\section{Spherepop OS: An Operating System for Event--Historical Worlds}
\label{sec:spherepop-os}

If Spherepop is taken seriously as a calculus of worldhood, it demands a corresponding execution environment. Traditional operating systems are designed around mutable state, reversible computation, and resource scheduling abstractions that presuppose rollback, overwrite, and idempotent retry. Spherepop OS inverts these assumptions. It is not an operating system for states, but for histories.

\subsection{Histories as the Primary Substrate}

In Spherepop OS, the fundamental unit of execution is not a process state, but a committed history. Every operation executed by the system corresponds to an irreversible event in the Spherepop calculus. Rather than maintaining a global mutable store, the system maintains a monotone ledger of events whose composition defines the current world.

This ledger is authoritative. There is no concept of undo, rollback, or speculative execution at the kernel level. Recovery from error is achieved only through explicit collapse, which quotients historical detail while preserving semantic residue. As a result, the operating system itself inhabits the same ethical and mechanical constraints as the agents it supports.

\subsection{Processes as Commitment Streams}

A process in Spherepop OS is defined as a stream of events emitted over time. Process identity is therefore historical rather than nominal: two processes are equivalent if and only if their event histories are equivalent up to collapse. Forking a process corresponds not to duplicating state, but to branching an option-space prior to commitment.

Inter-process communication is likewise reinterpreted. Messages are not transient signals, but binding events that introduce shared constraints between histories. Once a bind is established, neither process may violate it without inducing inconsistency, which must then be resolved via collapse.

\subsection{Files, Memory, and Persistence}

Files in Spherepop OS are not mutable blobs but stabilized projections of histories. Writing to a file is an act of collapse: a selective forgetting of the generative history in favor of a compact quotient suitable for reuse. Reading a file reintroduces this quotient as a binding constraint on future behavior.

Memory management follows the same principle. Instead of garbage collection, Spherepop OS performs historical compaction. Old commitments may be collapsed once their fine-grained structure no longer contributes to downstream decisions, reducing storage without erasing meaning. This approach aligns persistence, compression, and abstraction within a single formal mechanism.

\subsection{Scheduling and Time}

Scheduling in Spherepop OS is not primarily about fairness or throughput, but about ordering commitments. The scheduler enforces temporal bindings, ensuring that events occur in an order consistent with declared dependencies. Deadlock is interpreted not as a technical failure, but as a contradiction in commitments that requires explicit collapse or refusal to resolve.

Time itself is treated as emergent from event order rather than as an external clock. The system’s notion of “now” is simply the frontier of the history ledger.

\subsection{Security and Authority}

Security policies in Spherepop OS are expressed as refusals and binds rather than permissions. To deny an action is to permanently exclude a class of futures. Authority is therefore exercised through irreversible constraint rather than revocable access control.

This model yields a system that is inherently auditable. Every restriction, privilege, and exception is present as an explicit historical act, rather than as a latent configuration subject to silent change.

\section{Memory--First Cognitive Systems}
\label{sec:memory-first}

Spherepop also provides the foundation for a class of cognitive systems whose primary substrate is neither symbolic manipulation nor probabilistic sampling, but structured memory evolution. In such systems, reasoning is not the generation of sequences, but the transformation of latent memory states under irreversible constraint.

\subsection{Reasoning as History Construction}

In a Spherepop-based cognitive system, internal cognition proceeds as a sequence of events applied to an internal option-space. Each event commits the system to certain interpretations while excluding others. Reasoning is therefore inherently historical: conclusions are not merely outputs, but commitments that shape future inference.

This architecture aligns with a memory-centric view of cognition, in which latent memory trajectories, rather than surface narratives, carry causal weight. Explanations, when produced, are projections of these trajectories rather than drivers of them 1.

\subsection{Latent Memory States and Event Dynamics}

Let the internal state of the system at step $t$ be represented by a memory structure $M_t$. Each cognitive operation corresponds to an event $e_t$ that maps
\[
M_{t+1} = e_t(M_t),
\]
where $e_t$ may prune hypotheses, bind concepts, or collapse distinctions.

Crucially, these transformations are irreversible at the semantic level. Once a hypothesis is ruled out, it remains ruled out unless explicitly reintroduced through collapse. This ensures causal faithfulness: downstream reasoning depends on upstream commitments in a way that cannot be bypassed.

\subsection{Refusal as Cognitive Stabilization}

Refusal plays a central role in stabilizing cognition. By refusing classes of interpretations or actions, the system creates protected regions of its reasoning space. These refusals are not merely preferences; they are structural exclusions that prevent drift, oscillation, or opportunistic revision.

Because refusals increase commitment without forcing a positive choice, they allow the system to remain coherent without prematurely converging. This mechanism is especially important in long-horizon reasoning, where premature optimization can be as damaging as indecision.

\subsection{Collapse and Abstraction}

As reasoning proceeds, memory structures inevitably accumulate detail. Collapse provides the mechanism by which this detail is abstracted. By identifying memory states that are equivalent with respect to future action, the system reduces internal complexity while preserving causal relevance.

This operation underwrites generalization. Concepts are not learned by averaging over examples, but by collapsing distinctions that no longer matter. Abstraction is therefore not statistical, but historical.

\subsection{Action and Worldhood}

When such a system interacts with an external environment, its internal events must be coupled to irreversible external effects. Actions are therefore modeled as joint pops on internal and external option-spaces. A system that cannot bind its internal commitments to external consequences remains a simulator rather than an agent.

Spherepop thus provides a criterion for genuine agency: an agent is a system whose internal history constrains the external world, and whose external actions feed back into internal commitment.

\subsection{Interpretability and Oversight}

Because reasoning is encoded in latent memory trajectories, oversight is achieved by inspecting how events transformed these trajectories. Unlike narrative explanations, which may be post hoc or decorative, historical traces are causally upstream of behavior and therefore faithful by construction 2.

Interpretability, in this framework, is not a demand for verbalization, but a demand for access to the event history.

\section{Stratified Semantics: Calculus, Kernel, and View}
\label{sec:stratified-semantics}

One source of confusion in discussions of event--historical systems is the failure to distinguish between levels of semantic authority. Spherepop explicitly rejects monolithic semantics. Instead, it enforces a stratified architecture in which different layers serve different roles, with carefully constrained interactions between them. This stratification is not an implementation convenience; it is a semantic necessity.

\subsection{Three Semantic Layers}

Spherepop operates across three distinct semantic strata.

At the \emph{calculus layer}, Spherepop is an abstract event--history calculus. Its primitives are irreversible operators---pop, bind, refuse, and collapse---acting on option--spaces. At this level, the theory speaks in categorical terms: objects are spaces of admissible futures, morphisms are monotone constraints, and histories are composed arrows. The calculus provides meaning, ethics, and interpretive structure, but it does not execute.

At the \emph{kernel layer}, Spherepop is realized as a deterministic interpreter over an append-only event log. Here, events are concrete, canonical, and totally ordered. The kernel admits no speculation, rollback, or ambiguity. Given a prefix of the log, the resulting semantic state is uniquely determined. This layer is authoritative: it is the sole source of truth about what has occurred.

At the \emph{view layer}, Spherepop admits arbitrary observational projections. Snapshots, diffs, visual layouts, summaries, explanations, and speculative branches all live here. Views may collapse, annotate, reorder, or visualize kernel state, but they are explicitly forbidden from influencing it. This layer is non-authoritative by design.

These three layers correspond respectively to meaning, causation, and observation.

\subsection{Functorial Relationships}

The relationship between these layers is formal rather than metaphorical. The kernel layer may be understood as a concrete realization of the calculus layer, while the view layer arises as a family of functorial projections from kernel state.

Let $\mathsf{Pref}(\ell)$ denote the category of prefixes of an event log $\ell$, ordered by extension. Kernel execution defines a functor
\[
\mathsf{S}_\ell : \mathsf{Pref}(\ell) \to \mathsf{State},
\]
mapping each prefix to a uniquely determined kernel state. Any admissible view is then a functor
\[
\mathsf{V} : \mathsf{State} \to \mathsf{View},
\]
whose composition with $\mathsf{S}_\ell$ yields a view semantics that respects causal order.

The calculus layer sits above this structure. Its interpretation function factors through kernel execution: abstract events are mapped to concrete kernel events, and abstract histories correspond to log prefixes. Importantly, no functor exists from views back to the kernel. This asymmetry enforces causal discipline.

\subsection{Authority and Non-Interference}

A central invariant of Spherepop is the separation of cause from observation. Only kernel events may alter authoritative state. Views are free to discard information, speculate, or embellish, but such operations are semantically inert.

This discipline resolves a common tension in systems that attempt to combine introspection with determinism. By enforcing non-interference, Spherepop allows rich visualization, explanation, and exploration without risking semantic corruption. The calculus may reason about collapse, abstraction, and forgetting, but only the kernel may enact them authoritatively.

\subsection{Why Stratification Matters}

Without stratification, ethical interpretation threatens to pollute mechanics, and observational convenience threatens to undermine causality. Spherepop avoids both failures by making its layers explicit.

The calculus provides the language of worldhood. The kernel provides the machinery that enforces it. The view layer provides access without power. Together, these strata ensure that meaning accumulates irreversibly, while understanding remains flexible.

This stratified design is not optional. It is the structural condition that makes Spherepop coherent as both a philosophy of commitment and a working system.

\section{Collapse, Merge, and Equivalence as Quotienting}
\label{sec:collapse-quotient}

The concept of collapse occupies a privileged position in the Spherepop calculus. It is the only operator that reduces historical complexity rather than increasing it, and the only operation that explicitly trades detail for coherence. In the abstract calculus, collapse is introduced as a quotient on option--spaces. In Spherepop OS, this operation is realized concretely through equivalence induction and representative normalization. This section makes that correspondence explicit.

\subsection{Collapse in the Calculus}

At the level of the calculus, collapse is defined as an operation
\[
\operatorname{coll}_q : X \to X/{\sim_q},
\]
where $\sim_q$ is an equivalence relation determined by a collapse policy $q$. The policy specifies which distinctions are to be preserved for future reasoning and which are to be forgotten. The defining property of collapse is universal: any morphism out of $X$ that is insensitive to distinctions beyond $q$ factors uniquely through the quotient.

This formulation captures abstraction in its strongest sense. Collapse does not merely remove options; it removes \emph{differences}. Two futures that were previously distinct become identified, not because they are impossible, but because they are no longer meaningfully distinguishable with respect to future commitments.

\subsection{Merge and Collapse in the Kernel}

Spherepop OS realizes this abstract notion through a small set of kernel events, most notably \texttt{MERGE} and \texttt{COLLAPSE}. A \texttt{MERGE} event induces an equivalence between two semantic objects, while a \texttt{COLLAPSE} event induces equivalence over an entire region. In both cases, the kernel maintains equivalence classes via a union--find structure, selecting canonical representatives and normalizing all relations accordingly.

From a semantic perspective, these operations are not additive updates but quotient constructions. Once objects are merged, all references to them are rewritten to their representative, and the distinction between the merged objects is irretrievably lost at the authoritative level. This is precisely the behavior required of collapse in the calculus.

Importantly, objects that are no longer representatives are not deleted. They persist as historical artifacts, but they no longer participate independently in future structure. This mirrors the calculus-level claim that collapse forgets scaffolding while preserving outcome.

\subsection{Action Reduction and Renormalization}

In the Lagrangian formulation of Spherepop, collapse is the only operator that reduces the accumulated action:
\[
\mathcal{S} \longmapsto \mathcal{S} - \log|\ker q|.
\]
This reduction reflects the elimination of degrees of freedom that no longer contribute to future choice.

The kernel-level realization of this principle is compaction through equivalence. When multiple historical branches are identified, the effective dimensionality of the semantic state is reduced. Replay remains deterministic, but the space of distinctions that must be tracked shrinks. In this sense, collapse functions as a renormalization step on history: fine-grained distinctions are integrated out, yielding a coarser but more stable description.

This perspective clarifies why collapse is neither error correction nor optimization. It is a deliberate act of abstraction, undertaken when the cost of retaining detail exceeds its future value.

\subsection{Confluence and Order Independence}

A critical property of quotienting operations is confluence. In Spherepop OS, the final equivalence relation induced by a set of merge or collapse events is independent of their order. This is guaranteed by the algebraic properties of union--find and is formalized by merge confluence theorems in the kernel specification.

At the level of the calculus, this corresponds to the commutativity of collapse operations that act on independent regions. The order in which irrelevant distinctions are forgotten does not affect the final world, provided that the same identifications are eventually made. This reinforces the interpretation of collapse as abstraction rather than destruction.

\subsection{Collapse versus Erasure}

It is essential to distinguish collapse from erasure. Erasure deletes information outright, rendering the past inaccessible. Collapse, by contrast, preserves history while changing how it is interpreted. The authoritative log remains intact; only the semantic equivalence structure changes.

This distinction underwrites several of Spherepop OS’s most important properties, including deterministic replay, historical inspection, and late-joining observers. Collapse simplifies the present without falsifying the past. In philosophical terms, it enables growth without amnesia.

\subsection{Summary}

Collapse in Spherepop is not a metaphor, nor an implementation trick. It is a mathematically precise quotient operation that appears consistently across the calculus, the kernel, and the mechanics of commitment. By identifying collapse with equivalence induction, Spherepop unifies abstraction, compression, and forgetting within a single formal act. This unity is what allows histories to deepen without becoming unmanageable, and worlds to remain coherent without remaining brittle.

\section{Refusal, Authority, and Non--Interference}
\label{sec:refusal-authority}

The introduction of the refusal operator raises a legitimate concern: whether normative distinctions risk contaminating the mechanical semantics of the system. Spherepop addresses this concern not by weakening refusal, but by precisely locating where refusal is permitted to act. This section clarifies the role of refusal by situating it within the architecture of authority and non--interference.

\subsection{Geometric Equivalence and Normative Distinction}

At the level of the event--history calculus, refusal is geometrically indistinguishable from pop. Both operators induce the same restriction on the option--space:
\[
\llbracket \operatorname{refuse}_t \rrbracket
=
\llbracket \operatorname{pop}_t \rrbracket
\quad \text{in } \mathcal{O}.
\]
This equivalence is deliberate. The mechanics of constraint are indifferent to motivation. Whether an option is excluded instrumentally or deontically, the resulting future space is the same.

The distinction between pop and refuse therefore does not belong to the geometry of option--spaces. It belongs to interpretation. Spherepop encodes this distinction not by altering the kernel transition rules, but by recording it in auxiliary structures that are explicitly non-authoritative.

\subsection{Refusal and the Accounting Layer}

Normative information in Spherepop is carried by metadata and accounting, not by kernel state. This separation is enforced architecturally. Kernel events alter only the authoritative semantic substrate: objects, relations, and equivalence classes. Metadata may annotate these events, but it is not consulted by the kernel when determining state transitions.

Formally, refusal differs from pop only under an accounting functor
\[
\mathcal{A} : \mathcal{O} \to \mathcal{C},
\]
which assigns cost, debt, or ethical weight to events. This functor is additive over history, aligning directly with the Lagrangian formulation of commitment. Because $\mathcal{A}$ is not invertible back into kernel semantics, ethical distinctions cannot interfere with causality.

\subsection{Authority and Proposal Rejection}

In Spherepop OS, the only locus at which refusal may have causal force is prior to event commitment. Clients propose events; the arbiter either accepts or rejects them. A rejected proposal is never sequenced into the authoritative log and therefore never becomes part of history.

This form of refusal is external to the kernel. It does not modify state, induce equivalence, or alter relations. It is an exercise of authority over what may become real, not an operation within reality itself. Once an event is committed, its semantics are purely mechanical.

This design mirrors the calculus-level distinction between refusal as a principle and pop as an effect. Refusal governs admission to history; pop governs the shape of history once admitted.

\subsection{Non--Interference as a Semantic Invariant}

A central invariant of Spherepop is non--interference: no structure that is not itself an event may influence authoritative state. Views, metadata, explanations, layouts, speculative branches, and ethical annotations are all barred from causal feedback.

Refusal respects this invariant. Its normative content is visible to observers and agents, but opaque to the kernel. As a result, the system enjoys a strong form of semantic hygiene: ethics may interpret mechanics, but mechanics never interpret ethics.

This asymmetry is essential. If refusal were permitted to alter kernel transitions, the system would lose determinism and replayability. By contrast, by confining refusal to accounting and authority boundaries, Spherepop gains ethical expressiveness without sacrificing formal rigor.

\subsection{Commitment Without Coercion}

One consequence of this design is that refusal increases commitment without coercion. A refusal may permanently exclude futures while leaving the remaining space unconstrained. In the Lagrangian formulation, refusal increases commitment momentum $\pi$ without forcing a positive choice. This allows agents to stabilize their world without prematurely collapsing it.

This property distinguishes refusal from optimization. Optimization selects; refusal excludes. Spherepop treats the latter as foundational.

\subsection{Summary}

Refusal does not pollute Spherepop’s mechanics because it is never allowed to enter them. It operates at the boundaries of authority and interpretation, not within the kernel. By enforcing strict non--interference, Spherepop preserves deterministic causality while enabling a rich ethics of commitment. Refusal is thus not a loophole in the calculus, but one of its defining strengths.

\section{Spherepop and Psychoanalytic Time: A Comparison with Psychocinema}
\label{sec:psychocinema}

Recent work in film theory and psychoanalytic criticism has emphasized the centrality of temporal structure, absence, and retroactive meaning-making in human cognition. Helen Rollins’ \emph{Psychocinema} provides a particularly clear articulation of this tradition, drawing on Lacanian psychoanalysis to argue that cinematic meaning arises not from linear causality, but from the recursive reorganization of desire, memory, and identification across time. While Spherepop shares certain surface resonances with this account, the two frameworks diverge fundamentally in both their ontological commitments and their treatment of irreversibility.

\subsection{Lacanian Temporality and Retroactive Meaning}

In Lacanian psychoanalysis, meaning is not fixed at the moment of inscription. Instead, it emerges retroactively through what Lacan termed \emph{après-coup} (deferred action). An early event acquires significance only in light of later symbolic articulation. In cinematic terms, a scene may be reinterpreted by a later revelation, restructuring the viewer’s understanding of what came before.

Rollins extends this logic by arguing that cinema functions as a machine for producing such retroactive reconfigurations. The cut, the flashback, and the re-framing shot operate not merely as narrative devices, but as mechanisms that reorganize psychic structure. Memory, on this view, is plastic and revisable; the past is continually rewritten by the symbolic order.

This account emphasizes discontinuity, fragmentation, and the instability of meaning. The subject is constituted not by accumulated commitment, but by lack—by what is missing, repressed, or disavowed.

\subsection{Superficial Convergences}

At a descriptive level, Spherepop appears to echo several Lacanian themes. Both frameworks reject naive linear temporality. Both emphasize the importance of exclusion, absence, and structural constraint. Both treat narrative as something constructed rather than discovered.

Most notably, Spherepop’s collapse operator may appear analogous to psychoanalytic reinterpretation: a later operation restructures the significance of earlier distinctions. Similarly, refusal may seem to parallel repression or foreclosure, in which certain possibilities are excluded from conscious articulation.

However, these similarities are superficial. They arise from shared concern with temporality, not from shared ontology.

\subsection{Irreversibility versus Retroactivity}

The decisive difference between Spherepop and Lacanian models lies in the treatment of irreversibility. In psychoanalytic theory, reinterpretation is fundamentally retroactive. Meaning flows backward in time. The past is not fixed; it is continuously re-signified by the present.

Spherepop explicitly forbids this. While collapse may reduce the relevance of past distinctions, it does not alter what occurred. The authoritative history remains intact. Collapse quotients the interpretation of the past, not the past itself. There is no mechanism by which a later event can causally rewrite an earlier one.

Formally, this difference is captured by the directionality of morphisms. In Spherepop, histories compose forward:
\[
H = e_n \circ \cdots \circ e_1,
\]
and there exists no inverse operation capable of modifying $e_1$ in light of $e_n$. Any reinterpretation occurs strictly at the level of views, never at the level of kernel semantics.

Thus, where psychoanalysis treats meaning as inherently unstable and revisable, Spherepop treats meaning as cumulative and constrained.

\subsection{Absence as Lack versus Absence as Exclusion}

Lacanian theory centers lack as a constitutive absence: the subject is structured around what cannot be symbolized. Desire arises from this structural gap. In \emph{Psychocinema}, this manifests as an emphasis on what is withheld from the image, what is left unseen, and what can only be inferred.

Spherepop, by contrast, treats absence as an explicit act. An option is absent because it was excluded. A future does not exist because it was popped or refused. Absence is therefore not primordial; it is historical.

This distinction has ethical consequences. In a Lacanian framework, the subject is fundamentally alienated from itself by structural lack. In Spherepop, the agent is constituted by its commitments. What is missing is missing because it was given up, not because it was never available.

\subsection{Cinema, Spectatorship, and Worldhood}

Rollins emphasizes the role of the spectator in completing cinematic meaning. The viewer’s psychic investment retroactively stabilizes the narrative. Meaning is thus distributed across film and spectator, and remains perpetually open to reinterpretation.

Spherepop instead aligns meaning with worldhood. A world is not something that can be endlessly reinterpreted without consequence. To inhabit a world is to be bound by its history. Spectatorship, in this sense, corresponds to the view layer of Spherepop: a space of interpretation, projection, and re-framing that does not alter the underlying causal structure.

This distinction clarifies why Spherepop resists purely interpretive accounts of cognition. While interpretation is indispensable, it must be grounded in an authoritative history if it is to constrain action.

\subsection{From Psychoanalytic Narrative to Formal Commitment}

The contrast between Spherepop and Lacanian psychocinema can therefore be summarized succinctly. Psychoanalytic models privilege narrative reinterpretation over causal commitment. Spherepop inverts this priority. Narrative is a derived view; commitment is primitive.

Where psychoanalysis asks how meaning emerges from lack, Spherepop asks how meaning emerges from exclusion. Where psychoanalysis treats the past as plastic, Spherepop treats it as fixed but compressible. Where cinema invites endless reinterpretation, Spherepop demands that choices settle the world.

\subsection{Summary}

Spherepop does not deny the insights of psychoanalytic theory, nor the descriptive power of cinematic models of meaning. Instead, it formalizes what those models leave implicit: that without irreversible commitment, interpretation floats free of consequence. By replacing retroactive re-signification with forward-accumulating constraint, Spherepop offers a framework in which meaning is not endlessly deferred, but responsibly constructed.

\section{From Calculus to Visualization: Nested Scopes as Spatial Form}
\label{sec:visual-language}

One advantage of Spherepop over interpretive or narrative frameworks is that its primitives admit a direct spatial realization. Because the calculus is formulated in terms of nested option--spaces, monotone restriction, and quotienting, it can be rendered visually without loss of semantic fidelity. This section explains why Spherepop lends itself naturally to two-- and three--dimensional visual languages based on nested bubbles or spheres, and why such representations are more than pedagogical conveniences.

\subsection{Option--Spaces as Regions}

At the core of Spherepop lies the option--space $X$: the set of admissible futures consistent with the current history. Any restriction
\[
X' \subseteq X
\]
induced by a pop, refuse, or bind operation corresponds geometrically to the selection of a subregion. This observation immediately suggests a spatial metaphor in which option--spaces are rendered as regions, and events are rendered as operations that carve, constrain, or relate those regions.

In two dimensions, such regions may be depicted as nested areas or bubbles. In three dimensions, they become volumes or spheres. Crucially, the calculus does not require that these regions be metric or Euclidean; only inclusion and overlap matter. Visualization therefore respects topology rather than geometry.

\subsection{Nested Scopes and Worldhood}

Spherepop histories generate a hierarchy of scopes. Each irreversible event narrows, reshapes, or identifies parts of the option--space, producing a nested structure:
\[
X_0 \supseteq X_1 \supseteq \cdots \supseteq X_n.
\]
This nesting corresponds directly to the intuitive notion of “being inside” a world. Earlier, less constrained futures surround later, more committed ones. In a spatial rendering, deeper commitment appears as deeper nesting.

This stands in contrast to state-based diagrams, which typically depict transitions between disjoint nodes. Spherepop’s visual grammar emphasizes containment rather than succession, reinforcing the idea that later worlds are not alternatives to earlier ones, but refinements of them.

\subsection{Binding as Relational Geometry}

The bind operator introduces ordering constraints without eliminating futures. Visually, this may be represented not by shrinking a region, but by introducing structured relations within it. In two dimensions, such relations may appear as directional constraints or layered zones. In three dimensions, they may be rendered as oriented subvolumes or channels that impose precedence.

Importantly, binding does not collapse space; it shapes it. A visual language based on nested spheres allows bindings to coexist with optionality, making visible the distinction between restriction and elimination.

\subsection{Collapse as Boundary Identification}

Collapse has a particularly natural spatial interpretation. When two regions are identified under a collapse policy, the boundary between them disappears. Distinct bubbles merge into a single region, not by deletion, but by boundary identification.

This corresponds exactly to the quotient construction in the calculus. In a visual language, collapse is experienced as simplification: a reduction in the number of distinguishable compartments. The viewer perceives abstraction directly, as complexity is integrated out.

Because collapse preserves the underlying history while altering its representation, visual collapse aligns with the distinction between kernel semantics and views. The visualization changes; the authoritative log does not.

\subsection{Refusal and Protected Regions}

Refusal is especially well suited to visual representation. Because it excludes options without forcing positive choice, it may be depicted as the sealing off of regions rather than their deletion. A refused region remains visible as excluded, preserving interpretability without reopening possibility.

Such protected regions clarify the ethical dimension of refusal. They show not merely what is unavailable, but what has been deliberately set aside. In contrast to repression metaphors, these regions are not hidden; they are marked.

\subsection{Advantages over Narrative Visualization}

Traditional narrative visualization proceeds by sequencing frames or scenes. This encourages a retrospective, interpretive mode of engagement, in which meaning is reconstructed after the fact. Spherepop’s spatial visualization instead foregrounds structure over sequence. The viewer sees at once what is possible, what is excluded, and how constraints nest.

This difference mirrors the earlier contrast with psychoanalytic cinema. Where cinematic meaning unfolds through temporal reinterpretation, Spherepop meaning is apprehended through spatial containment. Time is present, but as depth rather than flow.

\subsection{Toward an Executable Visual Language}

Because the visual elements correspond directly to formal constructs, a Spherepop visualization need not be a passive diagram. Regions may be interactive, events may be enacted by direct manipulation, and collapse may be performed by explicit merging. In this way, the visual language becomes an interface to the calculus rather than a commentary on it.

Such a language would allow users to reason about commitment, scope, and abstraction spatially, without abandoning formal rigor. Nested bubbles and spheres are not metaphors imposed from outside, but faithful renderings of the calculus itself.

\subsection{Summary}

Spherepop’s suitability for two-- and three--dimensional visualization follows directly from its formal structure. Because meaning is encoded in nested scopes, monotone restriction, and quotienting, spatial representation preserves semantics rather than distorting it. Visualization thus becomes a natural extension of the calculus, providing an intuitive yet precise medium for inhabiting event--historical worlds.

\section{Cognitive Agents as Log--Bound Processes}
\label{sec:log-bound-agents}

With the semantic, mechanical, and visual foundations in place, we can now state a positive account of agency within Spherepop. An agent, in this framework, is not a function from observations to actions, nor a policy over states, but a process whose identity is constituted by an accumulating, authoritative history. Cognition is the internal construction of that history; action is its external projection.

\subsection{Agents as Historical Processes}

Formally, a Spherepop agent is a process that emits a sequence of events
\[
\gamma = (e_1, e_2, \dots, e_n)
\]
into an authoritative log. The agent’s identity is not given by a parameter vector or internal configuration at time $t$, but by the prefix of the log to which it is bound. Two agents that share the same history are indistinguishable in the strong semantic sense, regardless of their internal representations.

This definition enforces a crucial asymmetry: the agent may reason about multiple futures, but it may inhabit only one past. Speculation is permitted; commitment is singular.

\subsection{Internal Cognition as Event Simulation}

Internal cognition within a Spherepop agent consists of proposing candidate events against an internal option--space. These proposals may be explored, evaluated, or rejected, but they do not alter the authoritative history until committed. Cognition is therefore structurally separated from action.

Each accepted internal event transforms the agent’s internal memory state:
\[
M_{t+1} = e_t(M_t),
\]
where $M_t$ is itself an option--space enriched with bindings and exclusions. Because these transformations are irreversible at the semantic level, internal cognition accumulates commitment even before external action is taken.

This model yields a form of thinking that is neither purely symbolic nor purely statistical. Reasoning proceeds by narrowing, binding, and collapsing internal possibilities, rather than by generating sequences of representations.

\subsection{Memory as Causal Substrate}

In Spherepop, memory is not a store of representations but a record of constraints. What an agent remembers is what it has ruled out, what it has bound together, and what distinctions it has collapsed. This memory is causally upstream of all future reasoning.

As a consequence, explanation is not a post hoc narrative imposed on behavior, but a projection of the causal history that produced it. To explain an action is to point to the sequence of events that made alternative actions unavailable. Interpretability thus arises from access to history, not from transparency of internal representations.

\subsection{Action as Joint Commitment}

When an agent acts in the world, it performs a joint commitment: an internal event and an external event are bound together. Internally, the agent prunes and binds its own option--space; externally, the world’s option--space is likewise constrained. Action is therefore not output, but synchronization between histories.

An agent that cannot bind its internal commitments to external consequences remains a simulator. Conversely, an agent whose external actions do not feed back into internal history lacks learning. Genuine agency requires both directions of constraint.

\subsection{Refusal and Stability of Agency}

Refusal plays a central role in maintaining agent coherence over long horizons. By refusing classes of actions or interpretations, an agent creates protected regions within its option--space. These refusals stabilize identity by preventing oscillation between incompatible commitments.

Because refusal increases commitment without forcing positive selection, it allows agents to remain open where openness is valuable, while remaining closed where closure is necessary. This mechanism replaces brittle optimization with resilient worldhood.

\subsection{Collapse and Concept Formation}

As an agent accumulates history, internal memory inevitably grows complex. Collapse provides the mechanism by which this complexity is managed. By identifying internal distinctions that no longer affect future action, the agent forms concepts: equivalence classes over past experience.

Concept formation, in this view, is not inductive generalization over data points, but historical compression. Concepts are the residue of many commitments collapsed into a single handle. This explains both their stability and their context-sensitivity.

\subsection{Visualizing Agent Cognition}

The nested-scope visualization introduced earlier admits a direct cognitive interpretation. An agent’s internal world may be rendered as a hierarchy of nested regions, with deeper regions corresponding to more committed beliefs or plans. Refused regions appear sealed; bound regions appear structured; collapsed regions appear merged.

Such a visualization is not merely illustrative. It provides an externalization of the agent’s internal causal structure, allowing both the agent and observers to reason spatially about commitment, freedom, and abstraction. In this sense, visualization becomes a cognitive tool rather than a reporting mechanism.

\subsection{Worldhood and Responsibility}

Because Spherepop agents are log-bound, they are accountable to their own histories. Actions have consequences not only in the world, but in the agent’s future self. This temporal coupling grounds responsibility. To act is to bind oneself.

Worldhood, in this framework, is not the possession of a model of the world, but the condition of being constrained by one’s past. An agent without irreversible history may appear intelligent, but it remains uninhabited.

\subsection{Summary}

Spherepop redefines agency as the capacity to construct and inhabit a history. By treating cognition as event simulation, memory as accumulated constraint, and action as joint commitment, it offers a unified account of intelligence grounded in irreversibility. Agents are not optimizers over states, but authors of worlds—one exclusion at a time.

\section{The Agent as Endofunctor and Parser}
\label{sec:endofunctor-agent}

The preceding sections have described Spherepop agents operationally and phenomenologically. We now offer a more abstract characterization that unifies these views: a Spherepop agent may be understood as an endofunctor on a category of option--spaces, functioning simultaneously as a parser of events and as a transformer of worlds.

\subsection{Endofunctorial Structure}

Let $\mathcal{O}$ denote the category of option--spaces with monotone morphisms. A Spherepop agent induces an endofunctor
\[
\mathsf{A} : \mathcal{O} \to \mathcal{O},
\]
which maps a current world $X$ to a more committed world $\mathsf{A}(X)$ by applying a sequence of irreversible events. The functoriality condition expresses coherence: identity histories act trivially, and concatenation of histories corresponds to composition.

Crucially, $\mathsf{A}$ is not invertible. It preserves inclusion but not equivalence. In categorical terms, it is monotone but not an equivalence. This captures the agent’s essential asymmetry with respect to time: the agent moves forward by losing possibilities.

\subsection{Parsing as Commitment}

Viewed internally, the agent does not merely transform worlds; it \emph{parses} incoming structure. Each perceptual or informational input is treated not as a datum to be stored, but as a proposal for constraint. To parse is to decide which distinctions to preserve, which to bind, and which to discard.

In this sense, a Spherepop agent functions as a parser whose output is not a syntax tree, but a transformed option--space. Parsing is successful when it yields a committed world in which future action is better constrained. Ambiguity corresponds to retained optionality; understanding corresponds to exclusion.

This reframes cognition. Comprehension is not the construction of an internal representation isomorphic to the input, but the construction of a world in which the input no longer needs to be reconsidered.

\subsection{Functorial Identity and Agency}

Because the agent is identified with an endofunctor rather than a state, its identity is extensional over histories. Two agents are the same agent if they induce the same transformation on worlds. Internal representations, data structures, or implementations are irrelevant at this level of description.

This perspective clarifies why Spherepop agents resist cloning or resetting. Copying an agent without copying its history produces a different functor. Identity follows commitment, not configuration.

\subsection{Summary}

To describe a Spherepop agent as an endofunctor is to emphasize that agency is world-transforming rather than state-transitioning. The agent parses the world by committing to it, and in doing so, becomes inseparable from the history it constructs.

\section{Sheaf-Theoretic Interpretations of Worldhood}
\label{sec:sheaf-worldhood}

The stratified architecture of Spherepop admits a natural interpretation in sheaf-theoretic terms. This perspective provides a unifying language for understanding locality, global consistency, abstraction, and collapse, while remaining faithful to the system’s commitment to irreversibility.

\subsection{Local Worlds and Covers}

Let a world $X$ be viewed not as a single monolithic structure, but as a space admitting local perspectives. Each local perspective corresponds to a restriction of the option--space to a subregion or scope. The collection of such scopes forms a cover of $X$.

Assignments of structure to these scopes---partial histories, bindings, or visualizations---may be regarded as sections over the cover. These sections need not agree globally. Indeed, disagreement corresponds precisely to unresolved optionality.

\subsection{Gluing as Commitment}

The sheaf condition asserts that compatible local sections may be uniquely glued into a global section. In Spherepop terms, gluing is not automatic. It is an act of commitment. To glue is to pop distinctions that would otherwise keep local views separate.

Binding operations impose compatibility conditions between sections, while collapse enforces gluing by identifying differences that no longer matter. A world becomes globally coherent only insofar as the agent has committed to resolving local ambiguities.

Thus, global consistency is not a given; it is achieved through irreversible action.

\subsection{Collapse as Sheafification}

Collapse admits a direct interpretation as sheafification. Given a presheaf of local distinctions that may fail to glue, collapse quotients the structure until the sheaf condition is satisfied with respect to a chosen cover. Distinctions that obstruct gluing are identified.

This interpretation aligns precisely with the kernel semantics of equivalence induction and representative normalization. Collapse enforces consistency by reducing resolution, not by falsifying local data.

\subsection{Views as Sections}

Visualizations, explanations, and summaries correspond to sections of the sheaf over particular covers. Different views may emphasize different localities, resolutions, or abstractions. Because views are non-authoritative, they need not glue globally. Their purpose is navigation, not truth.

The authoritative history corresponds to the minimal global section obtained after all committed collapses. Everything else is a projection.

\subsection{Worldhood Revisited}

From the sheaf-theoretic perspective, to inhabit a world is to live within a structure that has been sufficiently sheafified. Ambiguity remains only where commitment has been withheld. The boundary between inside and outside a world corresponds to the boundary between glued and unglued structure.

This framing further clarifies the ethical dimension of Spherepop. To refuse to glue is to preserve plurality; to collapse is to accept unity. Both are legitimate acts, but both have consequences.

\subsection{Summary}

Sheaf theory provides a language in which Spherepop’s core intuitions become precise. Locality corresponds to optionality, gluing to commitment, and collapse to abstraction. Worldhood emerges not from total information, but from the disciplined resolution of incompatibility.

\section{Conclusion: Joy in a Finite World}
\label{sec:conclusion}

This essay has argued for a reorientation of how we think about computation, cognition, and agency. Rather than treating the world as a state to be updated or a distribution to be sampled, Spherepop treats the world as a history to be constructed. Meaning, on this view, is not found in representations or predictions, but in the irreversible commitments that shape what remains possible.

We began by introducing Spherepop as an event--history calculus whose primitive operations---pop, bind, refuse, and collapse---act directly on spaces of admissible futures. By elevating irreversibility from an implementation concern to a semantic axiom, the calculus reframes intelligence as the capacity to permanently exclude, order, and abstract possibilities. The Lagrangian formulation made this concrete: action accumulates through loss of optionality, commitment emerges as conjugate momentum, and freedom is something that is spent rather than preserved.

We then showed how this abstract calculus is realized operationally through a stratified architecture. The kernel enforces determinism and authority through an append-only log; the view layer provides interpretation without interference; and the calculus supplies meaning without execution. This separation allows ethics, visualization, and cognition to flourish without compromising causal integrity. Collapse, understood as quotienting, unifies abstraction, compression, and forgetting across all layers.

By contrasting Spherepop with psychoanalytic and cinematic accounts of temporality, we clarified what is distinctive about its treatment of meaning. Where interpretive frameworks emphasize retroactivity and reinterpretation, Spherepop insists on forward accumulation and responsibility. The past is not rewritten; it is compressed. Absence is not primordial lack, but the residue of deliberate exclusion. This difference grounds an ethics of agency in which commitments matter because they bind the future.

The discussion of visualization revealed that Spherepop is not merely compatible with spatial representation, but naturally expressed through it. Nested bubbles and spheres render option--spaces, scopes, bindings, and collapses directly visible. Such visual languages are not metaphors layered atop the calculus, but faithful projections of its structure, enabling agents to reason spatially about commitment and freedom.

We then articulated a positive account of agency. A Spherepop agent is a log--bound process whose identity is constituted by its history. Cognition is internal event simulation; memory is accumulated constraint; action is joint commitment between internal and external worlds. Refusal stabilizes identity without coercion, while collapse enables abstraction without erasure. In this framework, to be an agent is to be accountable to one’s past.

Finally, by recasting the agent as an endofunctor and interpreting worldhood through sheaf theory, we showed how Spherepop unifies parsing, cognition, and abstraction within a single mathematical language. Local ambiguity, global coherence, and ethical choice appear as different aspects of the same underlying structure: the disciplined resolution of incompatibility over time.

The joy of Spherepop lies precisely here. In a world of infinite possibility, nothing matters. It is only by giving things up---by popping futures, refusing paths, binding promises, and collapsing detail---that a world becomes inhabitable. Joy is not found in maximal freedom, but in the relief of commitment: the moment when possibility condenses into reality, and a finite path forward appears.

Spherepop offers neither a utopia nor a universal optimizer. It offers something more modest and more profound: a language for living in worlds that remember what they have ruled out.

\appendix
\section{Core Syntax}
\label{app:syntax}

This appendix presents a minimal surface syntax for Spherepop. The syntax is intentionally restrictive: it admits no general control flow, mutation, or rollback. Programs are histories.

\subsection{Grammar}

\begin{verbatim}
<program> ::= <statement>*

<statement> ::= <event> | <declaration>

<event> ::= pop <target>
          | refuse <target>
          | bind <target> "<" <target>
          | collapse <policy>

<declaration> ::= let <name> = <expression>

<target> ::= <name> | <path>

<policy> ::= <name>

<name> ::= identifier
\end{verbatim}

\subsection{Syntactic Irreversibility}

There is no syntax for undo, overwrite, or reassignment. Once an event is emitted, it cannot be syntactically referenced except as history. This enforces irreversibility at the language level.

\section{Abstract Syntax and Denotational Semantics}
\label{app:ast}

\subsection{Abstract Syntax Tree}

A Spherepop program is represented as a linear abstract syntax tree:
\[
\mathsf{AST} = [n_1, n_2, \dots, n_k],
\]
where each node $n_i$ is one of:
\[
\mathsf{Pop}(t),\;
\mathsf{Refuse}(t),\;
\mathsf{Bind}(a,b),\;
\mathsf{Collapse}(q).
\]

The ordering of nodes is semantically significant and may not be altered unless independence can be formally established.

\subsection{Denotational Semantics}

Let $\mathcal{O}$ be the category of option--spaces with monotone morphisms. The interpretation function is:
\[
\llbracket - \rrbracket : \mathsf{AST} \to (X_0 \to X_n),
\]
defined compositionally by:
\[
\llbracket [n_1,\dots,n_k] \rrbracket
=
\llbracket n_k \rrbracket \circ \cdots \circ \llbracket n_1 \rrbracket.
\]

Each node denotes a morphism in $\mathcal{O}$:
\[
\llbracket \mathsf{Pop}(t) \rrbracket = \llbracket \mathsf{Refuse}(t) \rrbracket : X \to X|_{\neg t},
\]
\[
\llbracket \mathsf{Bind}(a,b) \rrbracket : X \to X[a \prec b],
\]
\[
\llbracket \mathsf{Collapse}(q) \rrbracket : X \to X/{\sim_q}.
\]


\section{Lagrangian Formulation}
\label{app:lagrangian}

For reference, we summarize the discrete mechanics used throughout the paper.

\subsection{Configuration Variable}

Let $\Omega(X)$ denote the optionality of an option--space $X$.

\subsection{Local Lagrangian}

Each event $e_t$ induces a local cost:
\[
\mathcal{L}_t = \Delta\Omega_t + \lambda\,\Delta C_t,
\]
where $\Delta\Omega_t$ is geometric loss of optionality and $\Delta C_t$ is accounting cost.

\subsection{Action}

The action of a history $\gamma$ is:
\[
\mathcal{S}[\gamma] = \sum_t \mathcal{L}_t.
\]

\subsection{Commitment and Hamiltonian}

Commitment is defined as the conjugate momentum:
\[
\pi_t = -\frac{\partial \mathcal{L}_t}{\partial \Omega}.
\]

The Hamiltonian is:
\[
\mathcal{H}_t = \pi_t\,\Omega_t - \mathcal{L}_t,
\]
interpreted as remaining freedom.

\subsection{Collapse}

Collapse is the only operator that reduces action:
\[
\mathcal{S} \mapsto \mathcal{S} - \log|\ker q|.
\]

\section{Categorical Glossary}
\label{app:glossary}

\begin{description}
\item[Option--Space:]
An object representing admissible futures.

\item[History:]
A compositional sequence of monotone morphisms.

\item[Collapse:]
A quotient morphism identifying distinctions irrelevant to future action.

\item[Endofunctor:]
An agent viewed as a world-transforming functor $\mathcal{O} \to \mathcal{O}$.

\item[Sheaf:]
A structure assigning local sections to scopes, with gluing enforced by commitment.

\item[View:]
A non-authoritative projection of kernel state.
\end{description}

\begin{thebibliography}{99}

\bibitem{aristotle_metaphysics}
Aristotle.
\textit{Metaphysics}.
Translated by W. D. Ross.
Oxford University Press, 1924.

\bibitem{heidegger_being_time}
Martin Heidegger.
\textit{Being and Time}.
Translated by John Macquarrie and Edward Robinson.
Harper \& Row, 1962.

\bibitem{heidegger_technology}
Martin Heidegger.
The Question Concerning Technology.
In \textit{The Question Concerning Technology and Other Essays}.
Translated by William Lovitt.
Harper \& Row, 1977.

\bibitem{whitehead_process}
Alfred North Whitehead.
\textit{Process and Reality}.
Free Press, 1978.

\bibitem{lawvere_adjoints}
F. William Lawvere.
Adjointness in Foundations.
\textit{Dialectica}, 23(3–4):281–296, 1969.

\bibitem{maclane_categories}
Saunders Mac Lane.
\textit{Categories for the Working Mathematician}.
Springer, 2nd edition, 1998.

\bibitem{awodey_category}
Steve Awodey.
\textit{Category Theory}.
Oxford University Press, 2nd edition, 2010.

\bibitem{grothendieck_sga1}
Alexander Grothendieck.
\textit{Séminaire de Géométrie Algébrique du Bois Marie (SGA 1)}.
Springer, 1971.

\bibitem{maclane_moerdijk_sheaves}
Saunders Mac Lane and Ieke Moerdijk.
\textit{Sheaves in Geometry and Logic}.
Springer, 1992.

\bibitem{needham_visual_complex}
Tristan Needham.
\textit{Visual Complex Analysis}.
Oxford University Press, 1997.

\bibitem{needham_geometry_imaginary}
Tristan Needham.
\textit{Visual Differential Geometry and Forms}.
Princeton University Press, 2021.

\bibitem{meijer_category_effects}
Erik Meijer.
Category Theory for Programmers.
\textit{Communications of the ACM}, 61(12):58–64, 2018.

\bibitem{meijer_folds}
Erik Meijer, Maarten Fokkinga, and Ross Paterson.
Functional Programming with Bananas, Lenses, Envelopes and Barbed Wire.
In \textit{FPCA ’91 Proceedings}.
Springer, 1991.

\bibitem{jacobson_thermodynamics}
Ted Jacobson.
Thermodynamics of Spacetime: The Einstein Equation of State.
\textit{Physical Review Letters}, 75(7):1260–1263, 1995.

\bibitem{verlinde_gravity}
Erik Verlinde.
On the Origin of Gravity and the Laws of Newton.
\textit{Journal of High Energy Physics}, 2011(4):29, 2011.

\bibitem{friston_fep}
Karl Friston.
The Free-Energy Principle: A Unified Brain Theory?
\textit{Nature Reviews Neuroscience}, 11(2):127–138, 2010.

\bibitem{landauer_irreversibility}
Rolf Landauer.
Irreversibility and Heat Generation in the Computing Process.
\textit{IBM Journal of Research and Development}, 5(3):183–191, 1961.

\bibitem{bennett_logical_reversibility}
Charles H. Bennett.
Logical Reversibility of Computation.
\textit{IBM Journal of Research and Development}, 17(6):525–532, 1973.

\bibitem{ashby_variety}
W. Ross Ashby.
\textit{An Introduction to Cybernetics}.
Chapman \& Hall, 1956.

\bibitem{marr_vision}
David Marr.
\textit{Vision}.
W. H. Freeman, 1982.

\bibitem{rosen_life_itself}
Robert Rosen.
\textit{Life Itself}.
Columbia University Press, 1991.

\bibitem{lacan_four_fundamental}
Jacques Lacan.
\textit{The Four Fundamental Concepts of Psychoanalysis}.
Translated by Alan Sheridan.
W. W. Norton, 1978.

\bibitem{deleuze_cinema2}
Gilles Deleuze.
\textit{Cinema 2: The Time-Image}.
University of Minnesota Press, 1989.

\bibitem{rollins_psychocinema_2024}
Helen Rollins.
\textit{Psychocinema}.
Polity Press,
Cambridge,
2024.

\bibitem{turing_computable}
Alan M. Turing.
On Computable Numbers, with an Application to the Entscheidungsproblem.
\textit{Proceedings of the London Mathematical Society}, 42(2):230–265, 1936.

\bibitem{girard_linear_logic}
Jean-Yves Girard.
Linear Logic.
\textit{Theoretical Computer Science}, 50(1):1–101, 1987.

\bibitem{pierce_types}
Benjamin C. Pierce.
\textit{Types and Programming Languages}.
MIT Press, 2002.

\bibitem{dennett_intentional}
Daniel C. Dennett.
\textit{The Intentional Stance}.
MIT Press, 1987.

\end{thebibliography}

\end{document}
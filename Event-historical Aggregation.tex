\documentclass[11pt]{article}

% --------------------------------------------------
% Packages
% --------------------------------------------------
\usepackage[T1]{fontenc}
\usepackage{lmodern}
\usepackage{geometry}
\usepackage{microtype}
\usepackage{setspace}
\usepackage{amsmath,amssymb,amsthm,mathtools}
\usepackage{csquotes}
\usepackage{hyperref}

\geometry{margin=1in}
\setstretch{1.15}

% --------------------------------------------------
% Theorem Environments
% --------------------------------------------------
\newtheorem{definition}{Definition}
\newtheorem{proposition}{Proposition}
\newtheorem{theorem}{Theorem}
\newtheorem{remark}{Remark}

% --------------------------------------------------
% Macros
% --------------------------------------------------
\newcommand{\POP}{\textsc{pop}}
\newcommand{\LINK}{\textsc{link}}
\newcommand{\MERGE}{\textsc{merge}}
\newcommand{\COLLAPSE}{\textsc{collapse}}
\newcommand{\REFUSE}{\textsc{refuse}}

% --------------------------------------------------
% Title
% --------------------------------------------------
\title{Event--Historical Aggregation:\\
Map--Reduce as Commitment in Spherepop}
\author{Flyxion}
\date{}

\begin{document}
\maketitle

% ==================================================
\section*{Abstract}
% ==================================================

This essay develops an event--historical semantics for distributed aggregation,
recasting the classical map--reduce paradigm as a process of irreversible
commitment rather than value computation. Working within the Spherepop calculus,
computation is understood as a history of authorized, replayable transformations
whose meaning lies in the constraints they impose on future possibilities.
Mapping corresponds to the construction of local event histories that culminate
in committed summaries, while reduction is realized as a sequence of
\textsc{merge} events that irreversibly bind these local commitments into a
durable global object.

The paper introduces a formal semantics for event--historical aggregation,
derives algebraic correctness properties for provenance--guarded reducers, and
shows how associativity, commutativity, and idempotence emerge only after
quotienting histories by authorized collapse. Refusal is treated as a first--class
semantic operator, rendering invalid aggregations ontologically inadmissible
rather than merely erroneous. The resulting framework subsumes classical
map--reduce and CRDT semantics as special cases while extending them with
auditability, policy enforcement, and controlled forgetting.

% ==================================================
\section{Introduction}
% ==================================================

Map--reduce occupies a foundational role in distributed systems. Its enduring
appeal lies in a simple abstraction: large computations can be decomposed into
local mappings over shards of data and a reduction that aggregates partial
results. Correctness, scalability, and parallelism are achieved by requiring the
reducer to satisfy algebraic properties such as associativity and, in many
settings, commutativity and idempotence.

Despite its practical success, the classical formulation of map--reduce is
conceptually thin. Aggregation is treated as a function from values to values,
and the identity of the result is exhausted by its final numerical or symbolic
content. Provenance, policy constraints, and correctness conditions beyond pure
algebra are relegated to informal assumptions or external mechanisms.

This paper argues that this value--centric framing obscures what aggregation
actually does in practice. Real distributed aggregation is not merely the
calculation of a number; it is the construction of a durable commitment about
which inputs have been accepted, which combinations are permitted, and which
alternatives have been ruled out. These commitments persist across time, shape
future computation, and often carry normative or policy significance.

Spherepop provides a calculus in which such commitments are explicit. In
Spherepop, computation is event--sourced: objects are not defined by their
current state but by the irreversible history of authorized events that brought
them into being. Within this setting, map--reduce admits a natural and precise
reinterpretation. The map phase constructs many local histories, each culminating
in a committed summary. The reduce phase composes these histories through
authorized merges, producing a global object whose identity is inseparable from
the commitments encoded in its lineage.

The contribution of this essay is threefold. First, it provides a rigorous
event--historical semantics for map--reduce in Spherepop. Second, it derives the
algebraic correctness conditions of reducers as emergent properties of
provenance--guarded merge operations rather than as primitive assumptions.
Third, it demonstrates how refusal and controlled forgetting can be integrated
directly into the semantics of aggregation, yielding a framework that is both
more expressive and more faithful to real distributed practice.

% ==================================================
\section{Computation as Event History}
% ==================================================

Traditional programming models identify computation with the evaluation of
functions over values. A computation is correct if it produces the right output,
and the internal steps leading to that output are, at best, an implementation
detail. In contrast, Spherepop treats computation as the construction of a
history. An object exists if and only if there is an authorized sequence of
events that introduces it, transforms it, and commits it as a fact of the world.

This event--historical stance entails a fundamental shift in semantics. The
meaning of an object is not given by its extensional content alone, but by the
constraints its history imposes on future computation. A committed object rules
out alternative possibilities: once a \POP\ event has occurred, the system may
no longer behave as though the committed fact were absent or different.

Within this framework, aggregation cannot be understood as a mere operation on
values. Aggregation constructs an object whose identity records which inputs
have been bound together and under what conditions. The question \enquote{what
value did we compute?} is secondary to the question \enquote{what commitments
have we made?}

Map--reduce, reinterpreted in this light, is a disciplined method for constructing
global commitments from local ones. Mapping produces locally committed facts;
reduction binds those facts together into a larger commitment that persists as a
world--bearing artifact.

% ==================================================
\section{Mapping as Local Commitment}
% ==================================================

The map phase in Spherepop begins with the introduction of input shards as
authorized objects. Each shard is established as a legitimate participant in
the computation through a commitment event. A mapper then performs some local
analysis over the shard. This analysis may be pure and reversible at the level of
ordinary computation, but its output becomes meaningful only when it is
committed.

Formally, a mapper produces a summary object whose existence is fixed by a
\POP\ event. A subsequent \LINK\ event binds the summary to its originating
shard, establishing provenance as a structural property of the object rather
than as external metadata. Once these events have occurred, the mapper output
cannot be silently rewritten or replaced without creating a new history.

The result of mapping is therefore a collection of objects that are independent
in the sense that their histories share no reduction commitments, yet grounded
in the sense that each carries an explicit lineage to the data that produced it.
These objects form the raw material for reduction, not as interchangeable
values, but as historically situated facts.

% ==================================================
\section{Event--Historical Semantics}
% ==================================================

To reason rigorously about aggregation in Spherepop, we now introduce a formal
semantics in which computation is modeled as the construction and extension of
event histories. The purpose of this semantics is not to reproduce classical
denotational or operational accounts, but to make precise the sense in which
aggregation derives its meaning from irreversible commitment rather than from
terminal values.

\subsection{Histories and Authorization}

Let $\mathcal{E}$ denote a fixed set of primitive event types, including
$\POP$, $\LINK$, $\MERGE$, $\COLLAPSE$, and $\REFUSE$. A history is a finite,
totally ordered sequence
\[
H = (e_1, e_2, \ldots, e_n),
\]
where each $e_i \in \mathcal{E}$. Histories are constructed incrementally: at
each step, a candidate event may or may not be admissible, depending on the
history constructed so far.

Authorization is taken as a primitive relation
\[
\mathsf{auth}(H, e),
\]
which holds when the event $e$ is permitted as an extension of the history $H$.
Authorization subsumes typing constraints, policy rules, domain invariants, and
resource conditions. Crucially, authorization is evaluated with respect to the
entire prefix history, not merely the current state of some object.

\begin{definition}[Admissible Extension]
Given a history $H$ and an event $e$, the extension $H \cdot e$ is admissible if
and only if $\mathsf{auth}(H, e)$ holds.
\end{definition}

If an event is not authorized, no successor history exists. This absence is not
modeled as an error state, but as a lack of admissible continuation.

\subsection{Replay Semantics}

The meaning of a history is given by its replay. Let
\[
\llbracket H \rrbracket
\]
denote the object induced by replaying the events of $H$ in order. Replay is a
partial function from histories to objects: it is defined only for admissible
histories. Two histories that differ syntactically may nevertheless replay to
objects that are semantically equivalent under collapse, as discussed below.

Objects are therefore intensional: their identity depends on the history that
produced them, not merely on an extensional value.

\begin{remark}
This contrasts with state--based semantics, where histories are erased once their
effects are incorporated into a state. In Spherepop, history is never erased
implicitly; it can only be transformed or quotienting via explicit events.
\end{remark}

\subsection{Commitment and Monotonicity}

Commitment is introduced through the $\POP$ event. Informally, a $\POP$ asserts
that an object now exists as a fact of the world and may be referenced by future
events.

\begin{definition}[Commitment]
An object $x$ is committed at history $H$ if $H$ contains a $\POP(x)$ event.
\end{definition}

Commitment is irreversible in the following precise sense.

\begin{proposition}[Monotonicity of Commitment]
If an object $x$ is committed at history $H$, then for any admissible extension
$H' \succeq H$, $x$ remains committed in $H'$.
\end{proposition}

\begin{proof}
By definition, commitment is witnessed by the presence of a $\POP(x)$ event in
the history. Since histories are extended only by appending events, no admissible
extension can remove or negate a prior $\POP(x)$. Therefore, once committed, $x$
remains committed in all admissible future histories.
\end{proof}

This monotonicity property formalizes the intuitive notion that commitment rules
out alternative futures in which the committed fact did not occur.

\subsection{Refusal as Semantic Partiality}

Refusal plays a central role in the semantics of aggregation. Rather than being an
exceptional outcome within the system, refusal is modeled as the absence of an
authorized extension.

\begin{definition}[Refusal]
Given a history $H$ and a candidate event $e$, refusal occurs when
$\mathsf{auth}(H, e)$ does not hold. In this case, no history $H \cdot e$ exists.
\end{definition}

In particular, a $\REFUSE$ annotation may be used to document the reason for
inadmissibility, but it does not create a successor object. Refusal is therefore
ontological rather than operational: the forbidden aggregation simply cannot
happen.

\begin{theorem}[No Recovery from Refusal]
If an event $e$ is refused at history $H$, then there exists no history $H'$
extending $H$ in which the effects of $e$ are realized.
\end{theorem}

\begin{proof}
By definition, refusal means that $H \cdot e$ is not admissible. Since all
histories are constructed by admissible extensions, and no alternative event can
realize the same effects without violating authorization, there exists no
admissible history extending $H$ in which the effects attributed to $e$ occur.
\end{proof}

This theorem captures the distinction between refusal and error handling. An
error may be corrected or compensated for in later computation. A refusal
precludes the existence of any history in which the forbidden act took place.

\subsection{Collapse and Equivalence of Histories}

While histories are never implicitly erased, Spherepop permits explicit,
authorized forgetting through $\COLLAPSE$ events. Collapse defines an equivalence
relation on histories that preserves selected invariants while discarding others.

\begin{definition}[Collapse Equivalence]
Given a set of invariants $I$, a collapse operation induces an equivalence
relation $\equiv_I$ on histories such that $H \equiv_I H'$ if and only if
$\llbracket H \rrbracket$ and $\llbracket H' \rrbracket$ agree on all invariants
in $I$.
\end{definition}

Collapse is irreversible and idempotent: once distinctions are forgotten, they
cannot be recovered by further admissible extensions.

\begin{proposition}[Idempotence of Collapse]
If $H$ collapses to $H'$ under invariants $I$, then collapsing $H'$ again under
$I$ yields $H'$.
\end{proposition}

\begin{proof}
By definition, $H'$ already satisfies the equivalence constraints induced by $I$.
A second collapse under the same invariants cannot identify any further
distinctions and therefore leaves $H'$ unchanged.
\end{proof}

This completes the core semantic framework. With histories, authorization,
commitment, refusal, and collapse formally defined, we can now turn to the
semantics of aggregation itself and derive the algebraic properties of reducers
as theorems rather than assumptions.

% ==================================================
\section{Merge Semantics and Reducer Construction}
% ==================================================

With the event--historical framework in place, we now give a precise semantics
for reduction by merge. The central claim of this section is that reducer
correctness properties traditionally assumed as algebraic axioms arise, in
Spherepop, as consequences of authorization, provenance constraints, and
collapse.

\subsection{Reducer Objects and Provenance}

A reducer object is an object whose history includes one or more merge events.
Semantically, a reducer object represents a commitment that binds together a
collection of mapper outputs. Each reducer therefore carries both a payload and
a provenance component.

\begin{definition}[Reducer Summary]
A reducer summary is a pair $(v, P)$, where $v$ is a semantic payload and $P$ is a
finite set of shard identifiers indicating provenance.
\end{definition}

The interpretation of the payload $v$ is domain--specific. For example, in a
word--count application it may be a frequency table, while in numerical
aggregation it may be a sum or vector. Provenance, by contrast, has a fixed
semantic role: it constrains admissible composition.

\subsection{Merge as an Authorized Extension}

A merge is an event that takes two existing objects---typically a current reducer
and a mapper output---and produces a new reducer object whose history extends the
histories of its inputs.

\begin{definition}[Merge Authorization]
Let $(v_1, P_1)$ and $(v_2, P_2)$ be reducer summaries. The merge event
$\MERGE((v_1, P_1), (v_2, P_2))$ is authorized if and only if
$P_1 \cap P_2 = \varnothing$.
\end{definition}

When authorized, the merge produces a new summary
\[
(v_1, P_1) \oplus (v_2, P_2) = (v_1 \diamond v_2, P_1 \cup P_2),
\]
where $\diamond$ is a binary operation on payloads. When the authorization
condition fails, the merge is refused and no successor history exists.

\begin{remark}
The authorization condition enforces a strong correctness guarantee: each shard
may contribute at most once to a reducer history. Duplication is therefore not
handled by idempotence at the value level, but excluded at the level of
admissible histories.
\end{remark}

\subsection{Monotonicity and Irreversibility of Reduction}

Reduction by merge is monotone in a precise sense: each authorized merge strictly
extends the set of commitments encoded in the reducer history.

\begin{proposition}[Monotonic Growth of Provenance]
Let $r$ be a reducer summary with provenance $P$. If $r'$ is obtained from $r$ by
an authorized merge with a mapper output whose provenance is $Q$, then the
provenance of $r'$ is $P \cup Q$, and $P \subsetneq P \cup Q$.
\end{proposition}

\begin{proof}
By merge definition, $P \cap Q = \varnothing$ and the resulting provenance is
their union. Since $Q$ is nonempty for any nontrivial mapper output, the union is
a strict superset of $P$.
\end{proof}

This strict growth property implies irreversibility.

\begin{theorem}[Non--Reversibility of Merge]
Let $r'$ be obtained from $r$ by an authorized merge. There exists no admissible
history extending the history of $r'$ in which the commitments introduced by
that merge are undone.
\end{theorem}

\begin{proof}
Undoing the merge would require either removing provenance elements from the
summary or negating the effect of a prior $\POP$ event. Neither operation is
authorized under the event--historical semantics, as histories may only be
extended, not rewritten. Therefore no admissible extension can reverse the
commitments introduced by the merge.
\end{proof}

\subsection{Associativity up to Collapse}

Classical reducers rely on associativity to permit reordering and parallel
combination of inputs. In Spherepop, merge is not strictly associative at the
level of histories, since different merge orders produce distinct event
sequences. Associativity instead emerges at the level of collapsed objects.

\begin{definition}[Associativity up to Collapse]
Let $m_1, m_2, m_3$ be mapper outputs with pairwise disjoint provenance. Merge is
associative up to collapse if the reducer objects obtained by
\[
((m_1 \oplus m_2) \oplus m_3)
\quad \text{and} \quad
(m_1 \oplus (m_2 \oplus m_3))
\]
are equivalent under an authorized collapse that preserves payload and
provenance invariants.
\end{definition}

\begin{theorem}[Associativity of Provenance--Guarded Merge]
If the payload operation $\diamond$ is associative and commutative, then merge is
associative up to collapse.
\end{theorem}

\begin{proof}
Both merge trees produce reducer summaries with identical payload
$v_1 \diamond v_2 \diamond v_3$ and identical provenance
$P_1 \cup P_2 \cup P_3$. The only difference lies in the internal structure of
their histories. A collapse that preserves payload and provenance invariants
identifies these histories, yielding equivalent reducer objects.
\end{proof}

This result justifies parallel and incremental reduction without erasing the
historical information that distinguishes different execution paths.

\subsection{Commutativity and Structural Symmetry}

A similar argument establishes commutativity up to collapse.

\begin{proposition}[Commutativity up to Collapse]
If $\diamond$ is commutative, then for any two mapper outputs $m_1$ and $m_2$ with
disjoint provenance, the reducer objects obtained by merging $m_1$ into $m_2$
and $m_2$ into $m_1$ are equivalent under collapse.
\end{proposition}

\begin{proof}
The two merge orders produce summaries with identical payload and provenance.
As in the associative case, a collapse preserving these invariants identifies
the resulting histories.
\end{proof}

\subsection{Idempotence via Refusal}

Idempotence is often required to tolerate duplicate inputs. In Spherepop, this
property arises from refusal rather than from algebraic absorption.

\begin{theorem}[Idempotence by Inadmissibility]
Let $m$ be a mapper output with provenance $P$. Attempting to merge $m$ into a
reducer whose provenance already includes $P$ is refused. Consequently, each
mapper output can contribute to a reducer at most once.
\end{theorem}

\begin{proof}
If $P \subseteq P_r$, then $P \cap P_r \neq \varnothing$, violating the merge
authorization condition. The merge is therefore inadmissible, and no successor
history exists in which the duplicate contribution occurs.
\end{proof}

This completes the derivation of the classical algebraic properties of reducers
as consequences of event--historical semantics rather than as primitive axioms.
Reduction is thus revealed as a process of constructing increasingly constrained
objects through authorized composition.

% ==================================================
\section{Incremental and Streaming Reduction}
% ==================================================

A salient advantage of event--historical aggregation is that it supports
incremental and streaming computation without additional machinery. Because
each merge yields a new committed reducer object rather than mutating an
existing one, aggregation naturally unfolds as a monotone sequence of
commitments indexed by time.

Let $\{m_i\}_{i \ge 1}$ be a potentially unbounded stream of mapper outputs with
pairwise disjoint provenance. An incremental reducer constructs a sequence of
reducer objects $\{r_i\}_{i \ge 0}$ such that $r_0$ is an explicitly committed
empty summary and
\[
r_{i+1} = r_i \oplus m_{i+1}
\]
whenever the merge is authorized. Each $r_i$ is a distinct object with its own
history, and the transition from $r_i$ to $r_{i+1}$ is marked by a $\MERGE$
followed by a $\POP$.

\begin{proposition}[Prefix Soundness]
For any $i < j$, the reducer object $r_i$ represents a sound aggregation of the
prefix $\{m_1, \ldots, m_i\}$ of the input stream.
\end{proposition}

\begin{proof}
By construction, $r_i$ is obtained by a sequence of authorized merges starting
from $r_0$ and incorporating exactly the mapper outputs $m_1$ through $m_i$.
Authorization ensures that no mapper output outside this prefix can be included,
and monotonicity of commitment ensures that no included output can be removed.
\end{proof}

This prefix property implies that at any moment the reducer exposes a coherent,
auditable snapshot of the aggregation so far, without requiring global
synchronization or quiescence.

% ==================================================
\section{Checkpointing and Rollback as Event Construction}
% ==================================================

Checkpointing in Spherepop does not require copying state or persisting mutable
variables. To checkpoint a computation is simply to name a particular reducer
object and record its identity. Because reducer objects are immutable once
committed, a checkpoint is a stable reference to a specific history.

Rollback, correspondingly, is not a reversal of time. It is the construction of
a new branch of history that resumes aggregation from an earlier checkpoint.
Formally, let $r_k$ be a previously committed reducer object. A rollback resumes
computation by using $r_k$ as the base reducer for subsequent merges, thereby
producing a new sequence $\{r'_i\}$ whose histories extend that of $r_k$ but not
that of later reducer objects.

\begin{theorem}[Rollback without Reversal]
Rollback does not violate irreversibility: no committed reducer object is
destroyed or undone by resuming from an earlier checkpoint.
\end{theorem}

\begin{proof}
Rollback constructs new histories that extend an earlier committed history.
Later histories remain intact and committed, but are no longer extended along
the new branch. Since no history is shortened or rewritten, irreversibility is
preserved.
\end{proof}

This treatment of rollback aligns aggregation with versioned data structures
and persistent computation, while retaining explicit causal semantics.

% ==================================================
\section{Controlled Forgetting via Collapse}
% ==================================================

While histories are semantically significant, unbounded growth of reducer
histories is neither necessary nor desirable. Spherepop addresses this through
explicit collapse events that authorize forgetting under specified invariants.

A collapse operation takes a reducer object with history $H$ and produces a new
object with history $H'$ such that $H \equiv_I H'$ for a declared set of
invariants $I$. Typical invariants include payload totals and provenance sets,
while internal merge structure is often dispensable.

\begin{proposition}[Safety of Collapse]
If a collapse preserves payload and provenance invariants, then any property of
the reducer that depends only on these invariants is preserved under collapse.
\end{proposition}

\begin{proof}
By definition of collapse equivalence, the replay of $H$ and $H'$ agree on all
invariants in $I$. Any property definable solely in terms of these invariants
therefore has the same truth value for both objects.
\end{proof}

Collapse is itself an irreversible commitment. Once a history has been collapsed,
the distinctions it discards cannot be recovered by further admissible
extensions.

\begin{theorem}[Irreversibility of Forgetting]
There exists no admissible history extending a collapsed history that restores
discarded distinctions.
\end{theorem}

\begin{proof}
Restoring discarded distinctions would require introducing information not
derivable from the preserved invariants. Since admissible extensions may only
operate on existing committed objects and events, and collapse explicitly
forgets those distinctions, no authorized event can reconstruct them.
\end{proof}

This explicit treatment of forgetting distinguishes Spherepop from systems in
which optimization silently erases causal structure.

% ==================================================
\section{Auditability and Lineage Guarantees}
% ==================================================

A principal motivation for event--historical aggregation is auditability. Because
reducer objects retain their histories unless explicitly collapsed, it is always
possible to reconstruct the lineage of an aggregate.

\begin{theorem}[Lineage Completeness]
Let $r$ be a reducer object whose history has not been collapsed with respect to
provenance. Then the set of mapper outputs contributing to $r$ is uniquely
recoverable from its history.
\end{theorem}

\begin{proof}
Each authorized merge introduces explicit links from the new reducer object to
both its prior reducer and the mapper output being incorporated. By traversing
these links recursively, one can enumerate all mapper outputs contributing to
$r$. Authorization ensures that no mapper output appears more than once.
\end{proof}

Even after collapse, auditability can be preserved at a coarser granularity by
retaining provenance invariants. The choice of invariants thus determines the
resolution at which auditability is maintained.

Taken together, incremental reduction, checkpointing, rollback, and controlled
forgetting demonstrate that Spherepop supports the practical requirements of
distributed aggregation while preserving a principled semantic account of
commitment and history.

% ==================================================
\section{Worked Examples of Event--Historical Aggregation}
% ==================================================

This section presents concrete instances of map--reduce expressed in the
event--historical semantics of Spherepop. The goal is to demonstrate that common
distributed aggregation tasks arise naturally from the merge calculus, while
exhibiting properties---such as refusal and explicit provenance---that are
difficult to express in value--centric models.

\subsection{Word Frequency Aggregation}

Consider a corpus partitioned into shards $\{s_i\}$. Each shard contains a finite
sequence of tokens. A mapper processes shard $s_i$ to produce a local frequency
map $h_i : \Sigma \to \mathbb{N}$ together with provenance $P_i = \{i\}$.

The mapper output is committed via a $\POP$ event and linked to its shard. The
reducer aggregates these outputs using merge. The payload operation $\diamond$
is pointwise addition of frequency maps, which is associative and commutative.

\begin{theorem}[Correctness of Word Count Aggregation]
Let $r_n$ be the reducer object obtained by merging mapper outputs
$m_1, \ldots, m_n$ with pairwise disjoint provenance. Then the payload of $r_n$
assigns to each word $w \in \Sigma$ the total number of occurrences of $w$ across
shards $s_1, \ldots, s_n$.
\end{theorem}

\begin{proof}
By induction on $n$. The base case $n=0$ holds trivially for the empty reducer.
Assume correctness for $r_n$. Merging $m_{n+1}$ adds $h_{n+1}$ to the payload via
$\diamond$, yielding correct totals over all $n+1$ shards. Provenance guarantees
that no shard is counted twice.
\end{proof}

Attempting to re--merge a mapper output whose shard has already contributed is
refused, preventing duplication at the semantic level.

\subsection{Group--By Aggregation}

Group--by operations can be expressed by lifting reducers to keyed families. A
mapper emits pairs $(k, m)$, where $k$ is a key and $m$ a mapper output for that
key. The reducer maintains a map from keys to reducer objects, each governed by
its own provenance constraints.

\begin{proposition}[Independent Key Aggregation]
Reducer histories for distinct keys are independent, and merges for one key do
not affect the admissibility of merges for another.
\end{proposition}

\begin{proof}
Provenance sets are disjoint across keys by construction, and merge authorization
is checked per reducer object. Therefore histories evolve independently.
\end{proof}

This decomposition preserves parallelism while maintaining event--historical
auditability per key.

% ==================================================
\section{Relation to CRDT Semantics}
% ==================================================

Conflict--free replicated data types (CRDTs) provide a widely adopted framework
for distributed aggregation without coordination. At first glance, Spherepop
reducers appear similar, as both rely on associative and commutative combination
rules. However, the semantic commitments of the two frameworks differ
fundamentally.

CRDT semantics are state--based or operation--based but ultimately value--centric.
Correctness is expressed as eventual convergence: replicas that receive the same
set of updates, possibly in different orders, converge to the same state. The
causal history of updates is instrumental rather than meaningful.

Spherepop, by contrast, treats histories as first--class semantic objects. Merge
is an irreversible commitment that constructs a new object whose identity
records which inputs were accepted and how. Convergence, when desired, is
realized via explicit collapse rather than implicit erasure of history.

\subsection{Embedding CRDTs into Spherepop}

Despite these differences, CRDTs can be embedded into Spherepop under suitable
restrictions.

\begin{theorem}[CRDT Embedding]
Any state--based CRDT whose join operation is associative, commutative, and
idempotent can be represented as a Spherepop reducer by imposing the following
constraints: refusal rules are disabled, provenance is ignored or collapsed
immediately, and collapse is applied eagerly after each merge.
\end{theorem}

\begin{proof}
Under these constraints, each merge produces a reducer object that immediately
collapses to a canonical representative of its payload state. Since collapse
erases history, only the join--semilattice structure remains. Associativity,
commutativity, and idempotence of the join operation ensure convergence
equivalent to that of the original CRDT.
\end{proof}

This embedding shows that Spherepop strictly generalizes CRDT semantics.

\subsection{Limits of CRDT Expressiveness}

The converse embedding does not hold. CRDTs cannot express refusal, policy
constraints that render certain merges inadmissible, or explicit forgetting as a
committed act.

\begin{theorem}[Strict Extension]
There exist Spherepop reducers that cannot be faithfully represented as CRDTs.
\end{theorem}

\begin{proof}
Consider a reducer with provenance--guarded merge that refuses duplicate shard
contributions. CRDT idempotence absorbs duplicates silently, whereas Spherepop
renders them inadmissible. No CRDT state transition can distinguish refusal from
successful idempotent merge, so no faithful representation exists.
\end{proof}

This result highlights the semantic gain of event--historical aggregation: it
allows distributed computation to encode normative and policy constraints
directly in its semantics.

% ==================================================
\section{Attention as Aggregation on Graphs}
% ==================================================

The map--reduce interpretation developed thus far extends beyond classical batch
aggregation. In this section, we show that the core mechanism of an attention
head, as used in transformer architectures, can be reconstructed as a particular
form of aggregation over a graph. This reconstruction clarifies the structural
role of attention and prepares the ground for an event--historical
reinterpretation.

Let $G = (V, E)$ be a directed graph whose nodes represent tokens, entities, or
states, and whose edges represent potential interactions. Each node $i \in V$
carries an embedding vector $x_i \in \mathbb{R}^d$. A graph neural network (GNN)
updates node representations by aggregating information from neighbors. In its
most general form, a GNN layer computes
\[
h_i = \phi\!\left(x_i, \bigoplus_{j \in \mathcal{N}(i)} \psi(x_i, x_j, e_{ij})\right),
\]
where $\mathcal{N}(i)$ denotes the neighborhood of $i$, $\psi$ computes a
message from neighbor $j$ to $i$, $\oplus$ is an aggregation operator, and $\phi$
produces the updated representation.

Attention mechanisms instantiate a particular choice of $\psi$ and $\oplus$ in
which the contribution of each neighbor is weighted by a learned relevance
score. The critical observation is that attention is not fundamentally a
sequence operation; it is a structured aggregation over a graph whose edges are
dynamically reweighted.

% ==================================================
\section{Constructing an Attention Head on a Graph}
% ==================================================

An attention head may be defined on a graph by introducing three linear maps:
\[
Q : \mathbb{R}^d \to \mathbb{R}^{d_k}, \quad
K : \mathbb{R}^d \to \mathbb{R}^{d_k}, \quad
V : \mathbb{R}^d \to \mathbb{R}^{d_v}.
\]
For each node $i$, the query, key, and value vectors are given by
\[
q_i = Q x_i, \quad k_i = K x_i, \quad v_i = V x_i.
\]

Given a directed edge $(i, j) \in E$, the compatibility between $i$ and $j$ is
computed as
\[
\alpha_{ij} = \frac{\exp(\langle q_i, k_j \rangle)}{\sum_{k \in \mathcal{N}(i)}
\exp(\langle q_i, k_k \rangle)}.
\]
The updated node representation is then
\[
h_i = \sum_{j \in \mathcal{N}(i)} \alpha_{ij} v_j.
\]

This computation is a weighted reduction over the neighborhood of $i$. The
softmax normalization enforces a simplex constraint, ensuring that attention
weights sum to one. From an algebraic perspective, attention is therefore a
convex aggregation of neighbor values, parameterized by learned compatibility
functions.

\begin{remark}
Nothing in this construction depends essentially on linear order. Transformers
can be understood as GNNs operating on complete graphs with positional or causal
masking.
\end{remark}

% ==================================================
\section{Event--Historical Interpretation of Attention}
% ==================================================

The standard attention formulation is value--centric. It computes a weighted sum
and discards the intermediate relevance structure once the output vector is
produced. From an event--historical perspective, this erasure is semantically
significant.

We reinterpret attention in Spherepop by treating each attention head as a
reducer operating over a graph--indexed collection of mapper outputs. Each node
$j$ produces a mapper output consisting of its value vector $v_j$ together with
provenance identifying the source node. The attention head at node $i$ performs
a reduction over these mapper outputs.

The attention weights $\alpha_{ij}$ are not merely coefficients but encode a
policy of admissible influence. In Spherepop terms, they define which merges are
authorized and with what relative strength. A sharp attention distribution
corresponds to a near--exclusive commitment to a small subset of inputs, while a
diffuse distribution corresponds to a weaker, more distributed commitment.

\begin{definition}[Attention as Weighted Merge]
An attention head induces a merge operation in which the payload combination
operator $\diamond$ is a weighted sum, and authorization is governed by the
attention mask and normalization constraints.
\end{definition}

Under this interpretation, computing $h_i$ corresponds to constructing a reducer
object whose history records which neighbors were attended to and under what
weights. Standard transformers immediately collapse this history, retaining only
the final vector. Spherepop makes this collapse explicit and optional.

% ==================================================
\section{Attention, Refusal, and Masking}
% ==================================================

Attention masks, commonly used to enforce causality or sparsity, acquire a
natural interpretation as refusal rules. If an edge $(i, j)$ is masked, then the
corresponding merge is inadmissible: node $j$ cannot contribute to the reducer
history of node $i$.

\begin{proposition}[Masking as Refusal]
Let $(i, j)$ be a masked edge. Then any attempted merge incorporating $v_j$ into
the reducer at $i$ is refused.
\end{proposition}

\begin{proof}
Masking enforces $\alpha_{ij} = 0$ by construction. In event--historical terms,
this corresponds to the absence of authorization for the merge event involving
$j$. No admissible history exists in which $j$ contributes to $i$.
\end{proof}

This reframing highlights a limitation of standard attention: refusal is encoded
numerically rather than structurally. A masked edge contributes a zero vector,
but the system cannot distinguish between principled exclusion and accidental
irrelevance.

% ==================================================
\section{Multi--Head Attention as Parallel Reducers}
% ==================================================

Multi--head attention decomposes aggregation into multiple parallel reducers,
each operating with its own query, key, and value projections. In Spherepop
terms, each head constructs a distinct reducer history over the same set of
mapper outputs.

\begin{proposition}[Independence of Attention Heads]
Reducer histories corresponding to distinct attention heads are independent and
may be collapsed or audited separately.
\end{proposition}

\begin{proof}
Each head uses a distinct payload operation determined by its projections
$(Q_h, K_h, V_h)$. Since no merge events are shared across heads, their histories
are disjoint and evolve independently.
\end{proof}

The concatenation and projection step following multi--head attention may be
understood as a higher--level merge that binds these parallel reducer objects
into a single representation, again followed by collapse in standard practice.

% ==================================================
\section{Limits of Value--Centric Attention}
% ==================================================

From the event--historical viewpoint, standard attention mechanisms systematically
erase the very information that makes aggregation meaningful: which inputs were
considered, which were excluded, and how strongly each influenced the result.
All such structure is collapsed immediately into a vector.

This explains a number of empirical phenomena observed in transformer models,
including sensitivity to prompt phrasing, instability under long--horizon
composition, and difficulty enforcing hard constraints. These are not merely
optimization issues, but consequences of treating attention as a value
computation rather than as a commitment--forming process.

Spherepop suggests an alternative research direction: attention mechanisms that
retain, at least partially, their event histories, enabling explicit refusal,
auditable influence, and controlled forgetting.

% ==================================================
\section{A Categorical Semantics of Event--Historical Aggregation}
% ==================================================

The event--historical semantics developed in the preceding sections admits a
natural categorical formulation that clarifies the structural commitments of
Spherepop and unifies map--reduce, streaming aggregation, and attention under a
single formal lens. This categorical perspective does not replace the
event--level semantics; rather, it abstracts from it while preserving the
essential asymmetries introduced by commitment, refusal, and collapse.

\subsection{The Category of Event Histories}

We define a category $\mathcal{H}$ whose objects are event histories modulo
authorized collapse. Two histories are identified when they are equivalent under
a declared set of invariants. Morphisms in $\mathcal{H}$ are authorized
extensions of histories: a morphism $H \to H'$ exists if and only if $H'$ can be
constructed from $H$ by a finite sequence of authorized events.

Identity morphisms correspond to empty extensions, and composition is given by
concatenation of event sequences. Because authorization is history--dependent,
composition in $\mathcal{H}$ is partial: not all pairs of morphisms are
composable.

\begin{remark}
This partiality is essential. It encodes refusal as the absence of a morphism,
rather than as a morphism to an error object. Invalid aggregations therefore do
not merely fail; they are unrepresentable.
\end{remark}

\subsection{Merge as a Partial Monoidal Structure}

Merge induces a monoidal structure on $\mathcal{H}$, but one that is partial and
policy--dependent. When two objects $H_1$ and $H_2$ have compatible provenance,
their tensor product $H_1 \otimes H_2$ is defined as the history obtained by
merging them. The empty reducer object serves as the monoidal unit.

Associativity and symmetry of this tensor hold only up to collapse. That is,
$\mathcal{H}$ is not a strict monoidal category but a monoidal category modulo a
quotient functor induced by collapse. This precisely mirrors the earlier result
that reducer correctness properties hold only after identifying histories that
preserve the same invariants.

\subsection{Refusal as Non--Existence of Morphisms}

In this categorical setting, refusal corresponds to the non--existence of a
morphism. If merging two objects violates authorization constraints, then no
morphism exists that would compose them. This design choice has important
consequences. It ensures that invalid constructions cannot be accidentally
composed with subsequent operations, and it renders correctness a structural
property rather than a runtime condition.

\subsection{Collapse as a Quotient Functor}

Collapse is formalized as a functor
\[
C_I : \mathcal{H} \to \mathcal{H}_I
\]
that maps histories to equivalence classes preserving a chosen set of
invariants $I$. This functor is idempotent and non--invertible. Applying collapse
forgets distinctions permanently, but the fact that forgetting occurred remains
visible in the history as a committed event.

Different choices of $I$ yield different quotient categories, corresponding to
different trade--offs between fidelity and tractability.

% ==================================================
\section{Map--Reduce and Attention as Monoidal Folding}
% ==================================================

From the categorical perspective, map--reduce is a fold over a family of objects
in $\mathcal{H}$. Mapping produces a collection of independent objects, while
reduction folds them using the monoidal structure induced by merge. Unlike
classical folds, this folding operation is constrained by authorization and may
be partial.

Attention mechanisms fit naturally into this picture. An attention head is a
parameterized fold over a graph--indexed family of objects, where the fold
weights are determined by learned compatibility functions. Standard transformers
instantiate a fold followed immediately by collapse, erasing the event
structure.

Multi--head attention corresponds to parallel folds, whose results are
themselves merged at a higher level. This hierarchical folding structure mirrors
the layered architecture of modern neural networks while exposing the points at
which semantic commitments are made and then forgotten.

% ==================================================
\section{Conclusion and Future Directions}
% ==================================================

This essay has developed an event--historical account of distributed aggregation,
recasting map--reduce as a process of irreversible commitment rather than as a
function on values. Within the Spherepop calculus, mapping constructs local
histories that culminate in committed summaries, while reduction composes these
histories through authorized merges. Classical algebraic properties of reducers
emerge not as axioms but as theorems, holding only up to authorized collapse.

By treating refusal as semantic partiality and collapse as explicit forgetting,
the framework accommodates policy constraints, auditability, and controlled
abstraction as first--class concerns. Streaming aggregation, checkpointing, and
rollback arise naturally from the immutability of committed histories.

Extending the analysis to attention mechanisms reveals that attention is itself a
form of aggregation whose semantic structure is obscured by value--centric
implementations. Interpreting attention as event--historical folding exposes the
cost of immediate collapse and suggests new directions for architectures that
retain, reason over, and selectively forget their own commitments.

Several avenues for future work present themselves. One direction is the design
of attention mechanisms and graph neural networks that preserve partial event
histories to enable explicit refusal and auditability. Another is the
integration of event--historical aggregation with learning dynamics, treating
training itself as a sequence of commitments rather than as gradient descent in
parameter space. Finally, the categorical framework developed here invites
connections to traced monoidal categories, persistent data structures, and
semantics of irreversible computation.

More broadly, event--historical aggregation offers a way to align distributed
systems, machine learning architectures, and semantic computation around a
shared notion of meaning as constraint. Aggregates are no longer ephemeral
values, but durable objects whose identities encode how the world was allowed to
become the way it is.

\newpage
% ==================================================
\appendix
\section{Algebraic Properties of Provenance--Guarded Reducers}
% ==================================================

This appendix provides formal proofs of the algebraic properties attributed to
Spherepop reducers in the main text. The objective is to show that associativity,
commutativity, and idempotence arise as derived properties of event--historical
constraints, rather than as axioms imposed on reducer functions.

% --------------------------------------------------
\subsection{Reducer Domain and Partiality}
% --------------------------------------------------

Let $\mathcal{S}$ denote a countable set of shard identifiers. A reducer summary
is a pair $(v, P)$ where $v \in V$ is a payload drawn from some carrier set $V$
and $P \subseteq \mathcal{S}$ is a finite provenance set.

Let $\diamond : V \times V \to V$ be a binary operation on payloads.

\begin{definition}[Admissible Merge]
The merge operation $\oplus$ is a partial binary operation on reducer summaries
defined by
\[
(v_1, P_1) \oplus (v_2, P_2)
\]
if and only if $P_1 \cap P_2 = \varnothing$, in which case
\[
(v_1, P_1) \oplus (v_2, P_2) := (v_1 \diamond v_2, P_1 \cup P_2).
\]
If $P_1 \cap P_2 \neq \varnothing$, the merge is undefined.
\end{definition}

Undefinedness corresponds exactly to refusal in the event--historical semantics.

% --------------------------------------------------
\subsection{Monotonicity and Strict Growth}
% --------------------------------------------------

\begin{proposition}[Strict Provenance Growth]
If $(v', P') = (v, P) \oplus (w, Q)$ is defined, then $P \subsetneq P'$.
\end{proposition}

\begin{proof}
Since $Q$ is nonempty for any mapper output and $P \cap Q = \varnothing$, we have
$P' = P \cup Q$ with $P \subsetneq P'$.
\end{proof}

\begin{corollary}[No Cycles in Reducer Histories]
Reducer histories are acyclic with respect to merge ancestry.
\end{corollary}

\begin{proof}
Each merge strictly increases provenance size. Since provenance sets are finite,
no reducer can be its own ancestor.
\end{proof}

% --------------------------------------------------
\subsection{Associativity up to Collapse}
% --------------------------------------------------

Let $m_i = (v_i, P_i)$ for $i \in \{1,2,3\}$ with pairwise disjoint provenance.

\begin{theorem}[Associativity up to Provenance--Preserving Equivalence]
If $\diamond$ is associative, then
\[
(m_1 \oplus m_2) \oplus m_3
\quad \text{and} \quad
m_1 \oplus (m_2 \oplus m_3)
\]
are both defined and equivalent under the equivalence relation
\[
(v, P) \sim (v', P') \iff v = v' \;\wedge\; P = P'.
\]
\end{theorem}

\begin{proof}
Both expressions are defined because provenance sets are pairwise disjoint.
Associativity of $\diamond$ ensures
\[
(v_1 \diamond v_2) \diamond v_3 = v_1 \diamond (v_2 \diamond v_3),
\]
and set union is associative, yielding identical provenance
$P_1 \cup P_2 \cup P_3$.
\end{proof}

\begin{remark}
The histories producing these results differ syntactically, but collapse
identifies them by quotienting over internal merge structure.
\end{remark}

% --------------------------------------------------
\subsection{Commutativity}
% --------------------------------------------------

\begin{theorem}[Commutativity up to Collapse]
If $\diamond$ is commutative, then for any two summaries with disjoint provenance,
\[
m_1 \oplus m_2 \sim m_2 \oplus m_1.
\]
\end{theorem}

\begin{proof}
Commutativity of $\diamond$ implies $v_1 \diamond v_2 = v_2 \diamond v_1$, and
set union is commutative. Provenance equality follows.
\end{proof}

% --------------------------------------------------
\subsection{Idempotence via Inadmissibility}
% --------------------------------------------------

\begin{theorem}[Idempotence by Partiality]
For any reducer summary $m = (v, P)$,
\[
m \oplus m
\]
is undefined.
\end{theorem}

\begin{proof}
Since $P \cap P = P \neq \varnothing$, the admissibility condition fails.
\end{proof}

\begin{corollary}[Semantic Idempotence]
Each mapper output contributes to a reducer at most once.
\end{corollary}

\begin{remark}
Unlike algebraic idempotence, this property is enforced structurally. Duplicate
inputs do not collapse to the same value; they are forbidden from occurring.
\end{remark}

% --------------------------------------------------
\subsection{Resulting Algebraic Structure}
% --------------------------------------------------

\begin{theorem}[Reducer Quotient Structure]
Let $\sim$ be the provenance--preserving equivalence relation. The quotient
$(\mathcal{R}/\!\sim,\oplus)$ forms a commutative monoid.
\end{theorem}

\begin{proof}
Associativity and commutativity hold up to $\sim$. The empty reducer
$(v_0, \varnothing)$ serves as identity. Partiality disappears in the quotient
because undefined merges correspond to excluded elements.
\end{proof}

\begin{remark}
This monoid exists only after quotienting by collapse. At the level of histories,
no total algebraic structure exists.
\end{remark}

% ==================================================
\section{Embedding CRDT Semantics into Event--Historical Aggregation}
% ==================================================

This appendix formalizes the relationship between conflict--free replicated data
types (CRDTs) and Spherepop reducers. The central results are twofold. First,
CRDTs embed into Spherepop under specific semantic restrictions. Second, this
embedding is strict: there exist Spherepop reducers that cannot be faithfully
represented as CRDTs.

% --------------------------------------------------
\subsection{Preliminaries on CRDTs}
% --------------------------------------------------

We consider state--based CRDTs, as they admit the clearest algebraic
characterization. A state--based CRDT is defined by a triple $(S, \sqcup, \leq)$
where $(S, \leq)$ is a join--semilattice and $\sqcup : S \times S \to S$ is the
least upper bound operation.

Updates are monotone with respect to $\leq$, and replicas periodically exchange
states, merging them via $\sqcup$. Correctness is expressed as convergence:
replicas that have received the same set of updates converge to the same state,
independent of order or duplication.

% --------------------------------------------------
\subsection{CRDTs as Collapsed Reducers}
% --------------------------------------------------

We now show that CRDT semantics arise as a degenerate case of Spherepop
aggregation in which history is aggressively collapsed.

\begin{definition}[CRDT--Compatible Spherepop Reducer]
A Spherepop reducer is CRDT--compatible if it satisfies the following
constraints:
\begin{enumerate}
\item The payload carrier $V$ forms a join--semilattice.
\item The merge payload operation $\diamond$ coincides with $\sqcup$.
\item Provenance is either absent or collapsed immediately after each merge.
\item No refusal rules are enforced; all merges are admissible.
\end{enumerate}
\end{definition}

\begin{theorem}[CRDT Embedding]
For every state--based CRDT $(S, \sqcup, \leq)$, there exists a CRDT--compatible
Spherepop reducer whose collapsed semantics are equivalent to the CRDT's
convergence semantics.
\end{theorem}

\begin{proof}
Construct a Spherepop reducer with payload carrier $V = S$ and merge operation
$\diamond = \sqcup$. Let each update be represented as a mapper output whose
payload is the updated state. Because all merges are admissible and collapse is
applied eagerly, reducer histories never retain internal structure. After each
merge, the reducer collapses to the canonical join of all received states.

Associativity, commutativity, and idempotence of $\sqcup$ guarantee that the
resulting payload is independent of merge order and duplication, matching CRDT
convergence semantics.
\end{proof}

\begin{remark}
Under this embedding, Spherepop contributes no additional expressive power. All
event--historical structure is erased immediately, and aggregation reduces to
value computation.
\end{remark}

% --------------------------------------------------
\subsection{Eager Collapse and Event Erasure}
% --------------------------------------------------

CRDT convergence may be characterized as a discipline of mandatory forgetting.

\begin{proposition}[Convergence as Eager Collapse]
CRDT convergence is equivalent to applying a collapse operation after every
merge that preserves only the payload invariant.
\end{proposition}

\begin{proof}
Since only the join--semilattice value matters for CRDT correctness, all other
historical distinctions are semantically irrelevant. Eager collapse enforces this
irrelevance by quotienting histories immediately.
\end{proof}

This observation clarifies the philosophical difference between the two
frameworks: CRDTs treat history as an implementation detail, while Spherepop
treats history as semantically primary.

% --------------------------------------------------
\subsection{Non--Embeddability of Refusal}
% --------------------------------------------------

We now establish that the CRDT embedding is strict.

\begin{theorem}[Non--Embeddability of Refusal]
There exists no faithful embedding of provenance--guarded Spherepop reducers into
CRDT semantics.
\end{theorem}

\begin{proof}
Consider a Spherepop reducer whose merge operation refuses to combine summaries
with overlapping provenance. Suppose, for contradiction, that this reducer could
be faithfully embedded into a CRDT. In CRDT semantics, duplicate updates are
either absorbed by idempotence or merged redundantly without semantic effect.

In Spherepop, by contrast, duplicate provenance renders the merge inadmissible.
There exists no state transition corresponding to refusal: CRDTs cannot
distinguish between forbidden composition and harmless idempotent merge.
Therefore no embedding can preserve the refusal semantics.
\end{proof}

\begin{corollary}[Policy Non--Expressibility]
Any aggregation policy that forbids certain combinations of inputs is
inexpressible in CRDT semantics.
\end{corollary}

% --------------------------------------------------
\subsection{Non--Monotone Policies}
% --------------------------------------------------

CRDTs fundamentally rely on monotonic growth of state. Spherepop does not.

\begin{theorem}[Non--Monotone Constraints]
There exist Spherepop reducers whose admissibility conditions are non--monotone
with respect to payload growth, and which therefore cannot be represented by
CRDTs.
\end{theorem}

\begin{proof}
Consider a reducer that refuses merges once provenance size exceeds a fixed
threshold, enforcing a capacity constraint. This refusal depends on accumulated
history rather than on monotonic payload growth. No CRDT state transition can
encode such a constraint without violating convergence.
\end{proof}

% --------------------------------------------------
\subsection{Summary}
% --------------------------------------------------

CRDTs appear within Spherepop as a special case characterized by eager collapse,
absence of refusal, and exclusive concern with payload convergence. Spherepop
strictly generalizes this model by allowing history to remain meaningful,
enforcing policy through inadmissibility, and treating forgetting as an explicit
event rather than an implicit consequence of convergence.

% ==================================================
\section{Categorical Semantics of Event--Historical Aggregation}
% ==================================================

This appendix presents a categorical formulation of the event--historical
semantics underlying Spherepop aggregation. The purpose is to make precise the
structural claims made in the main text, namely that aggregation is a form of
monoidal composition constrained by authorization, refusal, and collapse, and
that attention mechanisms correspond to traced folds within this structure.

% --------------------------------------------------
\subsection{The Category of Histories}
% --------------------------------------------------

Let $\mathcal{H}_0$ be the collection of all admissible event histories. We define
a category $\mathcal{H}$ as follows.

\begin{definition}[Objects]
Objects of $\mathcal{H}$ are equivalence classes of histories under a chosen
collapse equivalence relation $\equiv_I$, where $I$ is a fixed set of invariants.
\end{definition}

\begin{definition}[Morphisms]
Given objects $[H]$ and $[H']$, a morphism $[H] \to [H']$ exists if and only if
there exists a representative history $H'' \in [H']$ such that $H''$ is an
authorized extension of a representative of $[H]$.
\end{definition}

Identity morphisms are given by empty extensions. Composition is given by
concatenation of event sequences when authorization permits.

\begin{proposition}[Well--Definedness]
Morphisms in $\mathcal{H}$ are well defined with respect to the equivalence
relation $\equiv_I$.
\end{proposition}

\begin{proof}
If $H \equiv_I H'$ and $H \preceq K$ is an authorized extension, then there exists
a corresponding extension $H' \preceq K'$ such that $K \equiv_I K'$, since
collapse preserves admissibility of future events that depend only on
invariants in $I$.
\end{proof}

% --------------------------------------------------
\subsection{Partiality and Refusal}
% --------------------------------------------------

\begin{theorem}[Partiality of Composition]
$\mathcal{H}$ is in general a partial category: composition of morphisms is not
total.
\end{theorem}

\begin{proof}
Authorization constraints depend on full history. There exist morphisms
$[H] \to [H']$ and $[H'] \to [H'']$ such that the corresponding concatenation of
events violates authorization. In this case, no composite morphism exists.
\end{proof}

\begin{remark}
Refusal corresponds categorically to the absence of a morphism, not to a morphism
into a distinguished error object.
\end{remark}

% --------------------------------------------------
\subsection{Merge as a Partial Monoidal Product}
% --------------------------------------------------

We now show that merge induces a monoidal structure on $\mathcal{H}$.

\begin{definition}[Tensor Product]
For objects $[H_1]$ and $[H_2]$, define $[H_1] \otimes [H_2]$ to be the object
represented by the history obtained by merging $H_1$ and $H_2$, if such a merge
is authorized. If the merge is not authorized, the tensor is undefined.
\end{definition}

\begin{theorem}[Partial Monoidal Structure]
$(\mathcal{H}, \otimes)$ is a partial monoidal category with unit object
$[H_\emptyset]$, the empty reducer history.
\end{theorem}

\begin{proof}
The empty history acts as a unit since merging with it introduces no provenance.
Associativity holds up to collapse by the associativity--up--to--equivalence
results proven in Appendix A. Partiality follows from authorization constraints.
\end{proof}

\begin{corollary}[Symmetry up to Collapse]
If merge payload operations are commutative, then $\mathcal{H}$ admits a symmetry
isomorphism $[H_1] \otimes [H_2] \cong [H_2] \otimes [H_1]$ up to collapse.
\end{corollary}

% --------------------------------------------------
\subsection{Collapse as a Quotient Functor}
% --------------------------------------------------

\begin{definition}[Collapse Functor]
Let $\mathcal{H}_I$ denote the category of histories collapsed with respect to
invariants $I$. The collapse functor
\[
C_I : \mathcal{H} \to \mathcal{H}_I
\]
maps each object to its equivalence class under $\equiv_I$ and each morphism to
the induced morphism between equivalence classes.
\end{definition}

\begin{theorem}[Idempotence and Non--Invertibility]
$C_I$ is idempotent and non--invertible.
\end{theorem}

\begin{proof}
Idempotence follows because collapsing twice with respect to the same invariants
introduces no further identifications. Non--invertibility follows because
distinct histories may be identified under collapse, and no functor can recover
the discarded distinctions.
\end{proof}

\begin{remark}
Different choices of $I$ yield different quotient categories, corresponding to
different semantic resolutions.
\end{remark}

% --------------------------------------------------
\subsection{Attention as Traced Monoidal Folding}
% --------------------------------------------------

We now formalize attention mechanisms within this categorical setting.

Let $G = (V, E)$ be a directed graph. For each node $i \in V$, let
$\{H_{j \to i}\}_{j \in \mathcal{N}(i)}$ be histories representing mapper outputs
from neighbors.

\begin{definition}[Attention Fold]
An attention head at node $i$ is a fold
\[
\mathsf{att}_i : \bigotimes_{j \in \mathcal{N}(i)} [H_{j \to i}] \to [H_i]
\]
parameterized by compatibility weights, followed by an optional collapse.
\end{definition}

\begin{theorem}[Attention as Traced Monoidal Operation]
Multi--layer attention networks correspond to traced monoidal structures in
$\mathcal{H}$, where residual connections induce trace operators.
\end{theorem}

\begin{proof}
Residual connections feed outputs back as inputs in subsequent layers. In a
monoidal category, such feedback corresponds to a trace. Partiality ensures that
only authorized feedback loops exist.
\end{proof}

\begin{corollary}[Collapse in Standard Transformers]
Standard transformer architectures correspond to applying collapse immediately
after each attention fold.
\end{corollary}

\begin{proof}
Transformer implementations retain only the resulting value vectors and discard
all intermediate relevance structure. This is precisely eager collapse with
respect to payload invariants.
\end{proof}

% --------------------------------------------------
\subsection{Summary}
% --------------------------------------------------

The categorical semantics developed here unify event--historical aggregation,
map--reduce, and attention mechanisms within a single partial monoidal framework.
Merge corresponds to tensoring, refusal to the absence of morphisms, and collapse
to quotienting by invariant--preserving equivalence. Attention emerges as a
traced fold that is ordinarily collapsed in practice, obscuring its underlying
commitment structure.

% ==================================================
\section{A BNF Grammar for the Spherepop Calculus}
% ==================================================

This appendix presents a formal grammar for Spherepop, expressed in
Backus--Naur Form (BNF). The grammar defines the syntactic structure of
event--historical programs independently of any particular execution engine.
Its purpose is to make explicit which constructs are primitive and how histories
are formed, extended, and constrained.

The grammar is intentionally minimal. All semantic force arises from event
authorization, refusal, and replay, rather than from complex syntactic forms.

% --------------------------------------------------
\subsection{Lexical Domains}
% --------------------------------------------------

We assume the following disjoint lexical domains:

\[
\begin{aligned}
\langle\textit{Identifier}\rangle &\quad \text{object names, shard IDs, variables} \\
\langle\textit{EventName}\rangle   &\quad \textsc{pop}, \textsc{link}, \textsc{merge}, \textsc{collapse}, \textsc{refuse} \\
\langle\textit{Invariant}\rangle   &\quad \text{named invariants (e.g.\ payload, provenance)} \\
\langle\textit{Literal}\rangle     &\quad \text{numbers, symbols, structured values} \\
\end{aligned}
\]

% --------------------------------------------------
\subsection{Programs and Histories}
% --------------------------------------------------

A Spherepop program is a history specification: a sequence of event statements.

\[
\begin{aligned}
\langle\textit{Program}\rangle ::= 
  &\ \langle\textit{History}\rangle
\end{aligned}
\]

\[
\begin{aligned}
\langle\textit{History}\rangle ::= 
  &\ \langle\textit{Event}\rangle \\
  &\ |\ \langle\textit{History}\rangle\ \langle\textit{Event}\rangle
\end{aligned}
\]

Histories are ordered and append--only. There is no syntactic construct for
deletion or mutation of prior events.

% --------------------------------------------------
\subsection{Events}
% --------------------------------------------------

\[
\begin{aligned}
\langle\textit{Event}\rangle ::= 
  &\ \langle\textit{PopEvent}\rangle \\
  &\ |\ \langle\textit{LinkEvent}\rangle \\
  &\ |\ \langle\textit{MergeEvent}\rangle \\
  &\ |\ \langle\textit{CollapseEvent}\rangle \\
  &\ |\ \langle\textit{RefuseEvent}\rangle
\end{aligned}
\]

Each event form corresponds to a semantic primitive defined in the main text.

% --------------------------------------------------
\subsection{Commitment Events}
% --------------------------------------------------

\[
\begin{aligned}
\langle\textit{PopEvent}\rangle ::= 
  &\ \textsc{pop}\ \langle\textit{Identifier}\rangle
\end{aligned}
\]

A \textsc{pop} event introduces an object as a committed fact. The grammar does
not enforce authorization; admissibility is a semantic property.

% --------------------------------------------------
\subsection{Provenance and Dependency}
% --------------------------------------------------

\[
\begin{aligned}
\langle\textit{LinkEvent}\rangle ::= 
  &\ \textsc{link}\ \langle\textit{Identifier}\rangle\ \texttt{->}\ \langle\textit{Identifier}\rangle \\
  &\ |\ \textsc{link}\ \langle\textit{Identifier}\rangle\ \texttt{->}\ \langle\textit{Identifier}\rangle\ \texttt{:}\ \langle\textit{Identifier}\rangle
\end{aligned}
\]

Links encode provenance, causal dependency, typing, or policy constraints.
The optional label refines the semantic role of the link but does not affect
syntax.

% --------------------------------------------------
\subsection{Merge Events}
% --------------------------------------------------

\[
\begin{aligned}
\langle\textit{MergeEvent}\rangle ::= 
  &\ \textsc{merge}\ \langle\textit{Identifier}\rangle\ \langle\textit{Identifier}\rangle\ \texttt{->}\ \langle\textit{Identifier}\rangle
\end{aligned}
\]

A merge event binds two existing objects into a new object. The target identifier
must be fresh. Authorization depends on provenance and policy rules external to
the grammar.

% --------------------------------------------------
\subsection{Controlled Forgetting}
% --------------------------------------------------

\[
\begin{aligned}
\langle\textit{CollapseEvent}\rangle ::= 
  &\ \textsc{collapse}\ \langle\textit{Identifier}\rangle \\
  &\ |\ \textsc{collapse}\ \langle\textit{Identifier}\rangle\ \texttt{keep}\ \langle\textit{InvariantList}\rangle
\end{aligned}
\]

\[
\begin{aligned}
\langle\textit{InvariantList}\rangle ::= 
  &\ \langle\textit{Invariant}\rangle \\
  &\ |\ \langle\textit{Invariant}\rangle\ \texttt{,}\ \langle\textit{InvariantList}\rangle
\end{aligned}
\]

Collapse events specify which invariants must be preserved. All other distinctions
are discarded irreversibly.

% --------------------------------------------------
\subsection{Refusal}
% --------------------------------------------------

\[
\begin{aligned}
\langle\textit{RefuseEvent}\rangle ::= 
  &\ \textsc{refuse}\ \langle\textit{Event}\rangle \\
  &\ |\ \textsc{refuse}\ \langle\textit{Event}\rangle\ \texttt{:}\ \langle\textit{Identifier}\rangle
\end{aligned}
\]

A refusal documents an inadmissible event. Semantically, refusal prevents the
existence of any successor history in which the refused event occurs.

% --------------------------------------------------
\subsection{Values and Literals (Optional Layer)}
% --------------------------------------------------

The core calculus is value--agnostic. When needed, literals may be introduced
through external binding mechanisms:

\[
\begin{aligned}
\langle\textit{LiteralBinding}\rangle ::= 
  &\ \langle\textit{Identifier}\rangle\ \texttt{:=}\ \langle\textit{Literal}\rangle
\end{aligned}
\]

Such bindings are interpreted as shorthand for constructing objects whose
payloads are given by the literal values. They do not alter the event semantics.

% --------------------------------------------------
\subsection{Well--Formedness Conditions (Informal)}
% --------------------------------------------------

The grammar permits syntactically valid but semantically inadmissible programs.
The following conditions are enforced semantically rather than grammatically:

\begin{itemize}
\item An identifier must be introduced by \textsc{pop} before it can be merged.
\item Merge targets must be fresh identifiers.
\item Authorization constraints determine whether merges and collapses are
      admissible.
\item Refusal prevents the existence of extended histories.
\end{itemize}

These constraints are intentionally excluded from the grammar to preserve a
clean separation between syntax and event--historical semantics.

% --------------------------------------------------
\subsection{Discussion}
% --------------------------------------------------

This grammar emphasizes that Spherepop is not a term--rewriting language but a
history--construction calculus. Programs specify not how values are transformed,
but which events are attempted and which commitments are made or refused.
Execution is replay, and correctness is admissibility.

The simplicity of the grammar reflects a design principle: expressive power
should reside in event semantics and authorization, not in syntactic complexity.

% ==================================================
\section{Small--Step Operational Semantics}
% ==================================================

This appendix derives a small--step operational semantics for Spherepop directly
from the core event grammar. The semantics is intentionally minimal: it defines
how histories are extended one event at a time, how refusal corresponds to
stuckness (absence of a successor configuration), and how replay materializes
objects as persistent artifacts.

\subsection{Configurations}

We model execution as a transition system over configurations. A configuration
is a triple
\[
\langle H,\ \Gamma,\ \Sigma \rangle,
\]
where $H$ is the committed event history, $\Gamma$ is an object environment, and
$\Sigma$ is a semantic store. The environment $\Gamma$ maps identifiers to object
references; the store $\Sigma$ maps object references to semantic records. A
semantic record minimally contains a payload component and a provenance
component, but the semantics is parameterized by the payload algebra.

\begin{definition}[Semantic Record]
A semantic record is a pair $\Sigma(o) = (v(o), P(o))$ where $v(o)$ is the payload
and $P(o) \subseteq \mathcal{S}$ is a finite provenance set.
\end{definition}

The semantics assumes a partial payload merge operator $\diamond$ on payloads and
a partial authorization predicate $\mathsf{Auth}$ that may depend on $H$, $\Gamma$
and $\Sigma$.

\subsection{Judgments}

We write a small step as
\[
\langle H,\Gamma,\Sigma\rangle \xrightarrow{e} \langle H',\Gamma',\Sigma'\rangle,
\]
meaning that event $e$ is admissible and produces a successor configuration. If
no successor exists, the configuration is stuck; refusal is represented by
non--existence of a step.

We will also use a helper judgment for freshness: $\mathsf{fresh}(x,\Gamma)$
holds when identifier $x$ is not bound in $\Gamma$.

\subsection{Rules for \textsc{pop}}

A \textsc{pop} introduces a new committed object name. Operationally, it allocates
a fresh object reference $o$ and binds it to the identifier. The payload and
provenance of a \textsc{pop} may be supplied externally (e.g.\ via literal binding)
or may default to a unit value.

\begin{proposition}[Pop Rule]
If $\mathsf{fresh}(x,\Gamma)$ and $\mathsf{Auth}(H,\Gamma,\Sigma,\textsc{pop}\ x)$
then there exists a fresh object reference $o$ such that
\[
\langle H,\Gamma,\Sigma\rangle \xrightarrow{\textsc{pop}\ x}
\langle H\cdot \textsc{pop}(x),\ \Gamma[x\mapsto o],\ \Sigma[o\mapsto (v_0,P_0)]\rangle.
\]
\end{proposition}

\begin{remark}
The choice of $(v_0,P_0)$ is parameterized. For mapper outputs, $(v_0,P_0)$ will
already encode the shard provenance; for empty reducers, $P_0=\varnothing$.
\end{remark}

\subsection{Rules for \textsc{link}}

Links do not change payloads; they register dependency structure. The semantics
may record links as part of the store, or treat them as history--only. We present
a conservative semantics that records them in $\Sigma$ as adjacency metadata
$\mathsf{Links}(o)$.

\begin{proposition}[Link Rule]
If $\Gamma(x)=o_x$, $\Gamma(y)=o_y$, and
$\mathsf{Auth}(H,\Gamma,\Sigma,\textsc{link}\ x\to y)$ then
\[
\langle H,\Gamma,\Sigma\rangle \xrightarrow{\textsc{link}\ x\to y}
\langle H\cdot \textsc{link}(x\to y),\ \Gamma,\ \Sigma'\rangle
\]
where $\Sigma'$ is $\Sigma$ updated so that $o_x$ records a link to $o_y$.
\end{proposition}

\subsection{Rules for \textsc{merge}}

Merge is the only primitive that necessarily changes payload and provenance. It
constructs a fresh target object whose semantic record is computed from the
sources. Authorization includes (at minimum) provenance disjointness and may
include policy.

Let $x,y,z$ be identifiers. Define $o_x=\Gamma(x)$, $o_y=\Gamma(y)$, and assume
$z$ is fresh. Let $\Sigma(o_x)=(v_x,P_x)$ and $\Sigma(o_y)=(v_y,P_y)$.

\begin{theorem}[Merge Rule]
If $\mathsf{fresh}(z,\Gamma)$, $P_x\cap P_y=\varnothing$, the payload merge
$v_x\diamond v_y$ is defined, and
$\mathsf{Auth}(H,\Gamma,\Sigma,\textsc{merge}\ x\ y\to z)$, then there exists a
fresh object reference $o_z$ such that
\[
\langle H,\Gamma,\Sigma\rangle \xrightarrow{\textsc{merge}\ x\ y\to z}
\langle H\cdot \textsc{merge}(x,y\to z),\ \Gamma[z\mapsto o_z],\ \Sigma'\rangle
\]
where $\Sigma'(o_z)=(v_x\diamond v_y,\ P_x\cup P_y)$ and $\Sigma'$ also records
links from $o_z$ to $o_x$ and $o_y$.
\end{theorem}

\begin{proof}
By the stated conditions, the semantic record for the merged object is defined.
Freshness ensures a new identity. Recording ancestry links is admissible and
preserves lineage.
\end{proof}

\subsection{Rules for \textsc{collapse}}

Collapse replaces an object by a collapsed representative that preserves a chosen
set of invariants. Operationally, collapse may either produce a new object name
or rewrite the semantic record of an existing object reference. To preserve the
append--only character, we adopt a persistent semantics: collapse allocates a new
object reference $o'$ and rebinds the identifier to it.

Let $\mathsf{Keep}_I$ denote an invariant projection operator, mapping semantic
records to a canonical representative preserving invariants $I$.

\begin{theorem}[Collapse Rule]
If $\Gamma(x)=o$, $\mathsf{Auth}(H,\Gamma,\Sigma,\textsc{collapse}\ x\ \texttt{keep}\ I)$,
and $\mathsf{Keep}_I(\Sigma(o))$ is defined, then there exists a fresh object
reference $o'$ such that
\[
\langle H,\Gamma,\Sigma\rangle \xrightarrow{\textsc{collapse}\ x\ \texttt{keep}\ I}
\langle H\cdot \textsc{collapse}(x;I),\ \Gamma[x\mapsto o'],\ \Sigma[o'\mapsto \mathsf{Keep}_I(\Sigma(o))]\rangle.
\]
\end{theorem}

\subsection{Rules for \textsc{refuse}}

A \textsc{refuse} statement is documentation of inadmissibility. Operationally,
it does not advance the computation, because it asserts that a particular event
has no admissible successor. We therefore treat it as a judgmental annotation,
not a transition.

\begin{definition}[Refusal and Stuckness]
A configuration is stuck on event $e$ if there is no configuration $C'$ such that
$C\xrightarrow{e} C'$. A \textsc{refuse} statement is well formed exactly when
its embedded event would be stuck at the current configuration.
\end{definition}

\begin{remark}
This matches the essay's semantics: refusal is absence of a continuation, not an
error value.
\end{remark}

% ==================================================
\section{A Static Type and Authorization System}
% ==================================================

This appendix gives a static discipline that approximates the dynamic
authorization predicate $\mathsf{Auth}$ and prevents large classes of inadmissible
histories from being written. The intent is not to decide authorization
completely, but to separate (i) syntactic well--formedness, (ii) static
admissibility, and (iii) dynamic policy checks.

\subsection{Typing Environments and Types}

We define a typing environment $\Delta$ mapping identifiers to types. Types track
at least two aspects: the payload kind and the provenance footprint.

Let $\kappa$ range over payload kinds (e.g.\ histogram, sum, set, embedding). Let
$\pi$ range over provenance descriptors. For static checking, provenance may be
approximated as either a concrete finite set (when known) or an abstract region
label.

\begin{definition}[Types]
A Spherepop object type is a pair $\tau = \kappa[\pi]$, where $\kappa$ is a payload
kind and $\pi$ is a provenance descriptor.
\end{definition}

We write $\mathsf{disjoint}(\pi_1,\pi_2)$ for a decidable predicate that
conservatively approximates disjointness of provenance. If it returns true,
dynamic provenance disjointness is guaranteed; if false, disjointness is unknown
and must be decided dynamically or refused conservatively.

\subsection{Typing Judgments}

Typing judgments have the form
\[
\Delta \vdash e : \Delta',
\]
meaning that under environment $\Delta$, event $e$ is statically admissible and
produces an updated environment $\Delta'$.

\subsection{Rules}

\begin{theorem}[Pop Typing]
If $x\notin \mathrm{dom}(\Delta)$ and $\tau$ is an assigned type for $x$, then
\[
\Delta \vdash \textsc{pop}\ x : \Delta[x\mapsto \tau].
\]
\end{theorem}

\begin{remark}
The grammar does not assign $\tau$; a front--end may infer it from literal
bindings or annotations, or leave it abstract.
\end{remark}

\begin{theorem}[Link Typing]
If $\Delta(x)=\tau_x$ and $\Delta(y)=\tau_y$, then
\[
\Delta \vdash \textsc{link}\ x\to y : \Delta.
\]
\end{theorem}

\begin{theorem}[Merge Typing]
If $\Delta(x)=\kappa[\pi_x]$, $\Delta(y)=\kappa[\pi_y]$, $z\notin \mathrm{dom}(\Delta)$,
and $\mathsf{disjoint}(\pi_x,\pi_y)$ holds, then
\[
\Delta \vdash \textsc{merge}\ x\ y\to z : \Delta[z\mapsto \kappa[\pi_x\cup \pi_y]].
\]
\end{theorem}

\begin{proof}
Static disjointness ensures the dynamic disjointness precondition. Payload kinds
must match to ensure that $\diamond$ is defined at the kind level. Provenance
union is tracked in the resulting type.
\end{proof}

\begin{theorem}[Collapse Typing]
If $\Delta(x)=\kappa[\pi]$ and $I$ is an invariant set admissible for $\kappa$,
then
\[
\Delta \vdash \textsc{collapse}\ x\ \texttt{keep}\ I : \Delta.
\]
\end{theorem}

\begin{remark}
Collapse does not change the kind, but it may coarsen provenance. A refined system
may update $\pi$ to a less precise descriptor to reflect forgetting.
\end{remark}

\subsection{Soundness Relative to Dynamic Authorization}

Let $\mathsf{Auth}$ include at least (i) kind compatibility, (ii) provenance
disjointness, and (iii) any additional policy constraints.

\begin{theorem}[Static Soundness]
If $\Delta \vdash e : \Delta'$ and the runtime configuration realizes $\Delta$,
then either $e$ is dynamically authorized or it is rejected only due to policy
constraints not expressible in $\Delta$.
\end{theorem}

\begin{proof}
The typing rules enforce the structural preconditions of authorization: freshness,
kind compatibility, and (conservative) provenance disjointness. Any remaining
rejections must therefore arise from constraints intentionally left to dynamic
policy.
\end{proof}

% ==================================================
\section{A Reference Interpreter Consistent with the Grammar}
% ==================================================

This appendix specifies a minimal reference interpreter that executes Spherepop
programs by replaying their event histories under the small--step semantics. The
interpreter is described as an algorithm rather than an implementation, so that
multiple concrete runtimes can be validated against the same reference behavior.

\subsection{Interpreter State}

The interpreter maintains a state $(H,\Gamma,\Sigma)$ as in Appendix~E, together
with an allocation counter for fresh object references.

The interpreter takes as input a parsed program, i.e.\ a sequence of event
statements $(e_1,\ldots,e_n)$.

\subsection{Execution Algorithm}

Execution proceeds sequentially. For each event $e_i$, the interpreter checks
static admissibility (optional), then checks dynamic authorization, then applies
the corresponding transition rule. If no transition rule applies, the interpreter
halts and reports refusal (inadmissibility) with an explanation when available.

\begin{verbatim}
Algorithm Replay(program):
    initialize H := empty
    initialize Γ := empty
    initialize Σ := empty
    for each event e in program:
        if optional_static_check_enabled:
            require Δ ⊢ e : Δ'  (else reject)
        if no small-step rule applies to ⟨H,Γ,Σ⟩ with label e:
            report REFUSAL(e, H, Γ, Σ)
            halt
        else:
            apply the unique rule to obtain ⟨H',Γ',Σ'⟩
            set (H,Γ,Σ) := (H',Γ',Σ')
    return ⟨H,Γ,Σ⟩
\end{verbatim}

\subsection{Determinism}

Determinism holds when allocation is deterministic (e.g.\ fresh references chosen
by a counter) and when payload merge $\diamond$ is deterministic.

\begin{theorem}[Determinism of Replay]
Assume deterministic allocation and deterministic payload operations. Then for any
program, replay produces either a unique final configuration or a unique first
refusal event.
\end{theorem}

\begin{proof}
By construction, each event statement has at most one applicable transition rule
given the current configuration. If no rule applies, refusal is determined at
that point. Otherwise the successor is uniquely determined.
\end{proof}

\subsection{Conformance Criteria}

A concrete implementation conforms to this reference interpreter if, for every
program, it agrees on (i) the first point of refusal, when refusal occurs, and
(ii) the final replayed semantic records for all committed identifiers, modulo
the chosen collapse invariants.

\begin{definition}[Interpreter Conformance]
An implementation conforms if for all programs $P$, either both reference and
implementation refuse at the same event index, or both terminate with equivalent
stores under invariant--preserving equivalence.
\end{definition}

% ==================================================
\begin{thebibliography}{99}
% ==================================================

\bibitem{Lamport1978}
L.~Lamport.
\newblock Time, clocks, and the ordering of events in a distributed system.
\newblock \emph{Communications of the ACM}, 21(7):558--565, 1978.

\bibitem{GrayReuter1993}
J.~Gray and A.~Reuter.
\newblock \emph{Transaction Processing: Concepts and Techniques}.
\newblock Morgan Kaufmann, 1993.

\bibitem{FowlerEventSourcing}
M.~Fowler.
\newblock Event sourcing.
\newblock \emph{martinfowler.com}, 2005.
\newblock Online essay.

\bibitem{Shapiro2011}
M.~Shapiro, N.~Preguiça, C.~Baquero, and M.~Zawirski.
\newblock Conflict-free replicated data types.
\newblock In \emph{Proceedings of the 13th International Symposium on Stabilization, Safety, and Security of Distributed Systems (SSS)}, 2011.

\bibitem{Baquero2017}
C.~Baquero, P.~Sérgio Almeida, and A.~Shoker.
\newblock Making operation-based CRDTs operation-based.
\newblock In \emph{Proceedings of the 21st International Conference on Principles of Distributed Systems (OPODIS)}, 2017.

\bibitem{Lynch1996}
N.~A. Lynch.
\newblock \emph{Distributed Algorithms}.
\newblock Morgan Kaufmann, 1996.

\bibitem{Girard1987}
J.-Y. Girard.
\newblock Linear logic.
\newblock \emph{Theoretical Computer Science}, 50(1):1--102, 1987.

\bibitem{AbramskyCoecke2004}
S.~Abramsky and B.~Coecke.
\newblock A categorical semantics of quantum protocols.
\newblock In \emph{Proceedings of the 19th IEEE Symposium on Logic in Computer Science (LICS)}, 2004.

\bibitem{JoyalStreetVerity1996}
A.~Joyal, R.~Street, and D.~Verity.
\newblock Traced monoidal categories.
\newblock \emph{Mathematical Proceedings of the Cambridge Philosophical Society}, 119(3):447--468, 1996.

\bibitem{Selinger2011}
P.~Selinger.
\newblock A survey of graphical languages for monoidal categories.
\newblock In \emph{New Structures for Physics}, Springer, 2011.

\bibitem{Bennett1982}
C.~H. Bennett.
\newblock The thermodynamics of computation—A review.
\newblock \emph{International Journal of Theoretical Physics}, 21(12):905--940, 1982.

\bibitem{Landauer1961}
R.~Landauer.
\newblock Irreversibility and heat generation in the computing process.
\newblock \emph{IBM Journal of Research and Development}, 5(3):183--191, 1961.

\bibitem{MacLane1998}
S.~Mac~Lane.
\newblock \emph{Categories for the Working Mathematician}.
\newblock Springer, 2nd edition, 1998.

\bibitem{Pearl2009}
J.~Pearl.
\newblock \emph{Causality: Models, Reasoning, and Inference}.
\newblock Cambridge University Press, 2nd edition, 2009.

\bibitem{Vaswani2017}
A.~Vaswani et~al.
\newblock Attention is all you need.
\newblock In \emph{Advances in Neural Information Processing Systems (NeurIPS)}, 2017.

\bibitem{Battaglia2018}
P.~Battaglia et~al.
\newblock Relational inductive biases, deep learning, and graph networks.
\newblock \emph{arXiv:1806.01261}, 2018.

\bibitem{Bronstein2021}
M.~M. Bronstein et~al.
\newblock Geometric deep learning: Grids, groups, graphs, geodesics, and gauges.
\newblock \emph{arXiv:2104.13478}, 2021.

\bibitem{Barandes2023}
J.~A. Barandes.
\newblock Unistochastic quantum theory.
\newblock \emph{Physical Review A}, 108(3), 2023.

\bibitem{Friston2010}
K.~Friston.
\newblock The free-energy principle: A unified brain theory?
\newblock \emph{Nature Reviews Neuroscience}, 11:127--138, 2010.

\bibitem{DodigCrnkovic2014}
G.~Dodig-Crnkovic.
\newblock Information, computation, cognition.
\newblock \emph{Entropy}, 16(1):245--256, 2014.

\bibitem{Winskel1987}
G.~Winskel.
\newblock Event structures.
\newblock In \emph{Advances in Petri Nets}, Springer, 1987.

\bibitem{Milner1989}
R.~Milner.
\newblock \emph{Communication and Concurrency}.
\newblock Prentice Hall, 1989.

\bibitem{Hewitt1973}
C.~Hewitt.
\newblock A universal modular actor formalism for artificial intelligence.
\newblock In \emph{Proceedings of IJCAI}, 1973.

\end{thebibliography}
% ==================================================

\end{document} 
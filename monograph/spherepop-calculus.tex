\documentclass{article}
\usepackage{amsmath}
\usepackage{amssymb}
\usepackage{amsthm}
\usepackage{mathpartir}
\usepackage{booktabs}
\usepackage{hyperref}
\usepackage{tabularx,adjustbox,array}
\newcolumntype{Y}{>{\raggedright\arraybackslash}X} % flexible ragged-right
\newcolumntype{C}{>{\centering\arraybackslash}p{2.2cm}} % fixed-width centered col
\usepackage{fullpage}

\title{Spherepop Calculus}
\author{Flyxion}

\date{\today}

\newtheorem{theorem}{Theorem}[section]
\newtheorem{lemma}[theorem]{Lemma}
\newtheorem{corollary}[theorem]{Corollary}
\newtheorem{proposition}[theorem]{Proposition}
\newtheorem{observation}[theorem]{Observation}

\begin{document}

\maketitle

\begin{abstract}
The \emph{Spherepop Calculus} (SPC) is a novel functional language and
type-theoretic framework that unifies abstraction, concurrency, and probabilistic
choice within a single geometric model of computation. SPC extends the
$\lambda$-calculus by reinterpreting abstraction and application as
\texttt{Sphere} and \texttt{Pop}, visualizing scope as nested spheres rather
than syntactic parentheses. It introduces \texttt{Merge} as a categorical
parallel operator with tensorial semantics, and \texttt{Choice} as a primitive
for probabilistic branching, expressible either internally or via a distribution
monad. The type system extends the Calculus of Constructions with dependent
types, while the denotational semantics is given in a presheaf topos enriched
with the Giry distribution monad. We establish meta-theoretic properties of SPC,
including preservation, progress, and adequacy of the probabilistic semantics,
together with a formal \emph{Independent Channels Lemma} that quantifies
aggregated risk across merged probabilistic branches. Historical analysis
positions SPC at the intersection of $\lambda$-calculus, $\pi$-calculus,
probabilistic $\lambda$-calculi, and dependent type theory, while translations
demonstrate that SPC strictly subsumes these calculi up to natural fragments.
The result is a language that serves both as a foundation for probabilistic and
concurrent computation and as a categorical framework for reasoning about
structured scope, uncertainty, and interaction.
\end{abstract}

\tableofcontents

\section{Introduction}

The study of computation has been shaped by a small number of canonical calculi,
each capturing different aspects of programming and reasoning. The
$\lambda$-calculus, introduced by Church in the 1930s, provides a minimal yet
universal account of higher-order functional abstraction. Process calculi such as
Milner’s $\pi$-calculus emphasize concurrency and interaction, while extensions
of $\lambda$-calculus with probabilistic primitives provide a foundation for
reasoning under uncertainty. Dependent type theories, beginning with Martin-Löf,
further embed computation into a constructive foundation for mathematics,
realized in proof assistants such as Coq.

The \emph{Spherepop Calculus} (SPC) proposes a synthesis of these traditions,
introducing a geometric reinterpretation of abstraction and application as
\texttt{Sphere} and \texttt{Pop}. Rather than treating scope as a merely
syntactic device (parentheses), SPC represents computation as the structured
opening and closing of \emph{spheres}, which can be nested, merged, or
collapsed. This geometric intuition makes scope explicit, supports natural
visualization, and provides a foundation for extending the calculus with
parallelism and probability.

SPC departs from the classical $\lambda$-calculus in three key ways:
\begin{enumerate}
  \item \textbf{Merge.} A primitive operator for nondeterministic or parallel
  composition, interpreted categorically as a tensor product in a monoidal
  category.
  \item \textbf{Choice.} A probabilistic branching operator, admitting two
  semantics: an internal version, where choice returns a value of type $A$, and a
  distribution-monad version, where choice yields an element of
  $\mathsf{Dist}(A)$.
  \item \textbf{Dependent extension.} SPC generalizes the Calculus of
  Constructions by allowing $\Pi$- and $\Sigma$-types in the presence of Merge
  and Choice, ensuring compositionality with categorical semantics in a presheaf
  topos.
\end{enumerate}

The contributions of this paper are fourfold:
\begin{itemize}
  \item We define the syntax, typing rules, and operational semantics of SPC,
  highlighting its core primitives (\texttt{Sphere}, \texttt{Pop},
  \texttt{Merge}, \texttt{Choice}).
  \item We give a denotational semantics in the presheaf topos
  $[\mathsf{Sphere}^{op},\mathsf{Set}]$, enriched with the Giry distribution
  monad, and prove adequacy with respect to the operational semantics.
  \item We establish meta-theoretic properties, including preservation,
  progress, and a formal \emph{Independent Channels Lemma} that quantifies
  disjunctive risk aggregation across merged probabilistic branches.
  \item We situate SPC within the historical development of computation, compare
  it with $\lambda$-calculus, $\pi$-calculus, and probabilistic $\lambda$-calculi,
  and provide translations showing that SPC strictly subsumes these calculi up
  to natural fragments.
\end{itemize}

Taken together, SPC offers a unified framework for functional abstraction,
concurrency, and probabilistic reasoning, supported by dependent types and
categorical semantics. It is simultaneously a programming language, a logical
foundation, and a conceptual model of computation where scope is geometric,
parallelism is tensorial, and probability is compositional.

\subsection{Motivation and Overview}

We establish the foundational properties of the Spherepop Calculus (SPC) through
three main technical contributions.

\paragraph{Type Safety.}
We prove \emph{Preservation} (well-typed terms remain well-typed under reduction)
and \emph{Progress} (a closed well-typed term is either a value or can take a
reduction step). These results extend the standard type-safety arguments for
$\lambda$-calculus to a setting enriched with dependent types, Merge, and
Choice.

\paragraph{Adequacy of Probabilistic Semantics.}
We show that the denotational semantics in the presheaf topos, enriched with the
distribution monad, is adequate with respect to the operational semantics.
Specifically, the expectation of any observable function under the operational
distribution coincides with its expectation under the denotational distribution.
This adequacy ensures that probabilistic reasoning in SPC is compositional.

\paragraph{Expressivity.}
We define translations from the simply-typed $\lambda$-calculus, probabilistic
$\lambda$-calculus, and a nondeterministic parallel fragment of $\pi$-calculus
into SPC. These translations preserve typing and operational behavior, and
demonstrate that SPC strictly subsumes each source calculus. As a corollary, SPC
provides a unifying framework for higher-order functions, probabilistic choice,
and parallel composition.

\paragraph{Independent Channels Lemma.}
Finally, we formalize the aggregation of independent probabilistic branches
under Merge. We prove that the probability of disjunctive failure across $n$
independent channels is given by $1 - \prod_{i=1}^n(1-p_i)$, aligning the
operational and denotational accounts. This lemma illustrates how SPC captures
structured reasoning about risk in concurrent probabilistic systems.

The contributions of this paper are as follows:

\begin{itemize}
  \item \textbf{Geometric scope model.}
  We introduce \emph{Sphere} and \emph{Pop} as geometric abstractions of
  $\lambda$-abstraction and application, making scope explicit through
  spherical nesting.

  \item \textbf{Parallel and probabilistic primitives.}
  We extend the calculus with \emph{Merge} (categorical tensor for parallel
  composition) and \emph{Choice} (probabilistic branching), with both internal
  and distribution-monad interpretations.

  \item \textbf{Dependent type system.}
  We integrate $\Pi$- and $\Sigma$-types into SPC, yielding a
  \emph{probabilistic, monoidal calculus of constructions}.

  \item \textbf{Denotational semantics.}
  We give a semantics in the presheaf topos $[\mathsf{Sphere}^{op},\mathsf{Set}]$
  enriched with the Giry distribution monad, and prove \emph{adequacy} with
  respect to operational semantics.

  \item \textbf{Meta-theory.}
  We establish preservation, progress, and a formal \emph{Independent Channels Lemma}
  quantifying disjunctive risk aggregation under Merge.

  \item \textbf{Expressivity.}
  We provide translations from $\lambda$-calculus, probabilistic $\lambda$-calculus,
  and a parallel fragment of $\pi$-calculus into SPC, proving SPC strictly subsumes
  these calculi.
\end{itemize}

\subsection{Structure of the Paper}

The remainder of the paper is organized as follows. 
Section~\ref{sec:syntax} introduces the syntax, types, and typing rules of SPC. 
Section~\ref{sec:operational} defines the operational semantics, including 
$\beta$-reduction, Merge, and Choice. 
Section~\ref{sec:denotational} presents the denotational semantics in a presheaf 
topos enriched with the distribution monad. 
Section~\ref{sec:meta} establishes the meta-theoretic results: preservation, 
progress, adequacy, and the Independent Channels Lemma. 
Section~\ref{sec:history} situates SPC in historical context, tracing antecedents 
from $\lambda$-calculus to probabilistic programming. 
Section~\ref{sec:positioning} compares SPC to related calculi and formalizes its 
expressivity. 
Section~\ref{sec:translations} gives constructive translations from $\lambda$-calculus, 
probabilistic $\lambda$-calculus, and a fragment of $\pi$-calculus into SPC. 
Finally, the appendices provide auxiliary lemmas, substitution proofs, and worked 
examples illustrating the calculus in action.

\section{Related Work}

The Spherepop Calculus (SPC) builds on several long-standing traditions in logic,
type theory, and programming language semantics, while introducing new
geometric primitives for abstraction, concurrency, and probabilistic choice.

\paragraph{Foundations in $\lambda$-calculus and type theory.}
The $\lambda$-calculus \cite{church1940formulation,barendregt1984lambda} is the
canonical foundation of functional abstraction, and type-theoretic refinements
such as Martin-Löf’s intuitionistic type theory \cite{martinlof1975intuitionistic}
and the Calculus of Constructions \cite{coquand1988calculus} enable expressive
proof systems with dependent types. SPC inherits this tradition but reinterprets
scope geometrically through \texttt{Sphere} and \texttt{Pop}, extending it with
native parallel and probabilistic constructs. The broad development of type
systems is summarized by Pierce \cite{pierce2002types}, which SPC builds upon by
fusing typing with concurrency and uncertainty.

\paragraph{Categorical semantics.}
The Curry–Howard–Lambek correspondence \cite{lambek1986categorical}, Mac Lane’s
coherence results \cite{maclane1963natural}, and Lawvere’s elementary topos
theory \cite{lawvere1970quantifiers} frame the categorical interpretation of
logical systems. Awodey \cite{awodey2010category} provides a modern reference
that situates these ideas within contemporary category theory. SPC draws on this
lineage by giving its denotational semantics in a presheaf topos enriched with a
distribution monad, with \texttt{Merge} modeled as a tensor. Street’s formal
theory of monads \cite{street1972two} and Moggi’s computational monads
\cite{moggi1991notions} provide the conceptual background for SPC’s treatment of
probabilistic and parallel computation as compositional effects.

\paragraph{Linear and resource-sensitive calculi.}
Girard’s linear logic \cite{girard1987linear} introduced explicit control of
resources in proof theory and computation. While SPC is not linear, its
geometric nesting of scopes and its tensorial \texttt{Merge} operator echo
concerns of resource management and structural invariants familiar from the
linear tradition.

\paragraph{Probabilistic semantics and programming.}
The semantics of probabilistic programs has a long history
\cite{kozen1981semantics,giry1982categorical}, recently extended to dependent
types and constructive foundations \cite{adams2020foundations}. Jacobs and
Mandemaker’s coalgebraic approach \cite{jacobs2015new} and Fritz’s synthetic
treatment of Markov kernels \cite{fritz2020synthetic} highlight how categorical
structures support probability theory. SPC contributes to this tradition by
internalizing probabilistic branching as a primitive operator (\texttt{Choice}),
interpretable both operationally and via monads. Contemporary probabilistic
programming languages such as Church, Anglican, and WebPPL
\cite{goodman2014design,van2018introduction,gordon2014probabilistic} show the
practical importance of such models; SPC offers a categorical and type-theoretic
foundation that could undergird future implementations.

\paragraph{Concurrency and process calculi.}
Milner’s CCS \cite{milner1980calculus} and $\pi$-calculus
\cite{milner1999communicating,milner1992calculus} provided the foundational
frameworks for concurrency and mobility of processes. SPC’s \texttt{Merge}
resembles nondeterministic parallel composition but is grounded in categorical
semantics rather than name-passing. This places SPC closer to categorical models
of concurrency, while retaining the operational intuition of interaction
between independent channels.

\paragraph{Implementations and proof assistants.}
The Calculus of Inductive Constructions is realized in Coq
\cite{bertot2004interactive}, while Agda
\cite{norell2009dependently} demonstrates dependently typed programming in a
more lightweight style. SPC shares the constructive spirit of these systems but
extends their expressivity with concurrency and probabilistic constructs at the
core. This suggests that SPC could serve as a foundation for future proof
assistants or domain-specific probabilistic programming tools with certified
semantics.

\paragraph{Summary.}
SPC thus situates itself at the intersection of four lines of research:
$\lambda$-calculus and dependent types, categorical logic and monads,
probabilistic semantics, and concurrency theory. Its novelty lies in
synthesizing these strands into a single, geometrically motivated calculus where
scope, parallelism, and probability are treated as first-class compositional
notions.


\section{Overview of the Language}

The \textbf{Spherepop Calculus (SPC)} is a higher-order, dependent-typed functional language designed to model computation through \textbf{spherical scoping primitives}. Its distinctive features are:

\subsubsection{1. Core Primitives}

\begin{itemize}
\item \textbf{Sphere} -- abstraction operator, analogous to $\lambda$, but visualized as opening a scope (``entering a bubble'').
\item \textbf{Pop} -- application operator, analogous to $\beta$-reduction, representing the act of collapsing a sphere with an argument.
\item \textbf{Merge} -- parallel or disjunctive composition, modeling nondeterministic or concurrent branching.
\item \textbf{Choice} -- probabilistic branching operator, either interpreted internally (as a weighted selector) or via the distribution monad.
\item \textbf{Nest} -- syntactic sugar for hierarchical application, enforcing structured scoping.
\end{itemize}

Together, these primitives generalize $\lambda$-calculus while encoding higher-order reasoning about scope, concurrency, and probability.

\subsubsection{2. Type Discipline}

SPC extends dependent type theory:

\begin{itemize}
\item \textbf{Dependent function types} ($\Pi$-types) capture parameterized spheres.
\item \textbf{Dependent sum types} ($\Sigma$-types) capture merged data.
\item \textbf{Merge} preserves type alignment, enforcing that both branches produce the same type.
\item \textbf{Choice} comes in two flavors:
  \begin{itemize}
  \item \textit{Internal}: returns an element of type $A$.
  \item \textit{Monadic}: returns a distribution over $A$, aligning with Giry's distribution monad.
  \end{itemize}
\end{itemize}

Thus, SPC can be seen as a \textbf{probabilistic Calculus of Constructions} variant.

\subsubsection{3. Operational Semantics}

\begin{itemize}
\item \textbf{$\beta$-reduction}: $\mathrm{Pop}(\mathrm{Sphere}(x:A.\,t),u) \to t[u/x]$.
\item \textbf{Merge}: associative and commutative; normalizes via structural equivalence.
\item \textbf{Choice}: reduces by stochastic sampling ($t$ w.p. $p$, $u$ w.p. $1-p$).
\end{itemize}

The calculus is confluent for the deterministic fragment and admits stochastic confluence properties for probabilistic terms.

\subsubsection{4. Denotational Semantics}

SPC is interpreted in a \textbf{presheaf topos} $[\mathsf{Sphere}^{op}, \mathsf{Set}]$:

\begin{itemize}
\item \textbf{Terms}: morphisms in the topos.
\item \textbf{Sphere/Pop}: exponential and evaluation morphisms.
\item \textbf{Merge}: monoidal product.
\item \textbf{Choice}: convex combination in the distribution monad, yielding true probabilistic semantics.
\end{itemize}

Adequacy theorems ensure that operational sampling aligns with denotational expectation.

\subsubsection{5. Meta-Theory}

SPC satisfies standard safety results:

\begin{itemize}
\item \textbf{Preservation} -- reduction preserves types.
\item \textbf{Progress} -- well-typed closed terms reduce or are values.
\item \textbf{Adequacy} -- denotational semantics agrees with operational probabilities.
\item \textbf{Independent Channels Lemma} -- formalizes disjunctive risk aggregation:
  \[
  \Pr[\text{doom in merged channels}] = 1 - \prod_i (1 - p_i).
  \]
\end{itemize}

\subsubsection{6. Conceptual Role}

Unlike traditional $\lambda$-calculus:

\begin{itemize}
\item SPC emphasizes \textbf{scoping as geometry} (spheres rather than parentheses).
\item Merge and Choice bring \textbf{parallelism and probability} into the core calculus.
\item The categorical semantics makes SPC not just a programming language, but a \textbf{theory of structured, probabilistic reasoning}, equally at home in logic, semantics, and physics.
\end{itemize}

\section{Syntax and Typing}\label{sec:syntax}

\subsection{Grammar of Terms}

\[
\begin{array}{rcll}
t,u & ::= & x & \text{variable} \\
     & \mid & a & \text{atom / constant} \\
     & \mid & \mathrm{Sphere}(x{:}A.\,t) & \text{abstraction} \\
     & \mid & \mathrm{Pop}(t,u) & \text{application} \\
     & \mid & \mathrm{Merge}(t,u) & \text{parallel / disjunction} \\
     & \mid & \mathrm{Nest}(t,u) & \text{syntactic sugar for Pop} \\
     & \mid & \mathrm{Choice}(p,t,u) & \text{probabilistic choice}
\end{array}
\]

\subsection{Types and Contexts}

Contexts: $\Gamma ::= \cdot \mid \Gamma, x{:}A$.

\subsection{Typing Rules}

\begin{mathpar}
\inferrule*[right=Var]
  {x{:}A \in \Gamma}
  {\Gamma \vdash x : A}

\inferrule*[right=Atom]
  { }
  {\Gamma \vdash a : A}

\inferrule*[$\Pi$-Form]
  {\Gamma \vdash A : \mathsf{Type}_i \\ \Gamma, x{:}A \vdash B : \mathsf{Type}_j}
  {\Gamma \vdash \Pi x{:}A.\,B : \mathsf{Type}_{\max(i,j)}}

\inferrule*[$\Pi$-Intro]
  {\Gamma, x{:}A \vdash t : B}
  {\Gamma \vdash \mathrm{Sphere}(x{:}A.\,t) : \Pi x{:}A.\,B}

\inferrule*[$\Pi$-Elim]
  {\Gamma \vdash f : \Pi x{:}A.\,B \\ \Gamma \vdash u : A}
  {\Gamma \vdash \mathrm{Pop}(f,u) : B[u/x]}

\inferrule*[$\Sigma$-Form]
  {\Gamma \vdash A : \mathsf{Type}_i \\ \Gamma, x{:}A \vdash B : \mathsf{Type}_j}
  {\Gamma \vdash \Sigma x{:}A.\,B : \mathsf{Type}_{\max(i,j)}}

\inferrule*[$\Sigma$-Intro]
  {\Gamma \vdash a : A \\ \Gamma \vdash b : B[a/x]}
  {\Gamma \vdash (a,b) : \Sigma x{:}A.\,B}

\inferrule*[right=Merge]
  {\Gamma \vdash t : A \\ \Gamma \vdash u : A}
  {\Gamma \vdash \mathrm{Merge}(t,u) : A}

\inferrule*[right=Choice]
  {\Gamma \vdash p : \mathsf{Prob} \\
   \Gamma \vdash t : A \\
   \Gamma \vdash u : A}
  {\Gamma \vdash \mathrm{Choice}(p,t,u) : A}
\end{mathpar}

\subsection{Dependent Types and Extensions}

The Spherepop Calculus extends the simply-typed core with full \emph{dependent
types}, generalizing the Calculus of Constructions. This enables the type of a
term to vary with the values of its arguments, and allows expressive encodings of
logical quantification, data structures, and program specifications.

\paragraph{Dependent Functions ($\Pi$-types).}
A dependent function type $\Pi x{:}A.\,B(x)$ represents a family of results $B(x)$
parameterized by an argument $x:A$. In SPC this is introduced by the abstraction
form $\mathsf{Sphere}(x{:}A.\,t)$ and eliminated by application
$\mathsf{Pop}(f,u)$. Typing rules are:

\[
\inferrule*[right=$\Pi$-Form]
  {\Gamma \vdash A : \mathsf{Type}_i \\
   \Gamma, x:A \vdash B(x) : \mathsf{Type}_j}
  {\Gamma \vdash \Pi x:A.\,B(x) : \mathsf{Type}_{\max(i,j)}}
\]

\[
\inferrule*[right=$\Pi$-Intro]
  {\Gamma, x:A \vdash t : B(x)}
  {\Gamma \vdash \mathsf{Sphere}(x{:}A.\,t) : \Pi x:A.\,B(x)}
\]

\[
\inferrule*[right=$\Pi$-Elim]
  {\Gamma \vdash f : \Pi x:A.\,B(x) \\
   \Gamma \vdash u : A}
  {\Gamma \vdash \mathsf{Pop}(f,u) : B(u)}
\]

\paragraph{Dependent Pairs ($\Sigma$-types).}
A dependent sum $\Sigma x{:}A.\,B(x)$ represents pairs $(a,b)$ where $a:A$ and
$b:B(a)$. SPC introduces such terms directly and allows pattern decomposition:

\[
\inferrule*[right=$\Sigma$-Form]
  {\Gamma \vdash A : \mathsf{Type}_i \\
   \Gamma, x:A \vdash B(x) : \mathsf{Type}_j}
  {\Gamma \vdash \Sigma x:A.\,B(x) : \mathsf{Type}_{\max(i,j)}}
\]

\[
\inferrule*[right=$\Sigma$-Intro]
  {\Gamma \vdash a : A \\
   \Gamma \vdash b : B(a)}
  {\Gamma \vdash (a,b) : \Sigma x:A.\,B(x)}
\]

Elimination is definable via dependent projection functions, aligning with the
standard rules of dependent type theory.

\paragraph{Interaction with Merge.}
The tensorial operator $\mathsf{Merge}$ requires both arguments to have the same
type. In the dependent setting, this condition is refined: if
$\Gamma \vdash M : A$ and $\Gamma \vdash N : A$, then
$\Gamma \vdash M \otimes N : A$. This ensures that parallel branches preserve
a uniform dependent type, while still allowing the type $A$ itself to depend on
contextual variables.

\paragraph{Interaction with Choice.}
The probabilistic operator $\mathsf{Choice}(p,M,N)$ is well-typed whenever
both branches agree on type:
\[
\Gamma \vdash M : A, \quad \Gamma \vdash N : A
\quad\Rightarrow\quad
\Gamma \vdash \mathsf{Choice}(p,M,N) : A.
\]
Alternatively, in the distributional extension, we admit:
\[
\Gamma \vdash \mathsf{Choice}(p,M,N) : \mathsf{Dist}(A).
\]
This makes explicit the transition from internal probabilistic branching to
external distribution-valued semantics.

\paragraph{Universes.}
As in the Calculus of Constructions, SPC assumes a cumulative hierarchy of type
universes $\mathsf{Type}_i$ indexed by natural numbers $i$. Dependent types can
be formed at any level, and type constructors preserve universe stratification.

\paragraph{Expressive Power.}
With $\Pi$- and $\Sigma$-types, SPC can encode first-order and higher-order
logic, inductive families, and dependent specifications of probabilistic
processes. In particular, the combination of dependent sums with Choice
permits \emph{probabilistic data structures} whose shape depends on sampled
random values, while dependent functions over Merge enable \emph{concurrent
families of proofs and programs}.

\medskip
In summary, the dependent extension equips SPC with the full expressivity of
modern type theories, while ensuring that Merge and Choice remain type-safe and
compositional within this enriched framework.

\begin{lemma}[Merge of Dependent Pairs yields a Joint Distribution]
\label{lem:merge-sigma-joint}
Let $A$ be a base type and let $\mathsf{Vec}:\mathbb{N}\to\mathsf{Type}$ be a
family. Suppose
\[
\Gamma \vdash P_1 : \Sigma n:\mathbb{N}.\,\mathsf{Vec}(n)
\qquad\text{and}\qquad
\Gamma \vdash P_2 : \Sigma m:\mathbb{N}.\,\mathsf{Vec}(m).
\]
Then:
\begin{enumerate}
\item[\textnormal{(Typing)}] $\Gamma \vdash \mathsf{Merge}(P_1,P_2) \;:\; 
      \big(\Sigma n:\mathbb{N}.\,\mathsf{Vec}(n)\big) \otimes 
      \big(\Sigma m:\mathbb{N}.\,\mathsf{Vec}(m)\big)$.
\item[\textnormal{(Operational)}] If $P_1 \Rightarrow (n,v)$ and $P_2 \Rightarrow (m,w)$
      by (possibly probabilistic) reduction, then
      \[
      \mathsf{Merge}(P_1,P_2) \;\Rightarrow\; (n,v)\ \Vert\ (m,w),
      \]
      modulo the associativity/commutativity congruence of $\mathsf{Merge}$.
\item[\textnormal{(Denotational)}] In the presheaf topos with distribution monad
      $\mathsf{Dist}$, if $\llbracket P_1 \rrbracket \in \mathsf{Dist}\!\big(\Sigma n.\,\mathsf{Vec}(n)\big)$
      and $\llbracket P_2 \rrbracket \in \mathsf{Dist}\!\big(\Sigma m.\,\mathsf{Vec}(m)\big)$
      are independent, then
      \[
      \llbracket \mathsf{Merge}(P_1,P_2) \rrbracket
      \;=\;
      \llbracket P_1 \rrbracket \otimes \llbracket P_2 \rrbracket
      \;\in\; \mathsf{Dist}\!\Big(\big(\Sigma n.\,\mathsf{Vec}(n)\big)\times\big(\Sigma m.\,\mathsf{Vec}(m)\big)\Big),
      \]
      where $\otimes$ is the product (independent) measure and the codomain
      identifies the tensor with the cartesian product object in the underlying topos.
\end{enumerate}
\end{lemma}

\begin{proof}[Sketch]
\emph{Typing.} By the $\mathsf{Merge}$ typing rule (tensor introduction),
if both branches share well-formed types, then
$\Gamma \vdash \mathsf{Merge}(P_1,P_2)$ has the tensor of those types.

\emph{Operational.} Reduction of $\mathsf{Merge}$ proceeds by reducing either
branch; if both normalize to $(n,v)$ and $(m,w)$, the merged term normalizes to
their parallel composition, up to A/C congruence.

\emph{Denotational.} By the semantics of $\mathsf{Merge}$ as tensor and of
probability via $\mathsf{Dist}$, independent branches denote product measures.
Therefore the denotation of the merge is the independent product of the branch
distributions, as stated.
\end{proof}

\begin{corollary}[Pushforward of Merged Dependent Pairs]
\label{cor:merge-pushforward}
Let $P_1$ and $P_2$ be as in Lemma~\ref{lem:merge-sigma-joint}, and let
\[
\Phi : \big(\Sigma n:\mathbb{N}.\,\mathsf{Vec}(n)\big) \times 
       \big(\Sigma m:\mathbb{N}.\,\mathsf{Vec}(m)\big) \;\to\; B
\]
be a dependent function into some type $B$. Then the SPC term
\[
\Phi\big(\mathsf{Merge}(P_1,P_2)\big)
\]
is well-typed with type $B$, and its denotational semantics is the pushforward
measure of the product distribution:
\[
\llbracket \Phi(\mathsf{Merge}(P_1,P_2)) \rrbracket
\;=\; \Phi_\* \big(\,\llbracket P_1 \rrbracket \otimes \llbracket P_2 \rrbracket\,\big).
\]
\end{corollary}

\begin{proof}[Sketch]
\emph{Typing.} By Lemma~\ref{lem:merge-sigma-joint}, $\mathsf{Merge}(P_1,P_2)$
has type $(\Sigma n.\,\mathsf{Vec}(n)) \otimes (\Sigma m.\,\mathsf{Vec}(m))$,
which is isomorphic to the product
$(\Sigma n.\,\mathsf{Vec}(n)) \times (\Sigma m.\,\mathsf{Vec}(m))$. Thus
applying $\Phi$ yields a term of type $B$.

\emph{Operational.} Whenever $P_1$ reduces to $(n,v)$ and $P_2$ to $(m,w)$,
the merged term reduces to $(n,v)\Vert(m,w)$, and applying $\Phi$ reduces
further to $\Phi((n,v),(m,w))$.

\emph{Denotational.} Since $\llbracket \mathsf{Merge}(P_1,P_2)\rrbracket$ is the
product distribution, and $\Phi$ is interpreted as a measurable map,
functoriality of the distribution monad implies that its denotation is the
pushforward $\Phi_\*$. ∎
\end{proof}



\paragraph{Worked Example: Dependent Probabilistic Function.}

Consider a type $\mathbb{N}$ of natural numbers, and let $\mathsf{Vec}(n)$ denote
the type of vectors of length $n$ over some base type $A$. We show how SPC can
express a probabilistic program that first samples a natural number $n$ and then
returns a vector of that length.

\medskip
\textbf{Step 1: Sample a length.}  
Define a probabilistic choice over natural numbers:
\[
L \;=\; \mathsf{Choice}(p,\,0,\,1) : \mathbb{N}.
\]
Here, $L$ evaluates to $0$ with probability $p$ and to $1$ with probability
$1-p$.

\medskip
\textbf{Step 2: Dependent family of vectors.}  
We form a dependent function type:
\[
F : \Pi n:\mathbb{N}.\, \mathsf{Vec}(n).
\]
Given $n$, $F$ produces a vector of length $n$.

\medskip
\textbf{Step 3: Construct dependent pair.}  
We bind the probabilistic result of $L$ with a vector of matching length:
\[
P \;=\; (\,L,\; \mathsf{Sphere}(n{:}\mathbb{N}.\, \mathsf{Vec}(n))\,) 
   : \Sigma n:\mathbb{N}.\,\mathsf{Vec}(n).
\]

\medskip
\textbf{Step 4: Probabilistic extension.}  
If we evaluate $P$ operationally, we obtain:
\[
P \;\to\; \begin{cases}
(0,\; \mathsf{Vec}(0)) & \text{with probability } p,\\[4pt]
(1,\; \mathsf{Vec}(1)) & \text{with probability } 1-p.
\end{cases}
\]

\medskip
\textbf{Denotational semantics.}  
In the presheaf topos enriched with the distribution monad, the denotation is:
\[
\llbracket P \rrbracket \;=\;
p \cdot \delta_{(0,\;\mathsf{Vec}(0))} \;+\;
(1-p) \cdot \delta_{(1,\;\mathsf{Vec}(1))}.
\]

\medskip
\textbf{Interpretation.}  
This term demonstrates how $\Sigma$-types interact with $\mathsf{Choice}$:
the type of the second component (the vector) depends on the sampled value of
the first component (the length). SPC thus supports \emph{probabilistic dependent
structures}, where the sampled data determines the type of subsequent
computation.

\paragraph{Worked Example: Dependent Pairs under Merge.}

Suppose we model two independent random processes that each return a natural
number $n$ together with a dependent vector of length $n$.  

\medskip
\textbf{Step 1: Two independent probabilistic samplers.}  
Let
\[
L_1 = \mathsf{Choice}(p,\,0,\,1) : \mathbb{N},
\qquad
L_2 = \mathsf{Choice}(q,\,1,\,2) : \mathbb{N}.
\]
Here, $L_1$ yields $0$ or $1$, while $L_2$ yields $1$ or $2$.

\medskip
\textbf{Step 2: Dependent pairs.}  
We construct dependent pairs where the second component is a vector of the
chosen length:
\[
P_1 = (\,L_1,\, \mathsf{Vec}(L_1)\,) : \Sigma n:\mathbb{N}.\,\mathsf{Vec}(n),
\]
\[
P_2 = (\,L_2,\, \mathsf{Vec}(L_2)\,) : \Sigma m:\mathbb{N}.\,\mathsf{Vec}(m).
\]

\medskip
\textbf{Step 3: Merge composition.}  
We combine these two processes using \texttt{Merge}:
\[
M = \mathsf{Merge}(P_1, P_2).
\]

\medskip
\textbf{Step 4: Operational behavior.}  
Operationally, $M$ branches into the four possible outcomes:
\[
M \;\to\; \begin{cases}
(0,\mathsf{Vec}(0)) \;\;||\;\; (1,\mathsf{Vec}(1)) & \text{with prob.\ } p \cdot q, \\[4pt]
(0,\mathsf{Vec}(0)) \;\;||\;\; (2,\mathsf{Vec}(2)) & \text{with prob.\ } p \cdot (1-q), \\[4pt]
(1,\mathsf{Vec}(1)) \;\;||\;\; (1,\mathsf{Vec}(1)) & \text{with prob.\ } (1-p) \cdot q, \\[4pt]
(1,\mathsf{Vec}(1)) \;\;||\;\; (2,\mathsf{Vec}(2)) & \text{with prob.\ } (1-p)\cdot(1-q).
\end{cases}
\]

\medskip
\textbf{Step 5: Denotational interpretation.}  
In the presheaf topos with distribution monad, $\mathsf{Merge}$ corresponds to
the product measure of two independent distributions. Thus
\[
\llbracket M \rrbracket
= \llbracket P_1 \rrbracket \otimes \llbracket P_2 \rrbracket,
\]
which is the convex combination above, represented as a joint distribution over
dependent pairs.

\medskip
\textbf{Interpretation.}  
This example shows how SPC can reason about \emph{dependent random structures in
parallel}. Merge enforces independence at the semantic level (tensor product of
distributions), while $\Sigma$-types ensure that each component is well-typed
with respect to the dependent relation between the chosen number and the vector
constructed from it. The combination thus yields a joint dependent distribution.

\paragraph{Worked Example: Concatenation as a Pushforward.}

Let $\mathsf{Vec}:\mathbb{N}\to\mathsf{Type}$ and let
\[
P_1 : \Sigma n:\mathbb{N}.\,\mathsf{Vec}(n)
\qquad\text{and}\qquad
P_2 : \Sigma m:\mathbb{N}.\,\mathsf{Vec}(m)
\]
be the dependent probabilistic pairs from the previous example. Define the
dependent post-processing map
\[
\Phi \;:\;
\big(\Sigma n:\mathbb{N}.\,\mathsf{Vec}(n)\big)
\times
\big(\Sigma m:\mathbb{N}.\,\mathsf{Vec}(m)\big)
\;\longrightarrow\;
\Sigma k:\mathbb{N}.\,\mathsf{Vec}(k)
\]
by
\[
\Phi\big((n,v),(m,w)\big) \;\triangleq\; \big(n{+}m,\;\mathsf{concat}(v,w)\big),
\]
where $\mathsf{concat}:\mathsf{Vec}(n)\times\mathsf{Vec}(m)\to\mathsf{Vec}(n{+}m)$
is length-index respecting.

\medskip
\textbf{Typing.}
For any $(n,v)$ and $(m,w)$ with $v:\mathsf{Vec}(n)$ and $w:\mathsf{Vec}(m)$,
$\mathsf{concat}(v,w):\mathsf{Vec}(n{+}m)$; hence
$\Phi((n,v),(m,w)):\Sigma k.\,\mathsf{Vec}(k)$ with $k=n{+}m$.
By Cor.~\ref{cor:merge-pushforward}, $\Phi(\mathsf{Merge}(P_1,P_2))$ is
well-typed.

\medskip
\textbf{Operational behavior.}
If $P_1 \Rightarrow (n,v)$ and $P_2 \Rightarrow (m,w)$, then
\[
\mathsf{Merge}(P_1,P_2) \;\Rightarrow\; (n,v)\ \Vert\ (m,w)
\quad\text{and}\quad
\Phi\big(\mathsf{Merge}(P_1,P_2)\big)\;\Rightarrow\; \big(n{+}m,\;\mathsf{concat}(v,w)\big).
\]

\medskip
\textbf{Denotational semantics (general form).}
Write $\mu=\llbracket P_1 \rrbracket \in \mathsf{Dist}(\Sigma n.\,\mathsf{Vec}(n))$
and $\nu=\llbracket P_2 \rrbracket \in \mathsf{Dist}(\Sigma m.\,\mathsf{Vec}(m))$,
assumed independent. Then
\[
\llbracket \Phi(\mathsf{Merge}(P_1,P_2)) \rrbracket
\;=\;
\Phi_{\*}(\,\mu \otimes \nu\,)
\;\in\; \mathsf{Dist}\!\big(\Sigma k.\,\mathsf{Vec}(k)\big),
\]
i.e.\ the pushforward of the product measure along
$((n,v),(m,w)) \mapsto (n{+}m,\mathsf{concat}(v,w))$.

\medskip
\textbf{Binary-choice instance.}
Instantiate $P_1,P_2$ with
\[
L_1=\mathsf{Choice}(p,0,1),\quad
P_1=(L_1,\mathsf{Vec}(L_1)),
\qquad
L_2=\mathsf{Choice}(q,1,2),\quad
P_2=(L_2,\mathsf{Vec}(L_2)).
\]
Then the four outcomes and their probabilities are:
\[
\begin{array}{c|c}
\text{Outcome of } \Phi(\mathsf{Merge}(P_1,P_2))
& \text{Probability} \\
\hline
\big(0{+}1,\;\mathsf{concat}(\mathsf{Vec}(0),\mathsf{Vec}(1))\big)
& p\,q \\[2pt]
\big(0{+}2,\;\mathsf{concat}(\mathsf{Vec}(0),\mathsf{Vec}(2))\big)
& p\,(1-q) \\[2pt]
\big(1{+}1,\;\mathsf{concat}(\mathsf{Vec}(1),\mathsf{Vec}(1))\big)
& (1-p)\,q \\[2pt]
\big(1{+}2,\;\mathsf{concat}(\mathsf{Vec}(1),\mathsf{Vec}(2))\big)
& (1-p)\,(1-q)
\end{array}
\]
Equivalently,
\[
\llbracket \Phi(\mathsf{Merge}(P_1,P_2)) \rrbracket
= pq\,\delta_{(1,\_)} + p(1-q)\,\delta_{(2,\_)}
+ (1-p)q\,\delta_{(2,\_)} + (1-p)(1-q)\,\delta_{(3,\_)},
\]
where each Dirac mass is at the corresponding concatenated vector of the shown
length (underscores emphasize value dependence on $v,w$).

\medskip
\textbf{Takeaway.}
This instance illustrates Cor.~\ref{cor:merge-pushforward}: the **type-level
shape** (the length index) evolves by the deterministic map $k=n{+}m$ while the
**distribution** over shapes and values is exactly the pushforward of the joint
(independent) distribution under concatenation. The construction is general:
any dependent post-processing $\Phi$ (e.g.\ zipping, padding, folding) yields a
well-typed SPC term whose semantics is the corresponding pushforward measure.

\section{Operational Semantics}\label{sec:operational}

\subsection{$\beta$-Reduction and Pop}

\begin{align*}
\mathrm{Pop}(\mathrm{Sphere}(x{:}A.\,t),u) &\;\to\; t[u/x] \\
\mathrm{Nest}(t,u) &\;\to\; \mathrm{Pop}(t,u) \\
\mathrm{Choice}(p,t,u) &\;\to\; \begin{cases}
t & \text{with prob.\ } p \\
u & \text{with prob.\ } 1-p
\end{cases}
\end{align*}

\subsection{Merge as Parallel Composition}

\[
\mathrm{Merge}(t,u) \equiv \mathrm{Merge}(u,t),
\quad
\mathrm{Merge}(t,\mathrm{Merge}(u,v)) \equiv \mathrm{Merge}(\mathrm{Merge}(t,u),v)
\]

\subsection{Choice (Internal and Monadic Variants)}

\subparagraph{Option B (Distribution Monad).}
\[
\inferrule*[right=Dist-Form]
  {\Gamma \vdash A : \mathsf{Type}_i}
  {\Gamma \vdash \mathsf{Dist}(A) : \mathsf{Type}_i}
\]
\begin{mathpar}
\inferrule*[right=Choice-Dist]
  {\Gamma \vdash p : \mathsf{Prob} \\
   \Gamma \vdash t : A \\
   \Gamma \vdash u : A}
  {\Gamma \vdash \mathrm{Choice}(p,t,u) : \mathsf{Dist}(A)}
\end{mathpar}

\subsection{Equational Properties of Merge}

The \texttt{Merge} operator in SPC is intended to model parallel or 
nondeterministic composition. Its equational theory ensures that the operational, 
type-theoretic, and denotational perspectives coincide.

\paragraph{Commutativity and Associativity.}
At the syntactic level, \texttt{Merge} is treated modulo the laws of a commutative 
monoid (with unit $\mathsf{Skip}$, if introduced):
\[
\mathrm{Merge}(t,u) \;\equiv\; \mathrm{Merge}(u,t),
\qquad
\mathrm{Merge}(t,\mathrm{Merge}(u,v)) \;\equiv\; 
\mathrm{Merge}(\mathrm{Merge}(t,u),v).
\]
These identifications allow arbitrary \texttt{Merge}-trees to be flattened into 
multisets of branches.

\paragraph{Idempotence (optional law).}
If \texttt{Merge} is interpreted as nondeterministic disjunction, one may further 
impose
\[
\mathrm{Merge}(t,t) \;\equiv\; t,
\]
yielding a commutative idempotent monoid. However, when \texttt{Merge} is 
interpreted tensorially in a symmetric monoidal category, this law does not 
hold in general. We therefore treat idempotence as an admissible equational 
extension in the nondeterministic reading, but not in the tensorial semantics.

\paragraph{Congruence.}
\texttt{Merge} is a congruence with respect to SPC reduction:
\[
t \to t' \quad\implies\quad
\mathrm{Merge}(t,u) \to \mathrm{Merge}(t',u),
\]
and symmetrically for the right argument. This ensures that parallel branches 
reduce independently.

\paragraph{Distribution over Choice.}
Operationally, \texttt{Merge} distributes over \texttt{Choice} in the sense that
\[
\mathrm{Merge}(\mathrm{Choice}(p,t,u),v) 
\;\;\Rightarrow\;\;
\mathrm{Choice}\big(p,\mathrm{Merge}(t,v),\mathrm{Merge}(u,v)\big).
\]
This law ensures compositional alignment between parallel and probabilistic 
constructs. In the distribution-monad semantics, it corresponds to bilinearity 
of convex combinations.
\subsection{Equational Properties of Merge}

The \texttt{Merge} operator behaves as a commutative and associative 
combiner of terms, subject to the following equations:

\begin{align*}
\mathrm{Merge}(t,u) &\equiv \mathrm{Merge}(u,t) \quad &&\text{(commutativity)} \\
\mathrm{Merge}(t,\mathrm{Merge}(u,v)) &\equiv \mathrm{Merge}(\mathrm{Merge}(t,u),v) 
  \quad &&\text{(associativity)} \\
\mathrm{Merge}(t,t) &\equiv t \quad &&\text{(idempotence, optional)} \\
\mathrm{Merge}(t,\mathbf{0}) &\equiv t \quad &&\text{(neutral element, if defined)}
\end{align*}

Here $\mathbf{0}$ denotes an optional identity element for \texttt{Merge}, 
representing the absence of a branch. Together, these laws allow any nested 
or reordered use of \texttt{Merge} to be flattened into a canonical form.

\paragraph{Summary.}
Thus, the equational properties of \texttt{Merge} situate SPC in the general 
theory of symmetric monoidal categories. Depending on interpretation, 
\texttt{Merge} may behave as:
\begin{itemize}
  \item a commutative monoid (nondeterministic or set-theoretic reading),
  \item or a symmetric monoidal tensor (categorical reading).
\end{itemize}
Both views agree on commutativity, associativity, and congruence, while diverging 
on idempotence. This flexibility allows SPC to uniformly encode both algebraic 
nondeterminism and categorical concurrency.


\section{Denotational Semantics}\label{sec:denotational}

\subsection{Presheaf Topos Semantics}

The presheaf category $[\mathsf{Sphere}^{op}, \mathsf{Set}]$ forms a topos, providing:
\begin{itemize}
\item Subobject classifier (truth sphere).
\item Finite limits and colimits.
\item Exponentials.
\item Internal intuitionistic higher-order logic: propositions = subspheres; proofs = morphisms preserving truth.
\end{itemize}

\paragraph{Distribution monad on $\mathcal{E}$.}
For each object $X$ of $\mathcal{E}$, let $\mathsf{Dist}(X)$ denote the object of
\emph{finitely supported subprobability measures} on $X$ (presheafwise), with unit and bind:
\[
\eta_X : X \to \mathsf{Dist}(X)
\qquad
\mu^\sharp_{X,Y} : \mathsf{Dist}(X) \times (X \to \mathsf{Dist}(Y)) \to \mathsf{Dist}(Y)
\]
satisfying the monad laws (left/right unit and associativity). In syntax we write
$\mathsf{return} \equiv \eta$ and $\mathsf{bind} \equiv \mu^\sharp$.

\paragraph{Core clauses.}
\[
\begin{aligned}
\llbracket \mathrm{Sphere}(x{:}A.\,t) \rrbracket
&:\; \llbracket \Gamma \rrbracket \to \llbracket A \rrbracket \Rightarrow \llbracket B \rrbracket
\quad\text{(exponential in } \mathcal{E}\text{)} \\
\llbracket \mathrm{Pop}(t,u) \rrbracket
&=\; \mathsf{ev} \circ \langle \llbracket t \rrbracket , \llbracket u \rrbracket \rangle
\quad:\; \llbracket \Gamma \rrbracket \to \llbracket B \rrbracket \\
\llbracket \mathrm{Merge}(t,u) \rrbracket
&=\; \langle \llbracket t \rrbracket , \llbracket u \rrbracket \rangle
\quad:\; \llbracket \Gamma \rrbracket \to \llbracket A \times A \rrbracket
\end{aligned}
\]

\paragraph{Probabilities and Choice (Option B).}
We interpret $\mathsf{Prob}$ as the presheaf of reals in $[0,1]$ (constant presheaf suffices).
Given $p : \llbracket \Gamma \rrbracket \to [0,1]$ and $t,u : \llbracket \Gamma \rrbracket \to \llbracket A \rrbracket$,
the denotation of \texttt{Choice} is the \emph{convex mixture}:
\[
\llbracket \mathrm{Choice}(p,t,u) \rrbracket
\;=\;
\big(\lambda \gamma.\; p(\gamma)\cdot \delta_{\llbracket t \rrbracket(\gamma)} \;+\; (1-p(\gamma))\cdot \delta_{\llbracket u \rrbracket(\gamma)} \big)
\]
\subsection{Sphere/Pop as Exponentials}

In categorical semantics, $\lambda$-abstraction and application correspond to
the exponential adjunction $C^{A} \dashv - \times A$. The \texttt{Sphere}
operator in SPC generalizes abstraction by constructing a morphism
$\mathsf{Sphere}(f) : A \to B$ as an object inside the exponential
$B^{A}$. Operationally, entering a \texttt{Sphere} is equivalent to opening a
scope, while \texttt{Pop} corresponds to instantiating this scope with an
argument. Formally, given a context $\Gamma \vdash f : A \to B$, we define:
\[
  \Gamma \vdash \mathsf{Sphere}(f) : B^A
  \qquad
  \Gamma \vdash \mathsf{Pop}(\mathsf{Sphere}(f), a) : B.
\]
The $\beta$- and $\eta$-laws of SPC reproduce the expected adjunction laws of
the exponential. In this sense, \texttt{Sphere/Pop} are not merely syntactic
operators but geometric witnesses of the exponential structure in the
underlying cartesian closed category.

\subsection{Merge as Tensor}

While ordinary $\lambda$-calculus models computation in a cartesian closed
category (CCC), SPC extends the structure to a symmetric monoidal category,
where \texttt{Merge} is interpreted as the tensor $\otimes$. Two computations
$M$ and $N$ may be combined via
\[
  \Gamma \vdash M : A \quad \Delta \vdash N : B
  \quad \implies \quad \Gamma, \Delta \vdash M \otimes N : A \otimes B.
\]
This generalizes parallel composition, enabling the expression of
independent channels of computation. The associativity and symmetry laws of the
monoidal tensor ensure that Merge is well-behaved up to canonical isomorphism:
\[
  (M \otimes N) \otimes P \cong M \otimes (N \otimes P),
  \qquad
  M \otimes N \cong N \otimes M.
\]
Thus \texttt{Merge} internalizes concurrency and compositional independence in
a categorical framework, echoing both tensor products in linear logic and
parallel composition in process calculi.

\subsection{Choice as Convex Mixture}

To model probabilistic branching, SPC introduces \texttt{Choice}. At the
operational level, this corresponds to selecting among terms with given
probabilities. Categorically, it is modeled as a convex combination of morphisms.
For two computations $M, N : A$ and $p \in [0,1]$, we define:
\[
  \Gamma \vdash \mathsf{Choice}(p, M, N) : A
\]
with denotational interpretation
\[
  \llbracket \mathsf{Choice}(p, M, N) \rrbracket
  = p \cdot \llbracket M \rrbracket + (1-p) \cdot \llbracket N \rrbracket.
\]
More generally, \texttt{Choice} extends to finite distributions
$\{(p_i, M_i)\}_{i}$ with $\sum_i p_i = 1$. This interpretation equips SPC with
the structure of a convex algebra, enabling probabilistic reasoning to be
internalized into the type system.

\subsection{Distribution Monad Structure}

The probabilistic semantics of SPC is governed by the distribution monad
$\mathcal{D}$ \cite{giry1982categorical}. Objects $A$ are mapped to probability
measures $\mathcal{D}(A)$, and morphisms $f : A \to \mathcal{D}(B)$ are Markov
kernels. The monadic unit $\eta : A \to \mathcal{D}(A)$ injects a pure value as
a Dirac measure, while the Kleisli extension $f^\ast : \mathcal{D}(A) \to
\mathcal{D}(B)$ defines the semantics of probabilistic bind. In SPC, the
interpretation of \texttt{Choice} is given by convex linear combinations in
$\mathcal{D}$, and the \texttt{Merge} operator lifts to independent product
measures when applied to independent channels. Thus the distribution monad
provides the categorical backbone for SPC’s probabilistic semantics, ensuring
that randomness is compositional and well-behaved with respect to scope and
parallelism.

\section{Meta-Theory}\label{sec:meta}

We establish several key meta-theoretic results for SPC. Proofs follow standard
techniques from type theory and probabilistic semantics, adapted to the
Sphere/Pop and Merge/Choice constructs.

\paragraph{Preservation.}
If $\Gamma \vdash M : A$ and $M \to M'$, then $\Gamma \vdash M' : A$.  
This is proved by induction on typing derivations, with separate cases for
\texttt{Sphere/Pop}, \texttt{Merge}, and \texttt{Choice} reductions.

\paragraph{Progress.}
If $\emptyset \vdash M : A$, then either $M$ is a value, or there exists $M'$
such that $M \to M'$.  
Probabilistic branching requires extending the progress theorem to guarantee
that at least one reduct exists with nonzero probability.

\paragraph{Adequacy.}
Operational and denotational semantics agree: if $\emptyset \vdash M : A$,
then the distribution generated by reduction steps of $M$ coincides with its
denotation in the distribution monad. This ensures that probabilistic choice is
faithfully represented.

\paragraph{Independent Channels Lemma.}
If $M : A$ and $N : B$ are independent computations, then
\[
  \llbracket M \otimes N \rrbracket
  = \llbracket M \rrbracket \otimes \llbracket N \rrbracket,
\]
where the right-hand side denotes the product measure in $\mathcal{D}(A \times
B)$. This lemma formalizes the compositionality of probability in SPC, ensuring
that risks aggregate correctly across independent branches.

\paragraph{Consistency and normalization.}
Since SPC extends the Calculus of Constructions conservatively with monadic
probability and tensorial parallelism, strong normalization for the pure
fragment implies consistency of the extended system. We conjecture, but do not
yet prove, that SPC enjoys probabilistic normalization almost surely, as in
other probabilistic $\lambda$-calculi.

\subsection{Preservation and Substitution}

\begin{theorem}[Preservation]
If $\Gamma \vdash M : A$ and $M \to M'$, then $\Gamma \vdash M' : A$.
\end{theorem}

\begin{proof}[Sketch]
By induction on the typing derivation of $M$.  

\emph{Case (Pop).}  
If $M = \mathsf{Pop}(\mathsf{Sphere}(x{:}A.\,t), u)$, then by $\Pi$-Intro we
have $\Gamma, x:A \vdash t : B$ and by typing of application we know
$\Gamma \vdash u : A$. Substitution gives $\Gamma \vdash t[u/x] : B[u/x]$,
which is exactly the reduct $M'$.

\emph{Case (Merge).}  
If $M = M_1 \otimes M_2$, reduction preserves the tensor typing
$\Gamma \vdash M_1 : A$, $\Delta \vdash M_2 : B \;\Rightarrow\; \Gamma, \Delta \vdash M_1 \otimes M_2 : A \otimes B$.
Any step taken by $M_1$ or $M_2$ yields a well-typed pair, by induction.

\emph{Case (Choice).}  
If $M = \mathsf{Choice}(p, M_1, M_2)$, then both branches satisfy
$\Gamma \vdash M_1 : A$ and $\Gamma \vdash M_2 : A$. The reduct is either $M_1$
or $M_2$, hence $\Gamma \vdash M' : A$.  
(Alternatively, under a distributional typing discipline,
$\Gamma \vdash M : \mathcal{D}(A)$, and reduction samples from this
distribution. In either case typing is preserved.)

All other cases follow structurally. ∎
\end{proof}

\subsection{Progress and Normal Forms}

\begin{theorem}[Progress]
If $\cdot \vdash M : A$, then either $M$ is a value, or there exists $M'$ such that $M \to M'$.
\end{theorem}

\begin{proof}[Sketch]
By induction on typing.  

\emph{Values.}  
Atoms and $\mathsf{Sphere}$ abstractions are values.  
Tensors $V_1 \otimes V_2$ are values whenever $V_1, V_2$ are values.  

\emph{Pop.}  
If $M = \mathsf{Pop}(\mathsf{Sphere}(x{:}A.\,t), u)$ with $u$ a value, then $M$
reduces by $\beta$-reduction to $t[u/x]$.  

\emph{Choice.}  
If $M = \mathsf{Choice}(p, M_1, M_2)$, then by definition it can reduce to one
of $M_1$ or $M_2$ (or, in the distributional interpretation, to a probabilistic
mixture). Thus progress holds.  

\emph{Merge.}  
If $M = M_1 \otimes M_2$ and at least one of $M_1, M_2$ reduces, then $M$ reduces
modulo the congruence laws of associativity and commutativity. If both are values,
then $M$ is a value.  

Thus every well-typed closed term either is a value or reduces. ∎
\end{proof}


\subsection{Adequacy of Probabilistic Semantics}

\begin{theorem}[Adequacy of Denotational Semantics]
For any closed term $t$ of type $A$:
\begin{itemize}
\item If $t \to^* v$ where $v$ is a value, then $\llbracket t \rrbracket = \delta_{\llbracket v \rrbracket}$
  (soundness).
\item If $t$ involves probabilistic Choice, then for any measurable $h : \llbracket A \rrbracket \to \mathbb{R}$,
\[
\mathbb{E}_{\text{operational}}[h] \;=\; \mathbb{E}_{\llbracket t \rrbracket}[h].
\]
\end{itemize}
\end{theorem}

\begin{proof}[Sketch]
For the deterministic fragment, by induction on evaluation contexts and the
soundness of $\beta$-reduction. For Choice, adequacy follows from the definition
of $\llbracket \mathrm{Choice}(p,t,u) \rrbracket$ as the convex mixture
$p \delta_{\llbracket t \rrbracket} + (1-p)\delta_{\llbracket u \rrbracket}$,
together with the monad laws for sequencing.
\end{proof}

\subsection{Independent Channels Lemma}

\begin{lemma}[Independent Channels]
Let $R_i = \mathrm{Choice}(p_i, D, S)$ with independent probability parameters $p_i \in [0,1]$.
Then
\[
\Pr[\mathsf{FoldOr}(\mathrm{Merge}(R_1,\dots,R_n)) = D]
= 1 - \prod_{i=1}^n (1 - p_i).
\]
\end{lemma}

\begin{proof}[Sketch]
By semantic interpretation in the distribution monad:
$\llbracket R_i \rrbracket = p_i \delta_D + (1-p_i)\delta_S$.
Independence is given by the product structure of $\mathsf{Dist}$.
Folding with disjunction yields the stated probability.
\end{proof}

% Sphere/Pop adequacy
\[
\adjustbox{max width=\textwidth}{%
\begin{tikzcd}[column sep=small]
\Gamma \times A 
  \ar[r,"{\langle f , \pi_2 \rangle}"] 
  \ar[d,swap,"{\text{oper.}}"]
& (A \Rightarrow B) \times A 
  \ar[r,"\mathsf{ev}"] 
& B \\
\Gamma \times A 
  \ar[rr,swap,"\llbracket \mathrm{Pop}(\mathrm{Sphere}(x{:}A.\,t),u)\rrbracket"] 
&& B
\end{tikzcd}}
\]
\tag{Sphere/Pop Adequacy}
\label{diag:spherepop}

% Merge adequacy
\[
\adjustbox{max width=\textwidth}{%
\begin{tikzcd}[column sep=small]
t \times u 
  \ar[rr,"\cong"] 
  \ar[d,swap,"{\text{oper. flatten}}"] 
&& t \otimes u 
  \ar[d,"{\text{tensor}}"] \\
\mathrm{Merge}(t,u) 
  \ar[rr,equal] 
&& \mathrm{Merge}(t,u)
\end{tikzcd}}
\]
\tag{Merge Adequacy}
\label{diag:merge}

% Choice adequacy
\[
\adjustbox{max width=\textwidth}{%
\begin{tikzcd}[column sep=small]
\Gamma 
  \ar[r,"{(p,t,u)}"] 
  \ar[d,swap,"{\text{sampling}}"]
& [0,1] \times A^2
  \ar[r,"\mathsf{mix}"] 
& \mathsf{Dist}(A) \\
\Gamma 
  \ar[rr,swap,"\llbracket \mathrm{Choice}(p,t,u) \rrbracket"] 
&& \mathsf{Dist}(A)
\end{tikzcd}}
\]
\tag{Choice Adequacy}
\label{diag:choice}

% Independent Channels Lemma adequacy
\[
\adjustbox{max width=\textwidth}{%
\begin{tikzcd}[column sep=small]
\prod_i \mathsf{Dist}(O) 
  \ar[r,"\otimes"] 
  \ar[d,swap,"{\text{oper. Merge}}"]
& \mathsf{Dist}(O^n) 
  \ar[r,"{\mathsf{anyDoom}_\*}"] 
& \mathsf{Dist}(O) \\
\mathsf{Dist}(O) 
  \ar[rr,equal] 
&& \mathsf{Dist}(O)
\end{tikzcd}}
\]
\tag{Independent Channels Adequacy}
\label{diag:indepchannels}


\section{Historical Antecedents}\label{sec:history}

\subsection{Lambda Calculus and Type Theory}
The $\lambda$-calculus, introduced by Church in the 1930s \cite{church1940formulation}, 
established the principle of treating functions as first-class terms and substitution 
as the primary mechanism of computation. Its simply-typed and polymorphic extensions 
formed the basis of modern type systems. Martin-Löf’s dependent type theory in the 
1970s \cite{martinlof1975intuitionistic} deepened this foundation by permitting types 
to depend on terms, thus aligning syntax, proof, and semantics in a single formalism. 
These traditions provided the logical backbone on which the Spherepop Calculus inherits 
its $\mathsf{Sphere}$ (abstraction) and $\mathsf{Pop}$ (application) constructs.

\subsection{Categorical and Topos-Theoretic Foundations}
Category theory in the 1960s and 70s (Mac Lane \cite{maclane1963natural}, Lambek and Scott \cite{lambek1986categorical}, 
Lawvere \cite{lawvere1970quantifiers}) reinterpreted logic and type theory within the 
algebraic framework of functors, natural transformations, and monoidal products. 
Lawvere’s elementary toposes demonstrated how categorical universes can serve as 
foundations for constructive logic, while Lambek’s insights connected $\lambda$-calculus 
and intuitionistic logic to cartesian closed categories. Street’s higher-categorical work 
\cite{street1972two} extended these correspondences, providing tools for reasoning about 
substitution, coherence, and higher-order structure. SPC draws directly on these traditions 
by interpreting $\mathsf{Merge}$ as tensor and by embedding its semantics into presheaf toposes.

\subsection{Probabilistic Semantics and Monads}
The rise of probabilistic program semantics in the 1980s (Kozen \cite{kozen1981semantics}) 
and the categorical probability monad (Giry \cite{giry1982categorical}) marked a turning 
point: probability could be treated not as an external measure but as a compositional effect. 
The monadic treatment unified branching, randomness, and expectation with the categorical 
structure of computation. In the 2000s–2010s, categorical probability theory and probabilistic 
programming languages (Goodman and Stuhlmüller \cite{goodman2014design}, 
Gordon et al. \cite{gordon2014probabilistic}, van de Meent et al. \cite{van2018introduction}) 
refined these approaches, incorporating them into type theory and proof assistants. 
SPC inherits this strand in its treatment of $\mathsf{Choice}$ as convex mixture, grounded 
in the distribution monad.

\subsection{Concurrency and Merge Operators}
In parallel, the theory of concurrency advanced through Milner’s CCS and $\pi$-calculus 
\cite{milner1980calculus,milner1999communicating}, formalizing nondeterministic composition 
and process interaction. Category theory again provided the unifying abstraction, treating 
concurrency via symmetric monoidal categories and tensorial composition. SPC borrows this 
lineage for its $\mathsf{Merge}$ operator, capturing both the algebraic laws of nondeterminism 
(commutativity, associativity, optional idempotence) and the structural laws of tensor products 
in monoidal categories. This dual inheritance allows SPC to represent both nondeterministic 
branching and true concurrency within a uniform syntax.

\subsection{Timeline of Antecedents}
\begin{itemize}
  \item \textbf{1930s--40s: Lambda Calculus.}
    Church introduces the untyped and simply-typed $\lambda$-calculus 
    \cite{church1940formulation}, laying the foundation for functional 
    abstraction and substitution.

  \item \textbf{1960s: Category Theory Foundations.}
    Mac Lane formalizes monoidal categories and coherence theorems \cite{maclane1963natural}.
    Lambek articulates the Curry--Howard--Lambek correspondence \cite{lambek1986categorical}.

  \item \textbf{1970s: Dependent Type Theory and Toposes.}
    Martin-Löf develops intuitionistic dependent type theory \cite{martinlof1975intuitionistic}.
    Lawvere introduces elementary toposes \cite{lawvere1970quantifiers}, embedding logic in category theory.

  \item \textbf{1980s: Higher Categories and Probabilistic Semantics.}
    Street formalizes 2-categories and coherence structures \cite{street1972two}.
    Coquand and Huet define the Calculus of Constructions \cite{coquand1988calculus}.
    Kozen introduces semantics of probabilistic programs \cite{kozen1981semantics}.
    Giry develops the probability distribution monad \cite{giry1982categorical}.

  \item \textbf{1990s: Calculus of Inductive Constructions.}
    Coq and related systems integrate inductive types with dependent type theory 
    \cite{bertot2004interactive}.

  \item \textbf{2000s--2010s: Probabilistic Type Theory and Programming.}
    Growth of probabilistic programming languages 
    \cite{goodman2014design,gordon2014probabilistic,van2018introduction}; 
    categorical probability theory connects monads, measure theory, and logic 
    \cite{jacobs2015new,fritz2020synthetic}.

  \item \textbf{2020s: Toward SPC.}
    Probabilistic dependent type theory (e.g. Adams et al. \cite{adams2020foundations}).
    Categorical unification of probabilistic semantics, concurrency, 
    and constructive logic inspires the Spherepop Calculus.
\end{itemize}


\section{Positioning of SPC}\label{sec:positioning}

\subsection{Comparison with $\lambda$-calculus}

Like the $\lambda$-calculus, SPC is founded on abstraction and application,
via \texttt{Sphere} and \texttt{Pop}. The $\beta$-reduction rule is preserved
verbatim, guaranteeing that SPC inherits the expressivity of higher-order
functional computation. Where SPC diverges is in the explicit visualization of
scope as a \emph{geometric sphere}, and in the enrichment with \texttt{Merge}
and \texttt{Choice}, which lie outside the traditional $\lambda$ syntax.

\subsection{Comparison with $\pi$-calculus}

Milner’s $\pi$-calculus introduced communication and concurrency via channels.
SPC’s \texttt{Merge} operator is reminiscent of nondeterministic choice and
parallel composition in process calculi, but with a type-directed and
categorical interpretation: \texttt{Merge} is modeled as a tensor in a monoidal
category, rather than as an interleaving of processes. SPC thus aligns with
structural concurrency but at a higher level of abstraction.

\subsection{Comparison with Probabilistic $\lambda$-calculus}

Kozen’s semantics of probabilistic programs and subsequent probabilistic
$\lambda$-calculi introduced random choice as a primitive. SPC’s
\texttt{Choice} operator situates itself within this tradition, but it offers
two levels of semantics: an \emph{internal} version, where Choice directly
returns a value of type $A$, and a \emph{monadic} version, where Choice
produces a distribution in $\mathsf{Dist}(A)$. The latter connects SPC
directly to Giry’s distribution monad and modern probabilistic programming
semantics, ensuring compositionality.

\subsection{Comparison with CoC/CIC}

SPC extends the Calculus of Constructions with probabilistic and parallel
operators. Its type system supports $\Pi$- and $\Sigma$-types, but enriches
the language with monoidal and stochastic constructs, which are not native to
CoC or CIC. In this sense, SPC is a \emph{probabilistic, monoidal calculus of
constructions}.

\subsection{Summary and Observation}

SPC can thus be positioned as:
\begin{itemize}
  \item A higher-order functional calculus (like $\lambda$-calculus),
  \item With categorical concurrency (akin to $\pi$-calculus, but tensorial),
  \item And integrated probabilistic semantics (via Choice and distribution monads),
  \item Embedded within dependent type theory (as in CoC).
\end{itemize}
This synthesis marks SPC as a novel framework: not merely a probabilistic
extension of $\lambda$-calculus, nor a categorical gloss on process calculi,
but a unification of \emph{functional abstraction, concurrency, and
probability} within one geometrically motivated language.

\paragraph{Discussion.}
The comparison highlights how SPC synthesizes elements from prior traditions while
introducing new primitives. From $\lambda$-calculus it inherits higher-order
abstraction and $\beta$-reduction; from $\pi$-calculus it adapts the idea of
parallel composition, but reinterprets it categorically as \texttt{Merge}.
Probabilistic choice is integrated more deeply than in probabilistic $\lambda$-calculi,
via both internal and distribution-monad semantics. Dependent types and presheaf
semantics provide a constructive foundation absent in the other calculi. Finally,
SPC’s visualization of scope as spheres establishes a geometric, intuitive model of
computation that distinguishes it from purely syntactic approaches.

\begin{observation}[Positioning of SPC]
The Spherepop Calculus inherits the expressive power of $\lambda$-calculus through
\texttt{Sphere}/\texttt{Pop}, while extending it with \texttt{Merge} as a categorical
parallel operator and \texttt{Choice} as a probabilistic primitive.
Unlike $\pi$-calculus, SPC does not rely on channel-based interleaving, but instead
models concurrency via tensorial semantics in a monoidal category.
Compared to probabilistic $\lambda$-calculi, SPC admits both an internal and a
distribution-monad interpretation of choice, ensuring compositional semantics.
Finally, by embedding all constructs into dependent type theory and presheaf topos
semantics, SPC achieves a synthesis of functional abstraction, concurrency, and
probability within a single constructive framework.
\end{observation}

\begin{table}[h]
\centering
\setlength{\tabcolsep}{4pt}      % tighter cell padding (default 6pt)
\renewcommand{\arraystretch}{1.12}
\footnotesize
\begin{adjustbox}{max width=\textwidth}
\begin{tabularx}{\textwidth}{|>{\raggedright\arraybackslash}p{3.0cm}|C|C|C|Y|}
\hline
\textbf{Feature} & $\lambda$-calc. & $\pi$-calc. & Prob.\ $\lambda$ & \textbf{SPC} \\
\hline
Core abstraction & $\lambda x.\,t$ & Processes/channels & $\lambda x.\,t$ & $\mathrm{Sphere}(x{:}A.\,t)$ \\
\hline
Application & $t\,u$ & Channel comm. & $t\,u$ & $\mathrm{Pop}(t,u)$ \\
\hline
Scope model & Parentheses & Channel scope & Parentheses & Spheres (bubble scope) \\
\hline
Concurrency & Not built-in & Parallel comp. & Not built-in & Yes: $\mathrm{Merge}(t,u)$ \\
\hline
Probabilistic choice & Not built-in & Not built-in & $\mathrm{flip}(p)$, rand. prims & $\mathrm{Choice}(p,t,u)$ \\
\hline
Dependent types & No & No & Rare ext. & Yes (prob.\ CoC) \\
\hline
Categorical semantics & CCC (cart.\ closed cat.) & Proc.\ cats. & Monads for prob. & Presheaf topos + Dist.\ monad \\
\hline
Evaluation & $\beta$-red. & Proc.\ interaction & $\beta$-red.\ + sampling & $\beta$-red.\ + merge + samp. \\
\hline
Scope visualization & Not emphasized & Channel diagrams & Not emphasized & Spheres, nesting \\
\hline
\end{tabularx}
\end{adjustbox}
\caption{Comparison of SPC with classical computational calculi.}
\end{table}


\subsection{Expressivity Proposition}

\begin{proposition}[Expressivity of SPC]
The Spherepop Calculus strictly subsumes:
\begin{enumerate}
  \item the simply-typed $\lambda$-calculus (via \texttt{Sphere}/\texttt{Pop}),
  \item probabilistic $\lambda$-calculus (via \texttt{Choice} with internal or monadic semantics),
  \item and a fragment of $\pi$-calculus corresponding to nondeterministic parallel composition (via \texttt{Merge}).
\end{enumerate}
\end{proposition}

\begin{proof}[Sketch]
(1) Every $\lambda$-term can be translated homomorphically:
\[
\lambda x.\,t \mapsto \mathrm{Sphere}(x{:}A.\,t),
\qquad
t\,u \mapsto \mathrm{Pop}(t,u).
\]
Subject reduction and $\beta$-equivalence are preserved by the operational semantics.

(2) For probabilistic $\lambda$-calculus, constructs such as
$\mathbf{flip}(p); t; u$ translate directly to
$\mathrm{Choice}(p,t,u)$ in SPC. In the monadic variant, the distribution
semantics of SPC subsumes the denotational semantics of Kozen and Giry.

(3) The parallel composition fragment of $\pi$-calculus, restricted to
independent nondeterministic branches, is captured by $\mathrm{Merge}(t,u)$,
with the tensorial semantics in SPC corresponding to categorical models of
process interleaving. While SPC does not encode full name-passing mobility, it
subsumes this nondeterministic fragment.

Strictness follows because SPC adds dependent types and presheaf topos
semantics, neither present in $\lambda$, probabilistic $\lambda$, nor $\pi$.
Thus, SPC properly extends each calculus.
\end{proof}

\section{Translations into SPC}\label{sec:translations}

We present compositional translations from three source calculi into the
Spherepop Calculus (SPC): simply-typed $\lambda$-calculus, probabilistic
$\lambda$-calculus, and a nondeterministic parallel fragment of $\pi$-calculus.
Each translation is type-directed, preserves typing, and enjoys operational
correspondence with SPC reduction.

\subsection{Notation}

We write $\mathcal{T}_\lambda$, $\mathcal{T}_{\mathrm{prob}\,\lambda}$,
$\mathcal{T}_\pi$ for the translations. A source typing context $\Gamma$ is
mapped pointwise to the SPC context with the same variable|type pairs.
Types are mapped homomorphically to SPC types (base types unchanged; function
types map to $\Pi$-types).

\subsection{Translation $\mathcal{T}_\lambda$ (Simply-typed $\lambda$)}

\paragraph{Source.}
\[
\begin{aligned}
\text{Types:}&\quad \tau ::= \alpha \mid \tau \to \tau \\
\text{Terms:}&\quad e ::= x \mid \lambda x{:}\tau.\,e \mid e\,e
\end{aligned}
\]

\paragraph{Type translation.}
\[
\llbracket \alpha \rrbracket = \alpha, \qquad
\llbracket \tau_1 \to \tau_2 \rrbracket = \Pi x{:}\llbracket \tau_1 \rrbracket.\ \llbracket \tau_2 \rrbracket
\ \text{ (with } x\notin \mathrm{FV}).
\]

\paragraph{Term translation (compositional).}
\[
\begin{aligned}
\mathcal{T}_\lambda(x) &= x \\
\mathcal{T}_\lambda(\lambda x{:}\tau.\,e) &= \mathrm{Sphere}(x{:}\llbracket \tau \rrbracket.\ \mathcal{T}_\lambda(e)) \\
\mathcal{T}_\lambda(e_1\,e_2) &= \mathrm{Pop}(\mathcal{T}_\lambda(e_1),\ \mathcal{T}_\lambda(e_2))
\end{aligned}
\]

\paragraph{Preservation.}
If $\Gamma \vdash e : \tau$ then $\Gamma \vdash \mathcal{T}_\lambda(e) : \llbracket \tau \rrbracket$.

\paragraph{Operational correspondence.}
If $e \to_\beta e'$ then $\mathcal{T}_\lambda(e) \to \mathcal{T}_\lambda(e')$ in SPC (one $\beta$-step).

\paragraph{Worked example.}
\[
(\lambda x{:}\alpha.\,x)\ a
\quad\mapsto\quad
\mathrm{Pop}(\mathrm{Sphere}(x{:}\alpha.\,x),\ a)
\ \to\ a.
\]

\subsection{Translation $\mathcal{T}_{\mathrm{prob}\,\lambda}$ (Probabilistic $\lambda$)}

\paragraph{Source (internal choice).}
Extend $e$ with $\,\mathbf{choice}(p, e_1, e_2)\,$ ($p\in[0,1]$). Typing: if
$\Gamma\vdash e_i:\tau$ and $\Gamma\vdash p:\mathsf{Prob}$ then
$\Gamma\vdash \mathbf{choice}(p,e_1,e_2):\tau$.

\paragraph{Type translation.}
As in $\lambda$; additionally $\llbracket \mathsf{Prob} \rrbracket = \mathsf{Prob}$.

\paragraph{Term translation.}
\[
\begin{aligned}
\mathcal{T}_{\mathrm{prob}\,\lambda}(\text{pure } e)
  &= \mathcal{T}_\lambda(e) \\
\mathcal{T}_{\mathrm{prob}\,\lambda}(\mathbf{choice}(p,e_1,e_2))
  &= \mathrm{Choice}(p,\ \mathcal{T}_\lambda(e_1),\ \mathcal{T}_\lambda(e_2)).
\end{aligned}
\]

\paragraph{Preservation.}
If $\Gamma \vdash e : \tau$ in the probabilistic $\lambda$-calculus (internal choice),
then $\Gamma \vdash \mathcal{T}_{\mathrm{prob}\,\lambda}(e) : \llbracket \tau \rrbracket$ in SPC
(Choice--A rule).

\paragraph{Operational correspondence.}
A single probabilistic step
$\mathbf{choice}(p, e_1, e_2) \Rightarrow e_1$ w.p.\ $p$ and $\Rightarrow e_2$ w.p.\ $1-p$
corresponds to
$\mathrm{Choice}(p,\mathcal{T}_\lambda(e_1),\mathcal{T}_\lambda(e_2))$ stepping to the corresponding branch
with the same probabilities.

\paragraph{Worked example.}
\[
\mathbf{choice}\!\left(\tfrac12,\,(\lambda x{:}\alpha.\,x)\ a,\,(\lambda x{:}\alpha.\,a)\ x\right)
\mapsto
\mathrm{Choice}\!\left(\tfrac12,\ \mathrm{Pop}(\mathrm{Sphere}(x{:}\alpha.\,x), a),\ \mathrm{Pop}(\mathrm{Sphere}(x{:}\alpha.\,a), x)\right).
\]

\paragraph{Monadic variant (Option B).}
If the source calculus types $\mathbf{choice}$ as $\mathsf{Dist}(\tau)$, use:
\[
\mathcal{T}_{\mathrm{prob}\,\lambda}^{\mathsf{mon}}(\mathbf{choice}(p,e_1,e_2))
  = \mathrm{Choice}(p,\ \mathcal{T}_\lambda(e_1),\ \mathcal{T}_\lambda(e_2)) : \mathsf{Dist}(\llbracket \tau \rrbracket),
\]
and compose with $\mathsf{return}/\mathsf{bind}$ in SPC as required. Preservation and adequacy follow from the
distribution-monad semantics.

\subsection{Translation $\mathcal{T}_\pi$ (Nondeterministic $\pi$-fragment)}

\paragraph{Source fragment.}
Processes $\;P,Q ::= 0 \mid P \mid Q \mid (\nu a)P \mid a(x).P \mid \overline{a}\langle b\rangle.P\;$,
with \emph{nondeterministic parallel} reduction where independent branches may proceed.
We translate the \emph{branching/parallel} structure and independent, non-communicating
subprocesses. (Name passing and synchronization are not encoded in this fragment.)

\paragraph{Type and term carriers.}
Assume a ground outcome type $O$ with $\mathsf{Safe},\mathsf{Doom}:O$. Let each
process $P$ denote an SPC term $t_P:O$ (the observable outcome). Communication-free
parallelism is mapped to \texttt{Merge}; restriction $(\nu a)$ is ignored in this fragment
or reflected in types as an abstract scope (no effect on $O$).

\paragraph{Translation.}
\[
\begin{aligned}
\mathcal{T}_\pi(0) &= \mathsf{Safe} : O \\[3pt]
\mathcal{T}_\pi(P \mid Q) &= \mathrm{Merge}(\mathcal{T}_\pi(P),\,\mathcal{T}_\pi(Q)) : O \\[3pt]
\mathcal{T}_\pi((\nu a)P) &= \mathcal{T}_\pi(P) : O \quad\text{(in the non-communicating fragment)} \\[3pt]
\mathcal{T}_\pi(\overline{a}\langle b\rangle.P) &= \mathcal{T}_\pi(P) \\
\mathcal{T}_\pi(a(x).P) &= \mathcal{T}_\pi(P)
\end{aligned}
\]
Intuitively, we erase name-passing but preserve \emph{parallel branching structure}.
When the observable is a disjunctive property (e.g.\ ``any branch dooms?''), SPC’s
$\mathrm{Merge}$ correctly aggregates branches.

\paragraph{Preservation (fragment).}
If $P$ and $Q$ are well-formed processes in the fragment, then
$\mathcal{T}_\pi(P \mid Q)$ is well-typed in SPC and has the same outcome type $O$.

\paragraph{Operational correspondence (fragment).}
If $P \mid Q \Rightarrow P' \mid Q$ by reducing an independent branch, then
$\mathrm{Merge}(\mathcal{T}_\pi(P),\mathcal{T}_\pi(Q)) \to
 \mathrm{Merge}(\mathcal{T}_\pi(P'),\mathcal{T}_\pi(Q))$
modulo Merge congruence. Similarly for the right branch.

\paragraph{Worked examples.}
\[
\begin{aligned}
&(0 \mid 0) \ \mapsto\ \mathrm{Merge}(\mathsf{Safe},\mathsf{Safe}) \\
&(0 \mid (\nu a)0) \ \mapsto\ \mathrm{Merge}(\mathsf{Safe},\mathsf{Safe}) \\
&P \mid Q \mid R\ \mapsto\ \mathrm{Merge}(\mathcal{T}_\pi(P),\mathrm{Merge}(\mathcal{T}_\pi(Q),\mathcal{T}_\pi(R))) \equiv
\mathrm{Merge}(\mathcal{T}_\pi(P),\mathcal{T}_\pi(Q),\mathcal{T}_\pi(R))\ \text{(flattened).}
\end{aligned}
\]

\subsection{Compositionality and Merge/Choice Interaction}

\paragraph{Flattening.}
SPC treats $\mathrm{Merge}$ as associative/commutative; thus the image of parallel
composition flattens canonically: $\mathcal{T}_\pi(P_1 \mid \cdots \mid P_n) =
\mathrm{Merge}(\mathcal{T}_\pi(P_1),\dots,\mathcal{T}_\pi(P_n))$.

\paragraph{Probabilistic branches.}
If a process or program contains probabilistic behavior (e.g.\ random guards),
translate local randomness to $\mathrm{Choice}$ and compose with $\mathrm{Merge}$.
Under independence, the doom-aggregation law holds:
\[
\Pr[\text{doom}] \;=\; 1 - \prod_{i=1}^n (1-p_i) \quad
\text{for } \mathrm{Merge}\big(\mathrm{Choice}(p_i,\mathsf{Doom},\mathsf{Safe})\big)_{i=1}^n.
\]

\subsection{Summary of Translation Properties}

\begin{itemize}
\item (\textbf{Typing}) Each translation preserves typing derivations into SPC.
\item (\textbf{Operational}) Source one-step reductions map to SPC steps
      (or equalities modulo Merge congruence) with matching probabilities for choice.
\item (\textbf{Adequacy}) For the monadic version, denotations commute with translation:
      $\llbracket \mathcal{T}_{\mathrm{prob}\,\lambda}(e)\rrbracket
      = \mathcal{T}^\mathcal{E}(\llbracket e \rrbracket)$, where
      $\mathcal{T}^\mathcal{E}$ is the induced functor on denotations.
\end{itemize}

\appendix

\section{Appendices}

\subsection{Context and Substitution Lemmas}

We recall the compositional translation $\mathcal{T}_\lambda$:
\[
\mathcal{T}_\lambda(x)=x,\qquad
\mathcal{T}_\lambda(\lambda x{:}\tau.\,e)=\mathrm{Sphere}(x{:}\llbracket\tau\rrbracket.\,\mathcal{T}_\lambda(e)),\qquad
\mathcal{T}_\lambda(e_1\,e_2)=\mathrm{Pop}(\mathcal{T}_\lambda(e_1),\,\mathcal{T}_\lambda(e_2)),
\]
and the type mapping $\llbracket\alpha\rrbracket=\alpha$, $\llbracket \tau_1\to\tau_2\rrbracket=\Pi x{:}\llbracket\tau_1\rrbracket.\,\llbracket\tau_2\rrbracket$ (with $x\notin \mathrm{FV}$).

\begin{lemma}[Context Lemma]
If $\Gamma \vdash e:\tau$ in STLC, then (the pointwise-mapped) $\Gamma \vdash \mathcal{T}_\lambda(e):\llbracket\tau\rrbracket$ in SPC.
\end{lemma}

\begin{proof}[Sketch]
By induction on the typing derivation of $e$. Variables are immediate. 
Abstractions use $\Pi$-Intro, applications use $\Pi$-Elim in SPC. The type mapping preserves well-formedness.
\end{proof}

\begin{lemma}[Substitution Lemma]
If $\Gamma,x{:}\tau\vdash e:\sigma$ and $\Gamma\vdash v:\tau$, then
\[
\Gamma \vdash \mathcal{T}_\lambda(e)[\,\mathcal{T}_\lambda(v)/x\,] : \llbracket\sigma\rrbracket.
\]
\end{lemma}

\begin{proof}[Sketch]
By induction on the structure of $e$.
The only nontrivial case is application of $\beta$ in SPC:
\[
\mathrm{Pop}(\mathrm{Sphere}(x{:}\llbracket\tau\rrbracket.\,\mathcal{T}_\lambda(e)),\,\mathcal{T}_\lambda(v))
\;\to\;
\mathcal{T}_\lambda(e)[\,\mathcal{T}_\lambda(v)/x\,],
\]
which is well-typed by $\Pi$-Intro/Elim and the IH.
\end{proof}

\subsection{Preservation Corollary}

\begin{corollary}[Preservation for $\mathcal{T}_\lambda$]
If $\Gamma \vdash e:\tau$ and $e\to_\beta e'$, then
$\Gamma \vdash \mathcal{T}_\lambda(e) : \llbracket\tau\rrbracket$ and 
$\mathcal{T}_\lambda(e) \to \mathcal{T}_\lambda(e')$ in SPC.
\end{corollary}

\begin{proof}[Sketch]
Single $\beta$-step in STLC corresponds to a single $\beta$-step in SPC by the translation and the Substitution Lemma.
\end{proof}

\subsection{End-to-End Worked Example}

\paragraph{Source term (prob.\ $\lambda$-calculus, internal choice)}
Let $\tau=\alpha$ be a base type and consider
\[
e \;=\; \mathbf{choice}\!\left(p,\; (\lambda x{:}\alpha.\,x)\,a,\; (\lambda x{:}\alpha.\,a)\,x_0 \right),
\]
with constants $a:\alpha$, variable $x_0:\alpha$, and $p:\mathsf{Prob}$.

Typing (source): $\cdot,p:\mathsf{Prob},a:\alpha,x_0:\alpha \;\vdash\; e:\alpha$.

\paragraph{Translation to SPC}
\[
\mathcal{T}_{\mathrm{prob}\,\lambda}(e)
\;=\;
\mathrm{Choice}\!\Big(
p,\;
\underbrace{\mathrm{Pop}(\mathrm{Sphere}(x{:}\alpha.\,x),\,a)}_{\text{branch }t},\;
\underbrace{\mathrm{Pop}(\mathrm{Sphere}(x{:}\alpha.\,a),\,x_0)}_{\text{branch }u}
\Big)
\;:\; \alpha.
\]

\paragraph{Operational trace in SPC}
\[
\begin{aligned}
&\mathrm{Choice}\!\big(p,\ \mathrm{Pop}(\mathrm{Sphere}(x{:}\alpha.\,x),\,a),\ \mathrm{Pop}(\mathrm{Sphere}(x{:}\alpha.\,a),\,x_0)\big) \\
&\qquad\to
\begin{cases}
\mathrm{Pop}(\mathrm{Sphere}(x{:}\alpha.\,x),\,a) \to a & \text{with prob.\ } p,\\
\mathrm{Pop}(\mathrm{Sphere}(x{:}\alpha.\,a),\,x_0) \to a & \text{with prob.\ } 1-p.
\end{cases}
\end{aligned}
\]
Hence in either branch the normal form is $a:\alpha$.

\paragraph{Denotational interpretation (Option B: distribution monad)}
In the presheaf topos $\mathcal{E}=[\mathsf{Sphere}^{op},\mathsf{Set}]$ with distribution monad $\mathsf{Dist}$,
\[
\llbracket \mathrm{Choice}(p,t,u)\rrbracket
\;=\; p\cdot\delta_{\llbracket t\rrbracket} + (1-p)\cdot\delta_{\llbracket u\rrbracket}.
\]
But $\llbracket t \rrbracket=\llbracket u \rrbracket=\llbracket a \rrbracket$, so
\[
\llbracket \mathcal{T}_{\mathrm{prob}\,\lambda}(e) \rrbracket
= p\cdot\delta_{\llbracket a \rrbracket} + (1-p)\cdot\delta_{\llbracket a \rrbracket}
= \delta_{\llbracket a \rrbracket}.
\]
Thus the denotation is a Dirac mass at $a$, matching the operational normal form.

\paragraph{Variant with distinct outcomes.}
Let $\alpha=O=\{\mathsf{Safe},\mathsf{Doom}\}$ and consider
\[
e'=\mathbf{choice}(p,\ \mathsf{Doom},\ \mathsf{Safe}) \quad:\ O.
\]
Then
\[
\mathcal{T}_{\mathrm{prob}\,\lambda}(e')=\mathrm{Choice}(p,\mathsf{Doom},\mathsf{Safe}),
\]
operationally sampling $\mathsf{Doom}$ w.p.\ $p$ and $\mathsf{Safe}$ w.p.\ $1-p$, and
\[
\llbracket \mathcal{T}_{\mathrm{prob}\,\lambda}(e') \rrbracket
= p\,\delta_{\mathsf{Doom}} + (1-p)\,\delta_{\mathsf{Safe}}.
\]

\subsection{Aggregation via Merge and Choice}

For independent hazards $R_i=\mathrm{Choice}(p_i,\mathsf{Doom},\mathsf{Safe})$,
define the SPC term
\[
T_n \;=\; \mathrm{FoldOr}\big(\mathrm{Merge}(R_1,\dots,R_n)\big).
\]
Under independence,
\[
\Pr[T_n=\mathsf{Doom}] \;=\; 1 - \prod_{i=1}^n (1-p_i),
\]
as established in the Independent-Channels Lemma; denotationally, this is the pushforward of the product distribution through the Boolean
disjunction $\mathsf{anyDoom}:O^n\to O$.

\bibliographystyle{plain}
\bibliography{references}

\end{document}
